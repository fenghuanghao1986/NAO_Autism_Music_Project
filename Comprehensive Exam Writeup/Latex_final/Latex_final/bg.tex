\chapter{Background} \label{chap:background}
\graphicspath{{./bg/fig/}}

%\section{Preliminaries} In this section, we briefly summarize the mathematical
%background needed for the rest of the proposal. Readers may refer to \cite{agg88, bar94,
%fau01,fau96,fle92,har00,luo96,may93,wol02} for further treatment on these subjects.
%

\section{Introduction}

In this chapter, we provide some necessary technical background and briefly review some of the
existing relevant techniques in the field of computer vision. These techniques can be classified in
five groups 1) panoramic imaging; 2) motion estimation; 3) 2-D super-resolution mosaicing; 4)
3-D-reconstruction; and 5) error analysis. Selected examples for further treatment of these topics
can be found in \cite{mil77, soa98, soa98p, sze99, tia96, tom92, ura89, wax87, wen87}.


\section{Panoramic Cameras}

Panoramic cameras allow us to capture a scene video over a larger field of view. This is specially
important for navigation, mapping, tracking, and obstacle avoidance in the application of unmanned
vehicles. Currently the design of existing panoramic sensors involve:

\bi

\item {\bf Mirrors:} This kind of panoramic imaging, using a combination of mirror(s) placed in an
arranged configuration in front of the camera, has been extensively explored by many researchers
\cite{bak98,bru99,gro01,nay97-2,nay97-1,pri01,swa01,tan02,yok98}. Panoramic stereo imaging systems
also have been studied in \cite{ben01,bha98,ish92,nay01,nen98,pel00}.

\item {\bf Wide-angle Lenses:} Use of very wide-angle lenses for an ordinary camera makes it
suitable for panoramic imaging \cite{gre86,gro01,zhu00,zhu99}.

\item {\bf Multiple Cameras:} These provide panoramic images at much higher image resolution.
Analyzing these types of panoramic sensors is not as straight forward as the previous ones
\cite{bak01,bak00,fer00,fir03-2,gro01,mcc97,nal98,neg01,ple03,swa00}. In this proposal, we
introduce a generalized camera model which is suitable for the modeling of such panoramic sensors
\cite{fir03-2}.

\ei

The first two groups of panoramic sensors are powered by a single ordinary camera. These deal with
a single optical center which makes them simpler to design and calibrate. Unlike the last group,
they suffer from limited image resolution and thus accuracy for many visual tasks that rely on
high spatial resolution.

\section{2-D Super-Resolution Mosaicing} Mosaicing and super resolution imaging are two ways
to combine information from multiple frames in video sequences. Mosaicing displays the information
from multiple frames in a single panoramic image, while super-resolution exploits regions appearing
in multiple frames to improve resolution and quality by reducing the noise \cite{zom00,zom01-1}.
Applying a super-resolution technique to 2-D mosaics improves the resolution while preserving the
geometry. The main assumption in super-resolution imaging is that the low-resolution images are the
projections of a high-resolution image onto the image plane. The goal is to find the
high-resolution image which fits this model. Presenting this in mathematical language:

Given $N$ camera images $\{I_N^{(n)}\}^N_{n=1}$, find the the high-resolution image $I$, which
minimizes the the error function:

\beq \D E(I)=\sum_{n=1}^N\left\|P_n(I)-I_N^{(n)}\right\|^2 \eeq where $P_n(I)$ is the projection of
$I$ onto the sampling grid of image $I_N^{(n)}$.

\section{Summary}
In this chapter, we summarized certain definitions, notations and mathematical background needed
for the rest of the thesis. These included geometric and photometric constraints, which are the
core of most motion and 3-D reconstruction techniques. In the next chapter, we propose a
generalization to the camera model that makes it suitable for the application of interest in our
research.

%While the geometric constraints are tightly related to the projection and the homogeneous
%transformation between two camera positions, the photometric constraints come from the underlying
%physics of the image formation.
