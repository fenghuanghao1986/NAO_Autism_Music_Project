\chapter{Design of a Log-Gabor Filter Bank} \label{apx_logGabor}
A Gabor, or log-Gabor, filter bank does not form an orthogonal basis
set and hence there is no unique or ideal arrangement of the
filters. Thus, the design of a filter bank is somewhat of an art but
the following should give you some guidelines. One aim is to produce
a filter bank that provides even coverage of the section of the
spectrum that you wish to represent. This can be achieved by making
the overlap of the filter transfer functions sufficiently large so
that when one sums the individual transfer functions the net result
is an even coverage of the spectrum. Thus, every point in the
spectrum ends up being represented equally in the final result. For
computational reasons one wants to achieve this even coverage of the
spectrum with a minimal number of filters.

A second aim, which conflicts with the first, is to ensure the
outputs of the individual filters in the bank are as independent as
possible. The whole aim of applying the filter bank is to obtain
information about our signal, if a filter's outputs are highly
correlated with those of its neighbors then we have an inefficient
arrangement of filters that do not provide as much information as
they should. To achieve independence of output the filters should
have minimal overlap of their transfer functions.

Thus the transfer functions of our filters should have the minimal
overlap necessary to achieve fairly even spectral coverage.

Here are the parameters you have to decide on, several are
interdependent.

The minimum and maximum frequencies you wish to cover. The filter
bandwidth to use. The scaling between centre frequencies of
successive filters. The number of filter scales. The number of
filter orientations to use. The angular spread of each filter.
Maximum frequency The maximum frequency is set by the wavelength of
the smallest scale filter, this is controlled by the parameter
minWaveLength. The smallest value you can sensibly use here is the
Nyquist wavelength of 2 pixels, however at this wavelength you will
get considerable alaising and I prefer to keep the minimum value to
3 pixels or above.

Minimum frequency The minimum frequency is set by the wavelength of
the largest scale filter. This is implicitly defined once you have
set the number of filter scales (nscale), the scaling between centre
frequencies of successive filters (mult), and the wavelength of the
smallest scale filter.

   maximum wavelength = minWavelength * mult^(nscale-1)

   minimum frequency = 1 / maximum wavelength

Filter bandwidth The filter bandwidth is set by specifying the ratio
of the standard deviation of the Gaussian describing the log Gabor
filter's transfer function in the log-frequency domain to the filter
center frequency. This is set by the parameter sigmaOnf . The
smaller sigmaOnf is the larger the bandwidth of the filter. I have
not worked out an expression relating sigmaOnf to bandwidth, but
empirically a sigmaOnf value of 0.75 will result in a filter with a
bandwidth of approximately 1 octave and a value of 0.55 will result
in a bandwidth of roughly 2 octaves.

Scaling between centre frequencies Having set a filter bandwidth one
is then in a position to decide on the scaling between centre
frequencies of successive filters (mult). It is here one has to play
off the conflicting demands of even spectral coverage and
independence of filter output.

Here is a table of values, determined experimentally, that result in
the minimal overlap necessary to achieve fairly even spectral
coverage.

 sigmaOnf  .85   mult 1.3
 sigmaOnf  .75   mult 1.6     (bandwidth ~1 octave)
 sigmaOnf  .65   mult 2.1
 sigmaOnf  .55   mult 3       (bandwidth ~2 octaves)

The number of filter orientations This, in conjunction with the
angular spread of each filter, specifies the resolution of the
orientation information you obtain from the filters. I have
traditionally used six orientations. The angular spread of each
filter Here again one plays off the demands of even spectral
coverage and independence of filter output. The angular interval
between filter orientations is fixed by the number of filter
orientations. In the frequency domain the spread of 2D log-Gabor
filter in the angular direction is simply a Gaussian with respect to
the polar angle around the centre. The angular overlap of the filter
transfer functions is controlled by the ratio of the angular
interval between filter orientations and the standard deviation of
the angular Gaussian spreading function. Within the code this ratio
is controlled by the parameter dThetaOnSigma. A value of
dThetaOnSigma = 1.5 results in approximately the minimum overlap
needed to get even spectral coverage.
