\chapter{The Iterative Closest Point algorithm} \label{apx_icp}

The Iterative Closest Point (ICP) algorithm has become the dominant
method for aligning three-dimensional models based purely on the
geometry, and sometimes color, of the meshes. The algorithm is
widely used for registering the outputs of 3D scanners, which
typically only scan an object from one direction at a time. ICP
starts with two meshes and an initial guess for their relative
rigid-body transform, and iteratively refines the transform by
repeatedly generating pairs of corresponding points on the meshes
and minimizing an error metric. The general framework of the
algorithm is represented in Table \ref{tab_icp}. Generating the
initial alignment may be done by a variety of methods, such as
tracking scanner position, identification and indexing of surface
features \cite{Faugeras86,Stein92}, ``spin-image'' surface
signatures \cite{Johnson97, Huber01}, computing principal axes of
scans \cite{Dorai97}, exhaustive search for corresponding points
\cite{Chen99}, or user input.
\begin{table}[tbp]
\begin{tabular}{|l|}
\hline
\\
 begin $E' \leftarrow + \infty$; \\
(Rot,Trans) $\leftarrow$ In Initialize-Alignment(Scene,Model);\\
~~~repeat\\
~~~~~~ $E \leftarrow E'$;\\
~~~~~~ Aligned-Scene $\leftarrow$
Apply-Alignment(Scene,Rot,Trans);\\
~~~~~~ Pairs $\leftarrow$ Return-Closest-Pairs(Aligned-Scene,Model);\\
~~~~~~ (Rot,Trans,$E'$) $\leftarrow$
Update-Alignment(Scene,Model,Pairs,Rot,Trans); \\
~~~ Until $|E'- E| < Threshold$ \\
~~~ return (Rot,Trans); \\
end\\
\hline
\end{tabular}
\caption{The iterative closest point algorithm.}\label{tab_icp}
\end{table}

Since the introduction of ICP by Chen and Medioni \cite{Chen91} and
Besl and McKay \cite{besl92}, many variants have been introduced on
the basic ICP concept. According to \cite{Rusinkiewicz_01}, these
variants affect one of six stages of the algorithm:

\begin{enumerate}
\item Selection of some set of points in one or both meshes.
\item Matching these points to samples in the other mesh.
\item Weighting the corresponding pairs appropriately.
\item Rejecting certain pairs based on looking at each pair
individually or considering the entire set of pairs.
\item Assigning an error metric based on the point pairs.
\item Minimizing the error metric.
\end{enumerate}

Rusinkiewicz \cite{Rusinkiewicz_thesis} has constructed a high-speed
ICP algorithm with consideration of the accuracy of the final answer
by combining some of the variants of the algorithm. The
implementation (C++ code) of this algorithm is available on the web
(http://www.cs.princeton.edu/gfx/proj/trimesh2/). We wrote a wrapper
to use this implementation of ICP in Matlab environment for 3D
facial image registration.
