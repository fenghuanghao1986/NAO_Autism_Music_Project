\chapter{Conclusion and Future Research Direction}

\section{Summary and Contributions}
In this dissertation, we have introduced a multi-modal scheme, i.e.,
2-D (texture) + 3-D (shape), for face modeling and recognition.
Recently, with advancements in computer vision technology and
capturing devices, scanning or constructing a 3-D dense model of the
face is easy. One of the main approaches to improve the face
recognition technology is to use 3-D data. By using the 3-D data,
some of the known problems that affect the performance of face
recognition such as pose variations and illumination variations can
be easily handled. On the other hand, 2-D face recognition
algorithms have been investigated for few decades and is in a mature
stage. However, the 2-D and the 3-D data have their own limitations
and we believe that only a multi-modal scheme can provide a robust
solution for the problem of face recognition. In this regard, we
have developed a multi-modal system and addressed various issues in
building such a system. In particular, we have studied and provided
solutions for the following problems: 2-D and 3-D facial landmark
points extraction, 3-D face shape matching, 2-D and 3-D face
modeling and recognition, fusion at the score level and experiments
on public databases (FRGC V2.0, 3-D Gavab, and UM face databases.)

We have developed algorithms for 2-D and 3-D face recognition. For
3-D face recognition, we have developed a novel algorithm based on
ridge images in the facial range images. For multi-modal face
recognition, we have introduced a technique based on Attributed
Relation Graph (ARG) that represent both the 3-D (shape) and 2-D
(texture) data in a single model. The model is a geometric graph
model with nodes and vertices such that the nodes of the graph
represent a set of landmark points (extracted by an improved active
shape model technique developed in Chapter three and Appendix A) and
the edges are defied based on Delaunay triangulation. A set of
attributes are extracted from the shape and texture data using the
log-Gabor filters and assigned to each node of the graph. In
addition, a set of geometric features are extracted from the edges
of the graph (mutual relations). The graph models (i.e., the ARGs)
are matched by calculating the similarity between the 2-D and 3-D
attributes as well as the mutual relations. The matching scores are
fused using two different techniques including the Dempster-Shafer
theory of evidence and the weighted sum rule. The ARG model has the
capability of integrating both the 2-D and 3-D data in a single
model. However, in case where the 2-D and 3-D are not registered, we
cannot integrate the 3-D information in the same graph model.
Therefore, we have used the algorithm based on ridge images (Chapter
four) to handle the 3-D face recognition. Then, we fuse the results
of 2-D face recognition based on ARGs and the results of 3-D face
recognition based on ridge images. We summarize the major
contributions of this research as follows:

\bi
\item
We have improved the Active Shape Model approach for 2-D facial
features extraction from color images. We have presented solutions
to solve some of the limitations of the Active Shape Model approach.

\item
We have developed an algorithm for 3-D facial features extraction
from range data (the inner corners of the eyes and tip of the nose.)
Compared to 2-D facial feature extraction, extracting facial
features from 3-D range images is more difficult. The main
difficulty with extracting facial landmarks from 3-D data is the
lack of texture. We used the extracted facial features for the
initial alignment of the ridge images during the matching process.

\item
We have developed an algorithm for 3-D face modeling and recognition
based upon ridge images. The ridge lines on the range image data
carry the most important distinguishing information of the 3-D face
and have high potential for face recognition. For matching the ridge
images of two faces (probe and gallery), the Hausdorff and the
Iterative Closest Points methods have been utilized. By using the
ridge images for shape matching, the computational complexity of 3-D
face matching have been reduced by two orders of magnitude.

\item
We have developed an algorithm for multi-modal (3-D + 2-D) face
recognition based on attributed relational graphs (ARG). The nodes
of the graph represent the locations of the facial landmark points.
A set of attributes are extracted from the shape and the texture of
the face using log-Gabor filters and are assigned to each node of
the graph. Also, a set of features that define the geometric
relations between the edges of the graph are extracted and used in
the representation to improve the performance rate of face
recognition.

\item We have developed a fusion technique based on the Dempster-Shafer theory
of evidence for fusion at the score level.

\item
We have evaluated the performance of our developed algorithms and
techniques for multi-modal face recognition using various databases
such the FRGC v2.0 face database, the 3-D Gavab face database, and
the University of Miami (UM) face database. As we have shown through
the thesis, we have achieved promising results using the developed
techniques.

\ei
\section{Future Work}
Face recognition is an important area of research that is
continuously progressing. Possible future developments, expansions,
and improvements of our presented algorithms in this thesis are as
follows:

\bi
\item
Improving the developed method for extracting the 3-D facial
features (the two inner corners of the eyes and the tip of the
nose.) Statistical modeling approaches have shown to be successful
in extracting facial features from 2-D textured images. Applying a
statistical approach such as the Active Shape Model technique to
select the best facial features from candidate points extracted by
thresholding the Gaussian curvature is a promising approach.

\item Expanding the current technique for 3-D facial landmark extraction or
developing a new technique to extract more than three facial
landmarks from the 3-D data (either range images or stereo-based
reconstructed data.) Extracting 3-D facial landmarks is one of the
most challenging problems that requires more attention and research.

\item
Improving the Active Shape Model technique for facial landmark
extraction by utilizing nonlinear models such as Kernel PCA or
manifold learning. This makes the process of landmark points
extraction under pose variations in 2-D facial images more robust.

\item
Extending the ARG approach to recognize faces with expressions. In
order to handle facial expressions, a good approach can be sub-graph
matching. In fact, instead of using the whole graph for matching,
the ARG graph can be matched partially (e.g., the upper part of the
face including the eyes, the eyebrows and the nose).

\item
Improving the technique for extracting ridge images for recognition.
For example, by using preprocessing techniques, ridge images can be
refined and as a result a set of smooth ridges could be extracted
and used for 3-D face matching.

\item
Besides the extracted ridge points in this dissertation, there is
another set of ravine points that can be extracted using the
$k_{min}$ principal curvature. It will be interesting to extract and
use these points either alone or in conjunction with the other ridge
points (extracted by thresholding the $k_{max}$ principal curvature)
for 3-D face recognition. \ei
