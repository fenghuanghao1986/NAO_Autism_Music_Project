\documentclass[12pt, a4paper]{article}
\setlength{\oddsidemargin}{0.5cm}
\setlength{\evensidemargin}{0.5cm}
\setlength{\topmargin}{-1.6cm}
\setlength{\leftmargin}{0.5cm}
\setlength{\rightmargin}{0.5cm}
\setlength{\textheight}{24.00cm} 
\setlength{\textwidth}{15.00cm}
\parindent 0pt
\parskip 5pt
\pagestyle{plain}

\title{Research Proposal}
\author{Huanghao Feng}
\date{Jan 2017}

\newcommand{\namelistlabel}[1]{\mbox{#1}\hfil}
\newenvironment{namelist}[1]{%1
\begin{list}{}
    {
        \let\makelabel\namelistlabel
        \settowidth{\labelwidth}{#1}
        \setlength{\leftmargin}{1.1\labelwidth}
    }
  }{%1
\end{list}}

\begin{document}
\maketitle

\begin{namelist}{xxxxxxxxxxxx}
\item[{\bf Title:}]
	Xylo-Bot: An Automated Music Teaching Robot Platform \\
	for Children with Autism and Beyond
\item[{\bf Author:}]
	Huanghao Feng
\item[{\bf Supervisor:}]
	Professor Mohammad Mahoor
\item[{\bf Degree:}]
	Ph.D.
\end{namelist}

\section*{Background} 
Autism is a general term used to describe a spectrum of complex developmental
brain disorders causing qualitative impairments in social interaction and results in
repetitive and stereotyped behaviors. Currently one in every 88 children in the United
States are diagnosed with ASD and government statistics suggest the prevalence rate of
ASD is increasing 10-17 percent annually \cite{Fetch2002}. Children with ASD experience deficits in
appropriate verbal and nonverbal communication skills including motor control, emotional
facial expressions, turn-taking and eye gaze attention \cite{RobotPlaymate2002}. Currently, clinical work such as Applied
Behavior Analysis (ABA) \cite{RollingRobot2002, MobileRobotic2002} has focused on teaching individuals with ASD
appropriate social skills in an effort to make them more successful in social situations \cite{Behavioral1964}.
With the concern of the growing number of children diagnosed with ASD, there is a high
demand for finding alternative solutions such as innovative computer technologies and/or
robotics to facilitate autism therapy. Therefore, research into how to design and use modern
technology that would result in clinically robust methodologies for autism intervention is
vital.\\

In social human interaction, non-verbal facial behaviors (e.g. facial expressions,
gaze direction, and head pose orientation, etc.) convey important information between
individuals. For instance, during an interactive conversation, the peer may regulate their
facial activities and gaze directions actively to indicate the interests or boredom. However,
the majority of individuals with ASD show the lack of exploiting and understanding these
cues to communicate with others. These limiting factors have made crucial difficulties for
individuals with ASD to illustrate their emotions, feelings and also interact with other
human beings. Studies have shown that individuals with autism are much interested to
interact with machines (e.g. computers, iPad, robots, etc.) than humans \cite{SocialInteract2003}. In this regard,
in the last decade several studies have been conducted to employ machines in therapy
sessions and examine the behavioral responses of people with autism. These studies have
assisted researchers to better understand, model and improve the social skills of individuals
on the autism spectrum.\\

This proposal presents the hypothesis and potential methodology of a study that aimed to design a
entertaining humanoid-robot music therapy/teaching-like sessions for capturing, modeling and 
enhancing the social skills of children with Autism. In particular we mainly focus on: gaze 
direction; joint attention; turn-taking; motor control; stress handling; music pitch recognition;
and music emotion understanding, investigate how the ASD and Typically Developing (TD) children employ 
their knowledge for interacting with the robot using music language.\\

\section*{Research question} Now state explicitly the hypothesis you aim to
test. Make references to the items listed in the Reference section
that back up your arguments for why this is a reasonable
hypothesis to test, for example the work of Knuth~\cite{knuth}.
Explain what you expect will be accomplished by undertaking this
particular project.  Moreover, is it likely to have any other
applications?\\
Music play and practice requires 

 
\section*{Method}
In this section you should outline how you intend to go
about accomplishing the aims you have set in the previous
section. Try to break your grand aims down into small,
achievable tasks. Try to estimate how long you will
spend on each task, and draw up a timetable for each
sub-task.

\section*{Software and Hardware Requirements}
Outline what your specific requirements will be with regard
to software and hardware, but note that any special requests
might need to be approved by your supervisor and the Head of
Department.

Overall, you should aim to produce roughly a two page document
(and certainly no more than four pages)
outlining your plan for the year.

\begin{thebibliography}{9}
\bibitem{knuth} D. E. Knuth. {\em The \TeX~book.}\/ Addison-Wesley,
Reading, Massachusetts, 1984.
\bibitem{lamport} L. Lamport. {\em \LaTeX~: A Document Preparation
System}.\/ Addison-Wesley, Reading, Massachusetts, 1986.
\bibitem{ken} Ken Wessen, Preparing a thesis using \LaTeX~, private
communication, 1994.
\bibitem{lamport2} L. Lamport. Document Production: Visual
or Logical, {\em Notices of the Amer. Maths. Soc.},\/ Vol. 34,
1987, pp. 621-624.
\end{thebibliography}


\end{document}

