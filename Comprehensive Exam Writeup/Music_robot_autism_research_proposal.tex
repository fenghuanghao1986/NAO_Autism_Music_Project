\documentclass[12pt]{report}
\setlength{\oddsidemargin}{0.5cm}
\setlength{\evensidemargin}{0.5cm}
\setlength{\topmargin}{-1.6cm}
\setlength{\leftmargin}{0.5cm}
\setlength{\rightmargin}{0.5cm}
\setlength{\textheight}{24.00cm} 
\setlength{\textwidth}{15.00cm}
\parindent 0pt
\parskip 5pt
\pagestyle{plain}

\title{Research Proposal}
\author{Huanghao Feng}
\date{Jan 2017}

\newcommand{\namelistlabel}[1]{\mbox{#1}\hfil}
\newenvironment{namelist}[1]{%1
\begin{list}{}
    {
        \let\makelabel\namelistlabel
        \settowidth{\labelwidth}{#1}
        \setlength{\leftmargin}{1.1\labelwidth}
    }
  }{%1
\end{list}}

\begin{document}
\maketitle

\begin{namelist}{xxxxxxxxxxxx}
\item[{\bf Title:}]
	Xylo-Bot: An Automated Music Teaching Robot Platform \\
	for Children with Autism and Beyond
\item[{\bf Author:}]
	Huanghao Feng
\item[{\bf Supervisor:}]
	Professor Mohammad Mahoor
\item[{\bf Degree:}]
	Ph.D.
\end{namelist}

\section*{Background} 
Autism is a general term used to describe a spectrum of complex developmental
brain disorders causing qualitative impairments in social interaction and results in
repetitive and stereotyped behaviors. Currently one in every 88 children in the United
States are diagnosed with ASD and government statistics suggest the prevalence rate of
ASD is increasing 10-17 percent annually \cite{Fetch2002}. Children with ASD experience deficits in
appropriate verbal and nonverbal communication skills including motor control, emotional
facial expressions, turn-taking, and eye gaze attention \cite{RobotPlaymate2002}. Currently, clinical work such as Applied
Behavior Analysis (ABA) \cite{RollingRobot2002, MobileRobotic2002} has focused on teaching individuals with ASD
appropriate social skills in an effort to make them more successful in social situations \cite{Behavioral1964}.
With the concern of the growing number of children diagnosed with ASD, there is a high
demand for finding alternative solutions such as innovative computer technologies and/or
robotics to facilitate autism therapy. Therefore, research into how to design and use modern
technology that would result in clinically robust methodologies for autism intervention is
vital.\\

In social human interaction, non-verbal facial behaviors (e.g. facial expressions,
gaze direction, and head pose orientation, etc.) convey important information between
individuals. For instance, during an interactive conversation, the peer may regulate their
facial activities and gaze directions actively to indicate the interests or boredom. However,
the majority of individuals with ASD show the lack of exploiting and understanding these
cues to communicate with others. These limiting factors have made crucial difficulties for
individuals with ASD to illustrate their emotions, feelings and also interact with other
human beings. Studies have shown that individuals with autism are much interested to
interact with machines (e.g. computers, iPad, robots, etc.) than humans \cite{SocialInteract2003}. In this regard,
in the last decade several studies have been conducted to employ machines in therapy
sessions and examine the behavioral responses of people with autism. These studies have
assisted researchers to better understand, model and improve the social skills of individuals
on the autism spectrum.\\

This proposal presents the hypothesis and potential methodology of a study that aimed to design a
entertaining humanoid-robot music therapy/teaching-like sessions for capturing, modeling and 
enhancing the social skills of children with Autism. In particular, we mainly focus on gaze 
direction; joint attention; turn-taking; motor control; music pitch recognition;
and music emotion understanding, investigate how the ASD and Typically Developing (TD) children employ 
their knowledge for interacting with the robot using music language.\\

\section*{Research Questions}
It is obvious that multiple physical and mental activities are involved during music
play and practice. That makes it perfect to have music activity in human-robot interaction for
children with autism due to the deficits they are suffering from. \\

Eye gaze and joint attention are required a significant amount during music instrument practice
and play. Participants will be required to not only have eye contact with the music
agent but also have joint attention with the music instrument at the same time. This will be one of
the main activity among the entire experimental sessions, and will be evaluated at the end.
Music teaching scenario creates a meaningful back and forth communication experience. This allows
participants to have better understanding of turn-taking skill and be able to adapt this 
in their daily life.\\

A good thing about using a percussion instrument is that requires the player to have better
motor control in order to deliver a perfect clear note with a proper strike. The strike motion
need some practice that would allow the participants be able to adopt motor control technique
in other activities of their life.\\

Listen to someone's favorite music can be cheerful and relax. However, learning instrument 
play can be stressful, and repetitive practice could be tedious especially in early age. This 
type of intra-personal feeling changes could be detectable by using affective computing. 
Emotion recognition and representation can be difficult for children with ASD, that makes 
it more eager to find out how do they feel inside during music activities. Music is also
a great 'language' of representing emotions, it would also be interesting to see if children
with ASD would be able to recognize the emotion behind certain piece of melody.

 
\section*{Experimental Session Design}
A novel interactive human-robot music teaching system design will be designed in 
the next 2 years. In order to make the robot play the xylophone properly, several things need 
to be done. First is to find a proper xylophone with correct timber; 
second, we have to arrange the xylophone in the proper position in front of the robot 
to make it visible and be reachable to play; third, finalize what are the activities 
should be included in experimental sessions; finally, design the intelligent music system for NAO.\\

\paragraph{Session Design Concept:}
Intervention sessions will be designed to accomplish the target questions. Due to children
concentration time, total activity time in each session should be well controlled between 
30 to 40 minutes and the total length of each session will be under one hour including rest
time by participants request. In order to make music practice more entertaining which can 
keep participants attracted and focused, all tasks will be represented as a 'challenge' format.
Different levels/difficulties of activities will be designed to fulfill the music teaching
concept after all. \\
\subparagraph{Baseline Session: }Participants will be asked to follow all the 
instructions contained in all the practices from the following intervention session 
including single bar strike, multiple bars strike, half song play and the whole
song play. We decided to choose a very popular kid's song "Twinkle Twinkle Little Star" 
for this specific session due to how well-known this song is in almost 
everyone's childhood.\\

\subparagraph{Intervention Sessions: }These sessions will be assigned to ASD group
particularly and include a single strike with color hints, multiple strikes with
colors hint, half-song practice and whole song practice. In this part, a customized 
song will be used through the rest of the sessions by the request of the individual. In the second
half of this intervention session set, single/multiple strikes were also covered
before the half/full song practice in order to have participants use the
color matching technique during the high level music play due to a lack of professional
music background knowledge. In addition, starting from session 2, a single strike
warm up practice was added before the formal music practice starts. This particular
practice was designed to have better motor control for ASD group so that
the robot is able to recognize notes properly and deliver the concept in telling the
difference between "making a sound" and "playing a musical note". \\

\subparagraph{Exit Session: }Both groups were assigned to go through the same steps
as the baseline session in choice of their own songs. We would like to see the 
difference between the two groups in learning a beloved song by their own.\\

\section*{Hardware and Software Requirements}
\subparagraph{NAO: A Humanoid Robot}
We are going to use a humanoid  robot called NAO which developed by Aldebaran Robotics in France. 
This robot can conduct most human behaviors. It also features an onboard multimedia 
system including four microphones for voice recognition and sound localization, 
two speakers for text-to-speech synthesis, and two HD cameras with maximum image 
resolution 1280 x 960 for online observation. NAO’s computer vision module includes 
facial and shape recognition units. By using the 
vision feature of the robot, it can see the instrument 
from its lower camera and be able to implement an eye-arm self-calibration 
system which allows the robot to have real-time micro-adjustment of its 
arm-joints in case of off positioning during music playing.\\

\subparagraph{Module Based Robot Music Teaching Platform:}A novel module-based robot-music 
teaching system will be designed in order to accomplish the mission. 
Three modules will be built in this intelligent system including module 1: eye-hand 
self-calibration micro-adjustment; module 2: joint trajectory generator; and 
module 3: real time performance scoring feedback.\\

\subparagraph{Module 1: Eye-hand Self-Calibration Micro-Adjustment}
Knowledge about the parameters of the robot's kinematic model is essential for 
tasks requiring high precision, such as playing the xylophone. While the kinematic 
structure is known from the construction plan, errors can occur, e.g., due to the 
imperfect manufacturing. After multiple rounds of testing, it was found the targeted angle chain 
of arms never actually equals the returned chain. We therefore are going to use a 
calibration method to accurately eliminate these errors.\\

To play the xylophone, the robot has to be able to adjust its motions according to
the estimated relative position of the instrument and the heads of the beaters it is 
holding. To estimate these poses, adopted in this thesis, we used a color-based technique.\\

The main idea is, based on the RGB color of the center blue bar, given a hypothesis 
about the instrument's pose, one can project the contour of the object's model into the 
camera image and compare them to actually observed contour. In this way, it is possible 
to estimate the likelihood of the pose hypothesis. By using this method, it allows
the robot to track the instrument with very low cost in real-time. \\

\subparagraph{Module 2: Joint Trajectory Generator}
Our system parses a list of hex-decimal numbers (from 1 to b) to obtain the sequence
of notes to play. It converts the notes into a joint trajectory using the beating
configurations obtained from inverse kinematics as control points. The timestamps
for the control points will be defined by the user in order to meet the experiment requirement.
The trajectory is then computed using Bezier interpolation in joint space by the
manufacturer-provided API and then sent to the robot controller for execution. In this
way, the robot plays in-time with the song.\\

\subparagraph{Module 3: Real-Time Performance Scoring Feedback}
The purpose of this system is to provide a real-life interaction experience using 
music therapy to teach kids social skills and music knowledge.  In this scoring 
system, two core features were designed to complete the task: 1) music detection;
2) intelligent scoring-feedback system.\\

Music, in the understanding of science and technology, can be considered as a combination 
of time and frequency. In order to make the robot detect a sequence of frequencies, we adopted the 
short-time Fourier transform (STFT) to this audio feedback system. This allows the robot to 
be able to understand the music played by users and provide the proper feedback as
a music teaching instructor.\\

In order to compare the detected notes and the target notes, we adopted an algorithm
which is normally used in information theory linguistics called Levenshtein Distance.
This algorithm is a string metric for measuring the difference between two sequences.\\

Based on the real life situation, we defined a likelihood margin for determining whether the result
is good or bad and where if the likelihood is greater than 66\% ~ 72\%, the system will consider it 
as a good result. This result will be passed to the accuracy calculation system to have the robot 
decide whether it needs to add more dosage to the practice.\\


\subparagraph{Electrodermal activity (EDA) Based Emotion Classification Approach} 
EDA has been used as an effective and reproducible electrophysiological method for 
investigating sympathetic nervous system function \cite{WearableDevice2016, AssociationBetween2013, SympatheticSkin1984, PrincipalComponent2000}. Note that the sympathetic nervous 
burst changes the skin conductance, which can be traced by analyzing the EDA signals\cite{SkinConduct2006, SympatheticSkin1981, DecodeChild2013}. The Q-sensor 
is a convenient wireless-based EDA device with no need for cables, boxes, or skin 
preparation. This device can track three types of data including EDA, temperature, 
and acceleration at the same time \cite{Validation2013}. It is worth mentioning that 
as of today, there has been no published work on emotion classification using the 
EDA signals collected by this dataset collected at the Georgia Institute of 
Technology \cite{DecodeChild2013}.\\

In this research, we will employ the C-Morlet wavelet function to process the acquired EDA 
signals, as it has been well used for time-frequency analysis of different bio-signals 
and classification \cite{MultKernel2016}. Using the C-Morlet mother wavelet, the real and imaginary wavelet coefficients are calculated at different scales. Then, the amplitude of these coefficients is calculated to provide the corresponding spectrogram. This spectrogram is then used as the feature space. More details can be found in \cite{hFeng18}.\\


\appendix

\baselineskip 0.21in
\bibliographystyle{plain}
\bibliography{ref}
\baselineskip 0.33in



\end{document}

