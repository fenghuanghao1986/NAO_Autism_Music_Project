\chapter{Xylo-Bot: A Interactive Human-Robot Music Teaching System Design} 
A novelty Interactive human-robot music teaching system design is presented in 
this chapter. In order to make robot play xylophone properly, several things need 
to be done before that. First is to find a proper xylophone with correct timber; 
second, we have to make the xylophone in a proper position in front of the robot 
that makes it to be seen properly and be reached to play; finally, design the 
intelligent music system for NAO. \\

\section{NAO: A Humanoid Robot}
We used a humanoid  robot called NAO developed by Aldebaran Robotics in France. 
NAO is 58 cm (23 inches) tall, with 25 degrees of freedom this robot 
can conduct most of the human behaviors. It also features an onboard multimedia 
system including, four microphones for voice recognition, and sound localization, 
two speakers for text-to-speech synthesis, and two HD cameras with maximum image 
resolution 1280 x 960 for online observation. As shown in Figure somewhere, these 
utilities are located in the middle of the forehead and the mouth area. NAO’s 
computer vision module includes facial and shape recognition units. By using the 
vision feature of the robot, that allows the robot be able to see the instrument 
from its lower camera and be able to do implement a eye-arm self-calibration 
system which allows the robot to have real-time micro-adjustment of its 
arm-joints in case of off positioning during music playing.\\

The robot arms have a length of approximately 31 cm. Each arm have five degrees 
of freedom and is equipped with the sensors to measure the position of each 
joint. To determine the pose of the instrument and the beaters' heads the robot 
analyzes images from the lower monocular camera located in its head, which has a 
diagonal field of view of 73 degree. These dimensions allows us to choose a 
proper instrument presented in next section.\\

Four microphones embedded on toy or NAO's head locations see figure somewhere. 
According the official Aldebaran documentation, these microphones has sensitivity 
of 20mV/Pa +/-3dB at 1kHz, and the input frequency range of 150Hz - 12kHz, data 
will be recorded as a 16 bits, 48000Hz, 4 channels wav file which meets the 
requirements for designing the online feedback audio score system.\\

\section{Accessories}
Due to the size of the toy xylophone which has been used in this study, several 
accessories have been designed and crafted using 3D printing and laser cut 
machines.\\

\subsection{Xylophone: A Toy for Music Beginner}

In this system we choose a Sonor Toy Sound SM soprano-xylophone with 11 sound 
bars of 2 cm in width. The instrument has a size of xx cm x xx cm x xx cm, 
including the resonateing body. The smallest sound bar is playable in an area of 
2.8 cm x 2 cm, the largest in an area of 4.8 cm x 2 cm. The instrument is 
diatonically tuned in C-Major/a-minor. The beaters/mallets, we use the pair which 
come with the xylophone with a modified 3D printed grips (details in next 
subsection) to allow the robot's hands to hold them properly. The mallets 
are approximately 21 cm in length include a head of 0.8 cm radius.\\ 
%(see figure somewhere with real instrument)

11 bars represent 11 different notes (11 frequencies) which covers 
approximate one and half octave scale starting from C6 to F7. \\
%(see figures or a table with different frequencies somewhere)

\subsection{Mallet Gripper Design}

According to NAO's hands size, we designed and 3D printed a pair of gripers to 
have the robot be able to hold the mallets properly. All dimensions can be found 
in figure somewhere.\\
%(attach both solidworkds and actural pics somewhere)

\subsection{Instrument Stand Design}

A wooden base has designed and laser cut to hold the instrument in a proper place 
in order to have the robot be able to play music. All dimensions can be found in 
figure somewhere below. (attach both actural and solidworks pics somewhere blow)\\

\section{Music Teaching System Design}
In this section, a novelty robot-music teaching system will be presented. Three 
modules will included in this intelligent system including eye-hand 
self-calibration real-time micro-adjustment, joint trajectory generator (in 
progress will not be presented in this version) and real time performance 
scoring feedback. (see a block diagram figure somewhere presenting the whole 
system)

\subsection{Module 1: Eye-hand Self-Calibration}
Knowledge about the parameters of the robot's kinematic model is essential for 
tasks requiring high precision such as playing the xylophone. While the kinematic 
structure is known from the construction plan, errors can occur, e.g., due to the 
imperfect manufacturing. After multiple times of test, the targeted angle chain 
of arms never equals to the returned chain in reality. We therefore use a 
calibration method to accurately eliminate these errors. 

\subsubsection{A. Color-Based Object Tracking}
To play the xylophone, the robot has to be able to adjust its motions according to
the estimated relative poses of the instrument and the heads of the beaters it is 
holding. The approach to estimating these poses which adopted in this thesis, we 
uses a color-based technique.\\
The main idea is, based on the RGB color of the center blue bar, given a hypothesis 
about the instrument's pose, one can project the contour of the object's model into the 
camera image and compare them to actually observed contour. In this way, it is possible 
to estimate the likelihood of the pose hypothesis. By using this method, it allows
the robot to track the instrument with very low cost in real-time.\\
%(see figure some where, need a sequence of screen shot to show hot it works, possibly
to show a flow chart regarding how to implement in the code)

\subsubsection{B. Calibration of Kinematic Parameters}
(In progress, will not present in this version. The idea is to use both positions 
of the instrument and beaters' heads to computes for each sound bar a suitable 
beating configuration for arm kinematics chain. Suitable means that the beater's 
head can be placed on the surface of the sound bar at the desired angle. From 
this configuration, the control points of a predefined beating motion are updated.) 

\subsection{Module 2: Joint Trajectory Generator}



\subsection{Module 3: Real-Time Performance Scoring Feedback}
The purpose of this system is to provide a back and forth interaction using music 
therapy to teach kid social skills and music knowledge.

\subsubsection{A. Short Time Fourier Transform}
The short-time Fourier transform (STFT) , is a Fourier-related 
transform used to determine the sinusoidal frequency and phase content of local 
sections of a signal as it changes over time.[1] In practice, the procedure for 
computing STFTs is to divide a longer time signal into shorter segments of equal 
length and then compute the Fourier transform separately on each shorter segment. 
This reveals the Fourier spectrum on each shorter segment. One then usually plots 
the changing spectra as a function of time.
In the discrete time case, the data to be transformed could be broken up into chunks 
or frames (which usually overlap each other, to reduce artifacts at the boundary). 
Each chunk is Fourier transformed, and the complex result is added to a matrix, which 
records magnitude and phase for each point in time and frequency. This can be expressed as:

${\displaystyle \mathbf {STFT} \{x[n]\}(m,\omega )\equiv X(m,\omega )=\sum _{n=-\infty }^{\infty }x[n]w[n-m]e^{-j\omega n}}$

likewise, with signal x[n] and window w[n]. In this case, m is discrete and $\omega$ 
is continuous, but in most typical applications the STFT is performed on a computer 
using the Fast Fourier Transform, so both variables are discrete and quantized.\\
The magnitude squared of the STFT yields the spectrogram representation of the Power Spectral Density of the function:

${\displaystyle \operatorname {spectrogram} \{x(t)\}(\tau ,\omega )\equiv |X(\tau ,\omega )|^{2}}$
 

https://en.wikipedia.org/wiki/Short-time_Fourier_transform

\subsection{Dialog System}

\subsubsection{Speech Recognition}
http://doc.aldebaran.com/2-1/naoqi/audio/alspeechrecognition.html

\subsubsection{Dynamic Oral Feedback}
reason to design the dynamic feedback, NPL may want to have it here. 



\section{Summary}

