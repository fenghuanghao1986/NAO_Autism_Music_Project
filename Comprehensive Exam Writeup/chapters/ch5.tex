\chapter{X-Elophone: A New Instrument with New Experiment Design}
In this section, a novelty instrument will be described in the following section.

reason why need this design. Due to the limitation of keys. This provides more 
possibility for different timber and major minor keys. That allows this system to 
play more customized song which kids love.

\section{Xylophone Modification}

\subsection{Components Selection}

\subparagraph{A. Piezo Vibration Sensor: }
The LDT0-028K is a flexible component comprising a 28 µm thick piezoelectric PVDF
polymer film with screen-printed Ag-ink electrodes, laminated to a 0.125 mm polyester 
substrate, and fitted with two crimped contacts. As the piezo film is displaced from 
the mechanical neutral axis, bending creates very high strain within the piezopolymer 
and therefore high voltages are generated. When the assembly is deflected by direct 
contact, the device acts as a flexible "switch", and the generated output is sufficient 
to trigger MOSFET or CMOS stages directly. If the assembly is supported by its contacts 
and left to vibrate "in free space" (with the inertia of the clamped/free beam creating 
bending stress), the device will behave as an accelerometer or vibration sensor. Adding 
mass, or altering the free length of the element by clamping, can change the resonant 
frequency and sensitivity of the sensor to suit specific applications. Multi-axis response 
can be achieved by positioning the mass off center. The LDTM-028K is a vibration sensor 
where the sensing element comprises a cantilever beam loaded by an additional mass to 
offer high sensitivity at low frequencies. 
%https://cdn.sparkfun.com/datasheets/Sensors/ForceFlex/LDT_Series.pdf

Also have to show the circuit, how to design this and attach the figure from here
https://www.sparkfun.com/datasheets/Sensors/Flex/MSI-techman.pdf
page 39

\subparagraph{B. Op-Amp: }
An operational amplifier (often op-amp or opamp) is a DC-coupled high-gain electronic 
voltage amplifier with a differential input and, usually, a single-ended output.[1] In 
this configuration, an op-amp produces an output potential (relative to circuit ground) 
that is typically hundreds of thousands of times larger than the potential difference 
between its input terminals. Operational amplifiers had their origins in analog computers, 
where they were used to perform mathematical operations in many linear, non-linear, and 
frequency-dependent circuits.
The popularity of the op-amp as a building block in analog circuits is due to its 
versatility. By using negative feedback, the characteristics of an op-amp circuit, its 
gain, input and output impedance, bandwidth etc. are determined by external components 
and have little dependence on temperature coefficients or engineering tolerance in the 
op-amp itself.
Op-amps are among the most widely used electronic devices today, being used in a vast 
array of consumer, industrial, and scientific devices. Many standard IC op-amps cost 
only a few cents in moderate production volume; however, some integrated or hybrid 
operational amplifiers with special performance specifications may cost over US 100 in 
small quantities.[2] Op-amps may be packaged as components or used as elements of more 
complex integrated circuits.
The op-amp is one type of differential amplifier. Other types of differential amplifier 
include the fully differential amplifier (similar to the op-amp, but with two outputs), 
the instrumentation amplifier (usually built from three op-amps), the isolation amplifier 
(similar to the instrumentation amplifier, but with tolerance to common-mode voltages 
that would destroy an ordinary op-amp), and negative-feedback amplifier (usually built 
from one or more op-amps and a resistive feedback network).
https://en.wikipedia.org/wiki/Operational_amplifier\\

https://ww1.microchip.com/downloads/en/DeviceDoc/21733j.pdf\\



\subparagraph{C. Multiplexer: }
In electronics, a multiplexer (or mux) is a device that selects between several analog 
or digital input signals and forwards it to a single output line.[1] A multiplexer of 
${\displaystyle 2^{n}} 2^{n}$ inputs has ${\displaystyle n}$ n select lines, which are 
used to select which input line to send to the output.[2] Multiplexers are mainly used 
to increase the amount of data that can be sent over the network within a certain amount 
of time and bandwidth.[1] A multiplexer is also called a data selector. Multiplexers can 
also be used to implement Boolean functions of multiple variables.
An electronic multiplexer makes it possible for several signals to share one device or 
resource, for example, one A/D converter or one communication line, instead of having one 
device per input signal.
Conversely, a demultiplexer (or demux) is a device taking a single input and selecting 
signals of the output of the compatible mux, which is connected to the single input, and 
a shared selection line. A multiplexer is often used with a complementary demultiplexer 
on the receiving end.[1]
An electronic multiplexer can be considered as a multiple-input, single-output switch, 
and a demultiplexer as a single-input, multiple-output switch.[3] The schematic symbol 
for a multiplexer is an isosceles trapezoid with the longer parallel side containing the 
input pins and the short parallel side containing the output pin.[4] The schematic on the 
right shows a 2-to-1 multiplexer on the left and an equivalent switch on the right. 
The ${\displaystyle sel}$ sel wire connects the desired input to the output.
The 74HC4051; 74HCT4051 is a single-pole octal-throw analog switch (SP8T) suitable for 
use in analog or digital 8:1 multiplexer/demultiplexer applications. The switch features 
three digital select inputs (S0, S1 and S2), eight independent inputs/outputs (Yn), a 
common input/output (Z) and a digital enable input (E). When E is HIGH, the switches are 
turned off. Inputs include clamp diodes. This enables the use of current limiting resistors 
to interface inputs to voltages in excess of VCC.
https://en.wikipedia.org/wiki/Multiplexer
https://cdn.sparkfun.com/assets/learn_tutorials/5/5/3/74HC_HCT4051.pdf\\


\subparagraph{D. Arduino UNO: }
The Arduino Uno is an open-source microcontroller board based on the Microchip ATmega328P 
microcontroller and developed by Arduino.cc.[2][3] The board is equipped with sets of digital 
and analog input/output (I/O) pins that may be interfaced to various expansion boards (shields) 
and other circuits.[1] The board has 14 Digital pins, 6 Analog pins, and programmable with 
the Arduino IDE (Integrated Development Environment) via a type B USB cable.[4] It can be 
powered by the USB cable or by an external 9-volt battery, though it accepts voltages between 
7 and 20 volts. It is also similar to the Arduino Nano and Leonardo.[5][6] The hardware 
reference design is distributed under a Creative Commons Attribution Share-Alike 2.5 license 
and is available on the Arduino website. Layout and production files for some versions of 
the hardware are also available.

The word "uno" means "one" in Italian and was chosen to mark the initial release of the 
Arduino Software.[1] The Uno board is the first in a series of USB-based Arduino boards,[3] 
and it and version 1.0 of the Arduino IDE were the reference versions of Arduino, now evolved 
to newer releases.[4] The ATmega328 on the board comes preprogrammed with a bootloader 
that allows uploading new code to it without the use of an external hardware programmer.[3]

While the Uno communicates using the original STK500 protocol,[1] it differs from all 
preceding boards in that it does not use the FTDI USB-to-serial driver chip. Instead, 
it uses the Atmega16U2 (Atmega8U2 up to version R2) programmed as a USB-to-serial converter.[7]
https://en.wikipedia.org/wiki/Arduino_Uno\\

show block diagram of the code:

\subsection{Circuit Design}

\section{Sound Design}

\subsection{ChucK: A On-the-fly Audio Programming Language}
https://www.researchgate.net/profile/Ge_Wang9/publication/259326122_The_ChucK_Programming_Language_A_Strongly-Timed_On-the-fly_Environmentality/links/0c96052b02acb79c2c000000.pdf\\

briefly describe the language, and show the block diagram of of this code as well. 
The computer has long been considered an extremely attractive tool for creating,
manipulating, and analyzing sound. Its precision, possibilities for new timbres, and
potential for fantastical automation make it a compelling platform for expression
and experimentation - but only to the extent that we are able to express to the
computer what to do, and how to do it. To this end, the programming language
has perhaps served as the most general, and yet most precise and intimate interface
between humans and computers. Furthermore, “domain-specific” languages can
bring additional expressiveness, conciseness, and perhaps even different ways of
thinking to their users.
This thesis argues for the philosophy, design, and development of ChucK, a
general-purpose programming language tailored for computer music. The goal is to
create a language that is expressive and easy to write and read with respect to time
and parallelism, and to provide a platform for precise audio synthesis/analysis and
rapid experimentation in computer music. In particular, ChucK provides a syntax
for representing information flow, a new time-based concurrent programming model
that allows programmers to flexibly and precisely control the flow of time in code (we
call this “strongly-timed”), and facilities to develop programs on-the-fly - as they
run. A ChucKian approach to live coding as a new musical performance paradigm is
also described. In turn, this motivates the Audicle, a specialized graphical environment 
designed to facilitate on-the-fly programming, to visualize and monitor ChucK
programs in real-time, and to provide a platform for building highly customizable
user interfaces.
In addition to presenting the ChucK programming language, a history of music
and programming is provided (Chapter 2), and the various aspects of the ChucK
language are evaluated in the context of computer music research, performance,
and pedagogy (Chapter 6). As part of an extensive case study, the thesis discusses
ChucK as a primary teaching and development tool in the Princeton Laptop Orchestra 
(PLOrk), which continues to be a powerful platform for deploying ChucK
1) to teach topics ranging from programming to sound synthesis to music composition, 
and 2) for crafting new instruments, compositions, and performances for
computer-mediated ensembles. Additional applications are also described, including 
classrooms, live coding arenas, compositions and performances, user studies,
and integrations of ChucK into other software systems.
The contributions of this work include the following. 1) A time-based programming 
mechanism (both language and underlying implementation) for ultraprecise audio 
synthesis, naturally extensible to real-time audio analysis. 2) A nonpreemptive, 
time/event-based concurrent programming model that provides fundamental flexibility 
and readability without incurring many of the difficulties of
programming concurrency. 3) A ChucKian approach to writing code and designing audio 
programs on-the-fly. This rapid prototyping mentality has potentially
wide ramifications in the way we think about coding audio, in designing/testing
software (particular for real-time audio), as well as new paradigms and practices
in computer-mediated live performance. 4) The Audicle as a new type of audio
programming environment that combines live development with visualizations. 5)
Extended case studies of using, teaching, composing, and performing with ChucK,
most prominently in the Laptop Orchestra. These show the power of teaching programming 
via music, and vice versa - and how these two disciplines can reinforce
each other.