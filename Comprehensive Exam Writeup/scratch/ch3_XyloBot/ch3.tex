\chapter{Xylo-Bot: A Interactive Human-Robot Music Teaching System Design} 
A novelty Interactive human-robot music teaching system design is presented in this chapter. In order to make robot play xylophone properly, several things need to be done before that. First is to find a proper xylophone with correct timber; second, we have to make the xylophone in a proper position in front of the robot that makes it to be seen properly and be reached to play; finally, design the intelligent music system for NAO. 

\section{NAO: A Humanoid Robot}
We used a humanoid  robot called NAO developed by Aldebaran Robotics in France [ref]. NAO is 58 cm (23 inches) tall, with 25 degrees of freedom this robot can conduct most of the human behaviors. It also features an onboard multimedia system including, four microphones for voice recognition, and sound localization, two speakers for text-to-speech synthesis, and two HD cameras with maximum image resolution 1280×960 for online observation. As shown in Figure 4-1, these utilities are located in the middle of the forehead and the mouth area. NAO’s computer vision module includes facial and shape recognition units. By using Choregraphe software (Shown  in  Figure 4-2), researcher can easily control NAO remotely. Inside the user interface we have access to NAO’s cameras. It is also easy to control different joints of the robot (see  Figure 4-3). This  allows the operator to control and monitor the different activities of robots online. 

\section{Accessories}
Due to the size limitation of the toy xylophone, we have to design some accessories for robot to able play.

\subsection{Xylophone: A Toy for Music Beginner}
Attach the picture of xylophone and describe the frequency of all notes.

\subsection{Mallet Gripper Design}
3D printed, need to measure some numbers and list them here, attach the SolidWorks
screen shot and actual pictures.

\subsection{Instrument Stand Design}
Laser cut, made of wood, need measurement of all dimensions, attach the SolidWorks
screen shot and actual pictures.

\section{Music Teaching System Design}
After all the preparation, we start to design the system, including joint trajectory, vision control and audio feedback

\subsection{Joint Trajectory}
Calibration of kinematic parameters. Try to explain it better at some point, if possible describer the future work may implemented using vision feedback system. 

\subsection{Auditory Feedback System}
The purpose of this system is to provide a back and forth interaction using music therapy to teach kid social skills and music knowledge.

\subsubsection{Short Time Fourier Transform}
The short-time Fourier transform (STFT), is a Fourier-related transform used to determine the sinusoidal frequency and phase content of local sections of a signal as it changes over time.[1] In practice, the procedure for computing STFTs is to divide a longer time signal into shorter segments of equal length and then compute the Fourier transform separately on each shorter segment. This reveals the Fourier spectrum on each shorter segment. One then usually plots the changing spectra as a function of time.
Use wiki to attach more pics and more info here.
https://en.wikipedia.org/wiki/Short-time_Fourier_transform

\subsection{Dialog System}

\subsubsection{Speech Recognition}
http://doc.aldebaran.com/2-1/naoqi/audio/alspeechrecognition.html

\subsubsection{Dynamic Oral Feedback}
reason to design the dynamic feedback, NPL may want to have it here. 



\section{Summary}

