\chapter{Xylo-Bot: An Interactive Music Teaching System} 
As mentioned above, music can be considered as a effective method
for emotion and non-verbal communication. Individuals with ASD are interested in 
interacting with social robot. Therefore, implement music therapy like intervention
sessions using humanoid social robot become possible. A novel interactive human-robot 
music teaching system design is presented in this chapter. Hardware and software
design will be discussed in following sections including experiment room setup, robot 
selection, instrument accessories and platform design.\\
In order to make the robot play the xylophone properly and be able to conduct a
music-based social interaction scenario, several things needed to be done. 
First is to find a proper xylophone with correct timber; second, we 
have to arrange the xylophone in the proper position in front of the robot 
to make it visible and be reachable to play; after that, a set of challenge based 
experimental sessions require to be constructed including baseline session, intervention
sessions and exit session with various level of activities; finally, the
module based interactive music teaching system will be designed and programmed 
which can be implemented into the experimental sessions.\\

\section{Hardware Selection and Design}

\subsection{NAO: A Humanoid Robot}
NAO, a humanoid robot were selected in the current research which sailed by SoftBank company. 
NAO is 58 cm (23 inches) tall, with 25 degrees of freedom. Most of the human body
movement can be performed by this robot. It also features an onboard multimedia 
system including four microphones for voice recognition and sound localization, 
two speakers for text-to-speech synthesis, and two HD cameras with maximum image 
resolution 1280 x 960 for online observation. As shown in Figure \ref{nao_body}, these 
utilities are located in the middle of forehead and mouth area. NAO’s 
computer vision module includes facial and shape recognition units. By using the 
vision feature of the robot, it can see the instrument from its lower camera 
and be able to implement an eye-arm self-calibration system which allows the 
robot to have real-time micro-adjustment of its arm joints in case of off 
positioning before music play.\\

\begin{figure}[tbp]
	\begin{center}
		\begin{tabular}{c}
			\epsfig{figure=./chapters/fig/naobody.eps, scale = .8}\label{nao_body} \\
		\end{tabular}
		\caption{A Humanoid Robot NAO: 25 Degrees of Freedom, 2 HD Cameras and 4 Microphones} \label{nao_body}
	\end{center}
\end{figure}

The robot arms have a length of approximately 31 cm. Position feedback sensors
are equipped in each joint of the robot in order to have real time localization
information from them which could provide well protection for robot safety. Each arm
has five degrees of freedom and is equipped with those sensors to measure the position  
of joint movement. To determine the pose of the instrument and the mallets' heads, the robot 
analyzes images from the lower monocular camera located in its head, which has a 
diagonal field of view of 73 degrees. By using these dimensions, proper size instrument
can be selected and more accessories can be built which will be presented in the follow
sections.\\

The four microphone locations embedded on the toy or NAO's head can be seen in figure \ref{mic_loc}. 
According to the official Aldebaran documentation, these microphones have sensitivity 
of 20mV/Pa +/-3dB at 1kHz, and an input frequency range of 150Hz - 12kHz. Data 
will be recorded as a 16 bit, 48000Hz, 4 channels wave file which meets the 
requirements for designing the online feedback audio score system will be described
in later section.\\

\begin{figure}[tbp]
	\begin{center}
		\begin{tabular}{c}
			\epsfig{figure=./chapters/fig/mic.eps, scale = 1.2}\label{mic_loc} \\
		\end{tabular}
		\caption{Microphone locations on NAO's head} \label{mic_loc}
	\end{center}
\end{figure}

\subsection{Accessories}
The purpose of this study is to have a toy-size humanoid robot to play and 
teach music for children with autism. Some necessary accessories 
needed to be purchased and made before the robot able to complete this task. 
All accessories will be discussed in the following paragraphs.\\

\subsubsection{Xylophone: A Toy for Music Beginner}
In this system, due to NAO's open arms' length, a Sonor Toy Sound SM soprano-xylophone 
with 11 sound bars of 2 cm in width were selected and purchased. The instrument has a size of 
31 cm x 9.5 cm x 4 cm, including the resonating body. The smallest sound bar is 
playable in an area of 2.8 cm x 2 cm, the largest in an area of 4.8 cm x 2 cm. The 
instrument is diatonically tuned in C-Major/a-minor. See Figure \ref{xylo640}.
The 11 bars of the xylophone represent 11 different notes (11 frequencies) which covers 
one and half octave scale starting from C6 to F7. \\ 
In order to provide music teaching enviorment system for children with autism, xylophone
is one of the best choice for such study. Xylophone, as well known as marimba, has 
categorized as a percussion instrument which consisting of a set of metal/wooden bars
struck with mallets to produce fine musical tones. Other than keyboar or drum, for playing 
xylophone properly, a unique techique need to be applied. A proper strike movement is
required in order to produce a beautiful note coming out of xylophone. This action
is perfect for practice motor control and the melody which played by user could also 
support the music emotion learning. \\

\begin{figure}[tbp]
	\begin{center}
		\begin{tabular}{c}
			\epsfig{figure=./chapters/fig/xylo640.eps, scale = .6}\label{xylo640} \\
		\end{tabular}
		\caption{Actual Xylophone and Mallets from NAO's Bottom Camera} \label{xylo640}
	\end{center}
\end{figure}

\subsubsection{Mallet Gripper Design}
For the mallets, we used the pair that came with the xylophone with a modified 
3D printed griper which allows the robot hands to hold them properly. The mallets 
are approximately 21 cm in length and include a head of 0.8 cm radius. Comparing 
to other design, this mallet gripper prides nature hoding position for robot, and
set up a proper modal for participants in holding the mallet stick in a good way. 
See Figure \ref{griper}.\\

\begin{figure}[tbp]
	\begin{center}
		\begin{tabular}{c}
			\epsfig{figure=./chapters/fig/grip.eps, scale = 0.5}\label{griper} \\
		\end{tabular}
		\caption{Mallet Griper} \label{griper}
	\end{center}
\end{figure}

\subsubsection{Instrument Stand Design}
By carefully measured dimentions, a wooden base was designed and laser cut to hold 
the xylophone in a proper height for robot crouching position. Using this fine adjusted position,
robot can easily fixed in a location and somehow have the same height level with the 
participants which makes it more nature in teaching activities for the entire time.
See Figure \ref{stand}.\\
%insert a pic with session on going

\begin{figure}[tbp]
	\begin{center}
		\begin{tabular}{c}
			\epsfig{figure=./chapters/fig/left_view.eps, scale = 0.3}\label{left} \\
			(a)\\
			\epsfig{figure=./chapters/fig/top_view.eps, scale = 0.3
			} \label{top}\\
			(b)\\
			\epsfig{figure=./chapters/fig/front_view.eps, scale = 0.6} \label{front}\\
			(c)
		\end{tabular}
		\caption{Instrument Stand: (a) Left View (b) Top View (c) Front View.} \label{stand}
	\end{center}
\end{figure}

\section{Experimental Sessions Design}

\subsection{Experiment Room}
All the sessions were held in a 11ft x 9.5ft x 10ft room located in the Ritchie School of Engineering Room 248, University of Denver.
6 HD surveillance cameras installed at corners, side wall and ceiling of the experimental room see Figure \ref{room}. 
One mini hidden microphone attached at the ceiling camera for sending real-time audio to the observation room in order to let the care
giver to listen to. An external hand-held audio recorder were set in front of the participant during
sessions for collecting high quality audios for future process.

\begin{figure}[tbp]
	\begin{center}
		\begin{tabular}{c}
			\epsfig{figure=./chapters/fig/room.eps, scale = .6}\label{room} \\
		\end{tabular}
		\caption{Schematic robot-based therapy session and video capturing setting} \label{room}
	\end{center}
\end{figure}

\begin{sidewaysfigure}[tbp]
	\begin{center}
		\begin{tabular}{c}
			\epsfig{figure=./chapters/fig/room_pana.eps, scale = .35}\label{room_pana} \\
		\end{tabular}
		\caption{Experiment Room} \label{room_pana}
	\end{center}
\end{sidewaysfigure}

As shown in Figure \ref{room_pana}, the observation room is located at the back of the one-way
mirror facing at the rear side of participants in order to avoid distraction while sessions on going.
Real-time video and audio were broadcasting to the observation room during sessions, which
allowed researchers to observe and to record in the meantime. Parents behind the mirror may also 
call off the session in case of emergency.\\

\subsection{Participant Selection}
Nine ASD kids (average age: 11.73, std: 3.11) and 7 TD kids (average age: 10.22, std: 2.06) were recruited in this
study. All participants were selected from the potential subject pool with help from Psychology department. 
For each participant in ASD group, 6 sessions were delivered including baseline session,
intervention sessions and exit session. As for TD control group, only baseline and exit sessions 
were required for each participant. Every session lasts for 30-60 minutes total depends on the difficulty
of each session and the performance of individuals. Typically, baseline and exit sessions length 
can be compariable for same subject. As for intervention sessions, duration should gradually increase
one after another due to the challenge level uprising.\\

\subsection{Session Detail}
Baseline and exit session contains 2 activities which are 1) music practice and 2) music game play.
Participants were asked to complete a challenge like full song play, starting from single note strike
with color hint. Multiple notes, half song play and full song play were coming one by one as long as 
participants aced the previous task. After practice part done, a freshly designed music game will be 
presented to participants which contains three novel entertaining game modes in it, participants
were allowed to communicate with robot regarding which mode to play with. Mode 1): robot will randomly
pick a song from its song bank and play for kids, after each play, a music feeling will be request from 
participants in order to find out whether music emotion can be recognized from early age of children; 
Mode 2): a sequence of melody will be randomly generated by robot with consonance (happy or confortable feeling) 
or dissonance (sad or unfortable feeling) style, an oral emotion feeling from participants will be requested 
and physical playback afterwards; Mode 3): allows participants to have a 5 seconds of free play and challenge the robot to
imitate from the participants what just played by them, after robot complete playing, performance will be rated by
the participants which provide a teaching experience for all subjects. Note that, there was no limitation 
for how many trails or modes each individual who wants to play for each visit, but at least play each mode 
once in single session. The only difference between baseline and exit session was the song which used in them, 
in baseline session, "Twinkle Twinkle Little Star" was used as a standard entry level song for all 
participants, and a customized song were chosen by each individual for exit session in order to 
motivate participants for better learning music, which makes it more difficult from the baseline session. 
By using the Module-Based Acoustic Music Interactive System, inputing multiple songs become possible and
less time consuming. More than 10 songs are collected in the song bank, such as "Can Can" by Offenbach, 
"Shake if off" by Talor Swift, "Spongebob Square Pants" from Spongebob cartoon and "You are my Sunshaine"
by Johnny Cash etc. Music styles are not only kid's song like "Twinkle Twinkle" but also across clasic, 
pop, ACG, folk and more can be played by using such platform. Because of the variate music style NAO
can play, learning motivation can be successfully delivered among all sessions.\\

Each intervention session has divided into three parts: S1) warm up; S2) single activity practice (with color hint); 
and S3) music game play. Starting from intervention sessions, a motivated customized song were used in the following 
sessions and have them more engage to multiple repetitive activities. 
The purpose of having warm up section is to have the motor control skill been practiced 
and meanwhile to help participants implement the motor skills in following activities. Single activity was 
based on music practice from baseline/exit session, other then those sessions, single activity will
only have one type of music practice each individual session, for instance, single note play were 
delivered in the first intervention session, then the next time this practice will become multiple notes 
play and the level of difficulty for music play were gradually increased session after session. This was in
order to make a challenge based engagement activity for ASD group for better motivation and emotion stimuli. 
As for music game play were remain the same as baseline/exit session. See details in Table \ref{session_detail}\\

\begin{sidewaystable}[tbp]
	\begin{center}
		\resizebox{\columnwidth}{!}{
\begin{tabular}{cccll}
	{\color[HTML]{666666} Session \#}          & {\color[HTML]{666666} Session Type}                              & {\color[HTML]{666666} Outline}                                                                                                                        & \multicolumn{1}{c}{{\color[HTML]{666666} Session Details}}                                                                                                                           & \multicolumn{1}{c}{{\color[HTML]{666666} Session Purpose}}                                                                                                             \\
	{\color[HTML]{666666} }                    & {\color[HTML]{666666} }                                          & \multicolumn{1}{l}{{\color[HTML]{666666} }}                                                                                                           & {\color[HTML]{666666} 1. Computer music listing, the participant will   be asked to listen to few music pieces composed by professional;}                                            & {\color[HTML]{666666} Have better understanding of how ASD group   perceive emotions from music.Compare EDA signals between ASD and TD groups,   find out similarity.} \\
	{\color[HTML]{666666} }                    & {\color[HTML]{666666} }                                          & \multicolumn{1}{l}{{\color[HTML]{666666} }}                                                                                                           & {\color[HTML]{666666} 2. Participant should   describe the feeling based on what they just listened, make selection on   computer or tell researcher/therapist;}                     & {\color[HTML]{666666} Compare EDA signals between ASD and TD groups,   find out similarity.}                                                                           \\
	{\color[HTML]{666666} }                    & {\color[HTML]{666666} }                                          & \multicolumn{1}{l}{\multirow{-3}{*}{{\color[HTML]{666666} Full   activities with "Twinkle Twinkle Little Star" whole song play,   human assessment}}} & {\color[HTML]{666666} 3. Participant free play the   music pieces (2 mins), computer/researcher record to see which music which   played /liked the most from Step 1.}               &                                                                                                                                                                        \\
	{\color[HTML]{666666} }                    & {\color[HTML]{666666} }                                          & {\color[HTML]{666666} }                                                                                                                               & {\color[HTML]{666666} 1. NAO plays the   the music piece select by participant;}                                                                                                     & {\color[HTML]{666666} Getting social   baseline from both groups in different tasks for future usage;}                                                                 \\
	{\color[HTML]{666666} }                    & {\color[HTML]{666666} }                                          & {\color[HTML]{666666} }                                                                                                                               & {\color[HTML]{666666} 2. Teach music piece from   Step 1 gradually, form single strike, color strike, rhythm strike, multiple   strike the entire music piece.}                      & {\color[HTML]{666666} Compare difference between both groups.}                                                                                                         \\
	\multirow{-6}{*}{{\color[HTML]{666666} 1}} & \multirow{-6}{*}{{\color[HTML]{666666} Baseline   Session}}      & \multirow{-3}{*}{{\color[HTML]{666666} Music Game}}                                                                                                   & {\color[HTML]{666666} 3. Participant free play   xylophone (2 mins), NAO record how kids played for post process.}                                                                   &                                                                                                                                                                        \\
	{\color[HTML]{666666} }                    & {\color[HTML]{666666} }                                          & {\color[HTML]{666666} }                                                                                                                               & {\color[HTML]{666666} 1. Participant   should hit an arbitrary bar after each robot’s struck;}                                                                                       & {\color[HTML]{666666} Improve in Joint   Attention skills;}                                                                                                            \\
	{\color[HTML]{666666} }                    & {\color[HTML]{666666} }                                          & {\color[HTML]{666666} }                                                                                                                               & {\color[HTML]{666666} 2. Repeat Step 1, while the   participant should use both hands in order;}                                                                                     & {\color[HTML]{666666} Colors Recognition.}                                                                                                                             \\
	{\color[HTML]{666666} }                    & {\color[HTML]{666666} }                                          & {\color[HTML]{666666} }                                                                                                                               & {\color[HTML]{666666} 3. NAO strikes single bar   associate with a random color, and participant should select the proper color   on a computer screen;}                             &                                                                                                                                                                        \\
	\multirow{-4}{*}{{\color[HTML]{666666} 2}} & {\color[HTML]{666666} }                                          & \multirow{-4}{*}{{\color[HTML]{666666} Signal note practice with color hints, robot assessment}}                                                      & {\color[HTML]{666666} 4. NAO show Red, Green, Blue,   Pink on its eyes, participant strike the matching color on xylophone.}                                                         &                                                                                                                                                                        \\
	{\color[HTML]{666666} }                    & {\color[HTML]{666666} }                                          & \multicolumn{1}{l}{{\color[HTML]{666666} }}                                                                                                           & {\color[HTML]{666666} 1. Review Step 4 from Session 2;}                                                                                                                              & {\color[HTML]{666666} Improve Verbal short-term Memory;}                                                                                                               \\
	{\color[HTML]{666666} }                    & {\color[HTML]{666666} }                                          & \multicolumn{1}{l}{{\color[HTML]{666666} }}                                                                                                           & {\color[HTML]{666666} 2. NAO demonstrates hitting   one or two bars in order and verbally repeats “one-two” with the movement;}                                                      & {\color[HTML]{666666} Improve the perception of numbers and counting.}                                                                                                 \\
	{\color[HTML]{666666} }                    & {\color[HTML]{666666} }                                          & \multicolumn{1}{l}{{\color[HTML]{666666} }}                                                                                                           & {\color[HTML]{666666} 3. Participant verbally says   “one-two” along with robot hitting movement;}                                                                                   &                                                                                                                                                                        \\
	\multirow{-4}{*}{{\color[HTML]{666666} 3}} & {\color[HTML]{666666} }                                          & \multicolumn{1}{l}{\multirow{-4}{*}{{\color[HTML]{666666} Multiple notes practice with color hints, robot assessment}}}                               & {\color[HTML]{666666} 4. Repeat Step 2, use “color   names” instead of “one-two”, kid should imitate the strikes along with oral   response.}                                        &                                                                                                                                                                        \\
	{\color[HTML]{666666} }                    & {\color[HTML]{666666} }                                          & {\color[HTML]{666666} }                                                                                                                               & {\color[HTML]{666666} 1. Review Step 4   from Session 3;}                                                                                                                            & {\color[HTML]{666666} Improve auditory   imitation skill;}                                                                                                             \\
	{\color[HTML]{666666} }                    & {\color[HTML]{666666} }                                          & {\color[HTML]{666666} }                                                                                                                               & {\color[HTML]{666666} 2. Imitation from the robot   while striking 3 or 4 notes in order considering the time interval between   each notes with saying “colors”;}                   & {\color[HTML]{666666} Good cooperation between eye and hands   movements;}                                                                                             \\
	{\color[HTML]{666666} }                    & {\color[HTML]{666666} }                                          & {\color[HTML]{666666} }                                                                                                                               & {\color[HTML]{666666} 3. Repeat Step 2 without oral   response.}                                                                                                                     & {\color[HTML]{666666} Improve the Visual pursuit;}                                                                                                                     \\
	{\color[HTML]{666666} }                    & {\color[HTML]{666666} }                                          & {\color[HTML]{666666} }                                                                                                                               & {\color[HTML]{666666} 4. Participant play 3 or 4   notes on a virtual xylophone on tablet/computer, NAO imitate what has been   played on real xylophone.}                           & {\color[HTML]{666666} Sharing attention between different tasks;}                                                                                                      \\
	\multirow{-5}{*}{{\color[HTML]{666666} 4}} & {\color[HTML]{666666} }                                          & \multirow{-5}{*}{{\color[HTML]{666666} First half song practice with color hints}}                                                                    &                                                                                                                                                                                      & {\color[HTML]{666666} Improve in playing turn-taking and group games   in robot-child interactions.}                                                                   \\
	{\color[HTML]{666666} }                    & {\color[HTML]{666666} }                                          & \multicolumn{1}{l}{{\color[HTML]{666666} }}                                                                                                           & {\color[HTML]{666666} 1. Review Step 3 from Session 4;}                                                                                                                              & {\color[HTML]{666666} Working memory of children with autism, Dual   task performance;}                                                                                \\
	{\color[HTML]{666666} }                    & {\color[HTML]{666666} }                                          & \multicolumn{1}{l}{{\color[HTML]{666666} }}                                                                                                           & {\color[HTML]{666666} 2. NAO demonstrates striking   2 or 3 notes with saying “note names from A to G”, participant verbally   repeats the notes name while robot hitting the bars;} & {\color[HTML]{666666} Improve in joint attention skill;}                                                                                                               \\
	{\color[HTML]{666666} }                    & {\color[HTML]{666666} }                                          & \multicolumn{1}{l}{{\color[HTML]{666666} }}                                                                                                           & {\color[HTML]{666666} 3. Imitation from the robot   while striking 3 or 4 notes in order considering the time interval between   each notes with saying “note names/colors”;}        & {\color[HTML]{666666} Improve in shifting attention skill;}                                                                                                            \\
	\multirow{-4}{*}{{\color[HTML]{666666} 5}} & \multirow{-17}{*}{{\color[HTML]{666666} Intervention   Session}} & \multicolumn{1}{l}{\multirow{-4}{*}{{\color[HTML]{666666} Second half song practice with color hints, robot assessment}}}                             &                                                                                                                                                                                      & {\color[HTML]{666666} Improve in auditory memory of children.}                                                                                                         \\
	{\color[HTML]{666666} 6}                   & {\color[HTML]{666666} Exit Session}                              & \multicolumn{1}{l}{{\color[HTML]{666666} Full Activities   with customized whole song play, human assessment}}                                        & {\color[HTML]{666666} 1. Repeat   Pre-Session}                                                                                                                                       & {\color[HTML]{666666} Compare results   between ASD vs TD, Pre-Session vs Post-Session.}                                                                              
\end{tabular}
}
\label{session_detail}
\end{center}
\end{sidewaystable}

\section{Module-Based Acoustic Music Interactive System Design}
In this section, a novel module-based robot-music teaching system will be presented. 
Several missions need to be accomplished here: a) make robot play sequence of notes or melody
fluently; b) have robot play note accurately; c) be able to adopt multiple songs easily;
d) be able to have social communication between robot and participants; e) be able to 
deliver learning and teaching experience for participants; f) fast response and accurate
decision making. In order to fulfill these tasks, a module-based acoustic music interactive system
were designed in this wokr. Three modules have been built in this intelligent system including Module 1: eye-hand 
self-calibration micro-adjustment; Module 2: joint trajectory generator; and 
Module 3: real time performance scoring feedback. See Figure \ref{module}.\\

\begin{figure}[tbp]
	\begin{center}
		\begin{tabular}{c}
			\epsfig{figure=./chapters/fig/module_blocks.eps, scale = .5}\label{module} \\
		\end{tabular}
		\caption{Block Diagram of Module-Based Acustic Music Interactive System} \label{module}
	\end{center}
\end{figure}

\subsection{Module 1: Eye-hand Self-Calibration Micro-Adjustment}
Knowledge about the parameters of the robot's kinematic model is essential for 
tasks requiring high precision, such as playing the xylophone. While the kinematic 
structure is known from the construction plan, errors can occur, e.g., due to the 
imperfect manufacturing. After multiple rounds of testing, it was found the targeted angle chain 
of arms never actually equals the returned chain. We therefore used a 
calibration method to accurately eliminate these errors.\\
 

\subsubsection{Color-Based Object Tracking}
To play the xylophone, the robot has to be able to adjust its motions according to
the estimated relative position of the instrument and the heads of the beaters it is 
holding. To estimate these poses, adopted in this thesis, we 
used a color-based technique.\\
The main idea is, based on the RGB color of the center blue bar, given a hypothesis 
about the instrument's pose, one can project the contour of the object's model into the 
camera image and compare them to actually observed contour. In this way, it is possible 
to estimate the likelihood of the pose hypothesis. By using this method, it allows
the robot to track the instrument with very low cost in real-time. See Figure \ref{color_detection}\\
\begin{figure}[tbp]
	\begin{center}
		\begin{tabular}{c}
			\epsfig{figure=./chapters/fig/blue.eps, scale = 0.3}\label{single_color_a} \\
			(a)\\
			\epsfig{figure=./chapters/fig/all_color.eps, scale = 0.3
			} \label{all_color_b}\\
			(b)\\
			\epsfig{figure=./chapters/fig/color_detection.eps, scale = 0.6} \label{color_detection_c}\\
			(c)
			\end{tabular}
			\caption{Color Detection From NAO's Bottom Camera: (a) Single Blue Color Detection (b) Full Instrument Color Detection (c) Color Based Edge Detection.} \label{color_detection}
	\end{center}
\end{figure}

\subsection{Module 2: Joint Trajectory Generator}
Our system parses a list of hex-decimal numbers (from 1 to b) to obtain the sequence
of notes to play. It converts the notes into a joint trajectory using the beating
configurations obtained from inverse kinematics as control points. The timestamps
for the control points will be defined by the user in order to meet the experiment requirement.
The trajectory is then computed using Bezier interpolation in joint space by the
manufacturer-provided API and then sent to the robot controller for execution. In this
way, the robot plays in-time with the song.\\

\subsection{Module 3: Real-Time Performance Scoring Feedback}
The purpose of this system is to provide a real-life interaction experience using 
music therapy to teach kids social skills and music knowledge.  In this scoring 
system, two core features were designed to complete the task: 1) music detection;
2) intelligent scoring-feedback system.\\


\subsubsection{A. Music Detection}
Music, in the understanding of science and technology, can be considered as a combination 
of time and frequency. In order to make the robot detect a sequence of frequencies, we adopted the 
short-time Fourier transform (STFT) to this audio feedback system. This allows the robot to 
be able to understand the music played by users and provide the proper feedback as
a music teaching instructor.\\

The short-time Fourier transform (STFT) , is a Fourier-related transform used to 
determine the sinusoidal frequency and phase content of local sections of a signal 
as it changes over time. In practice, the procedure for computing STFTs is to divide 
a longer time signal into shorter segments of equal length and then compute the 
Fourier transform separately on each shorter segment. This reveals the Fourier 
spectrum on each shorter segment. One then usually plots the changing spectra as 
a function of time. In the discrete time case, the data to be transformed could 
be broken up into chunks or frames (which usually overlap each other, to reduce 
artifacts at the boundary). Each chunk is Fourier transformed, and the complex 
result is added to a matrix, which records magnitude and phase for each point in 
time and frequency. This can be expressed as:
\\

${\displaystyle \mathbf {STFT} \{x[n]\}(m,\omega )\equiv X(m,\omega )=\sum _{n=-\infty }^{\infty }x[n]w[n-m]e^{-j\omega n}}$
\\

likewise, with signal x[n] and window w[n]. In this case, m is discrete and $\omega$ 
is continuous, but in most typical applications, the STFT is performed on a computer 
using the Fast Fourier Transform, so both variables are discrete and quantized.\\
The magnitude squared of the STFT yields the spectrogram representation of the Power 
Spectral Density of the function:
\\

${\displaystyle \operatorname {spectrogram} \{x(t)\}(\tau ,\omega )\equiv |X(\tau ,\omega )|^{2}}$\\

After the robot detects the notes from user input, a list of hex-decimal number will be
returned. This list will be used in two purposes: 1) to compare with the target list
for scoring and sending feedback to user; 2) used as a new input to have
robot playback in the game session as discussed in the next chapter.\\

\begin{figure}[tbp]
	\begin{center}
		\begin{tabular}{c}
			\epsfig{figure=./chapters/fig/stft.eps, scale = 1.5}\label{stft} \\
		\end{tabular}
		\caption{Melody Detection with Short Time Fourier Transform} \label{stft}
	\end{center}
\end{figure}

\subsubsection{B. Intelligent Scoring-Feedback System}
In order to compare the detected notes and the target notes, we used an algorithm
which is normally used in information theory linguistics called Levenshtein Distance.
This algorithm is a string metric for measuring the difference between two sequences.\\

In our case, the Levenshtein distance between two string-like hex-decimal numbers 
${\displaystyle a,b}$ (of length ${\displaystyle |a|}$ and ${\displaystyle |b|}$ respectively) 
is given by ${\displaystyle \operatorname {lev} _{a,b}(|a|,|b|)}$ where
\\

${\displaystyle \qquad \operatorname {lev} _{a,b}(i,j)={\begin{cases}\max(i,j)&{\text{ if }}\min(i,j)=0,\\\min {\begin{cases}\operatorname {lev} _{a,b}(i-1,j)+1\\\operatorname {lev} _{a,b}(i,j-1)+1\\\operatorname {lev} _{a,b}(i-1,j-1)+1_{(a_{i}\neq b_{j})}\end{cases}}&{\text{ otherwise.}}\end{cases}}}$\\
\\

where ${\displaystyle 1_{(a_{i}\neq b_{j})}}$ is the indicator function equal to 0 when 
${\displaystyle a_{i}=b_{j}}$ and equal to 1 otherwise, and ${\displaystyle \operatorname {lev} _{a,b}(i,j)}$ 
is the distance between the first ${\displaystyle i}$ characters of ${\displaystyle a}$ and the
first ${\displaystyle j}$ characters of ${\displaystyle b}$.

Note that the first element in the minimum corresponds to deletion (from ${\displaystyle a}$ to 
${\displaystyle b}$), the second to insertion and the third to match or mismatch, depending on 
whether the respective symbols are the same. Table \ref{LD} demonstrates how to apply this
principle in finding the Levenshtein distance of two words "Sunday" and "Saturday".\\

\begin{figure}[tbp]
	\begin{center}
		\begin{tabular}{c}
			\epsfig{figure=./chapters/fig/example_LD.eps, scale = .6}\label{LD} \\
		\end{tabular}
		\caption{An Example of Compute Levenshtein Distance for "Sunday" and "Saturday"} \label{LD}
	\end{center}
\end{figure}

Based on the real life situation, we defined a likelihood margin for determining whether the result
is good or bad: \\
  
${likelihood = \dfrac{len(target) - lev_{target,source}}{len(target)}}$\\

where if the likelihood is greater than 66\% - 72\%, the system will consider it as a good result.
This result will be passed to the accuracy calculation system to have the robot decide whether it
needs to add more dosage to the practice. More details will be discussed in the next chapters
as it relates to the experiment design.\\

\section{Summary}
In this chapter, we have discussed both hardware and software design for the experiment sessions
by using humanoid social assistive robot NAO in music teaching and playing.\\

From Chapter One, we determined to have NAO as a music teaching instructor be able
to both teach children simple music and deliver social content simultaneously.
In order to have the system ready, we first chose the proper agent, a robot named NAO, which is
kid-friendly and has complex social abilities. Second, based on the size of the robot, some necessary
accessories were purchased and handcrafted. A toy-sized color coded xylophone became the
best option and based on the size and position, a wooden based xylophone stand was 
customized and assembled. Due to the limitation of NAO hands size, a pair of mallet gripers 
were 3-D printed and customized. At last, an intelligent module-based acoustic music 
interactive system was designed fully from scratch in order to complete the well designed experiment 
sessions. With all preparation, three modules were able to have the robot 
play, listen and teach the music freely. This allows NAO to become a great companion for 
children in both music learning and social communication. \\

Experiment session contains 3 types: baseline session, intervention sessions and exit session.
A set of difficulty gradually increased music teaching activities were well designed among all sessions
in order to deliver better music content for participants. Both ASD and TD group kids were
experiencing a game like music challenge from playing single note to the whole song. Music
game were designed to keep the session less tedious comparing to the learning part, it also
provides a oppertunity for us to learn more about the difference in learning and teaching 
music from our participants.\\

Module-based acoustic music interactive system design were challenging. Several problems needed
to be solved. In order to imporve the robot strike xylophone accuracy, Module 1 provides an autonomous 
self awareness positioning system for the robot to localize the instrument and make micro adjustment 
for arm joints that helps NAO plays the note bar properly. Multiple songs were required be able to 
played by NAO, program each song with specific arm movements sounds ridiculous. A easy music score
inputing method need to be completed before session starts. Module 2 allows the robot to be able 
to play any customized song of the user's request. This means that any songs which can be translated 
to either C-Major or a-minor key can have a well-trained person type in the hex-decimal playable 
score and allow the robot to be able to play it in seconds. Music teaching requests real-time feedback,
Module 3 was designed for providing real life music teaching experience for system users. Two key features
of this module are designed: music detection and smart scoring feedback. Short time Fourier transform
and Levenshtein distance are adopted to fulfill the requirement which allows the robot to understand
music and provide proper dosage of practice and oral feedback to users.\\

