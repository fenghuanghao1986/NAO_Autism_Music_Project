\chapter{Discussion and Conclusion}
The results indicates that the conferred music education platform will be thought-about as a decent tool for facilitate 
improving fine motor control, turn-taking skills and social activities engagement. The automated music 
detection system created a self-adjusting surroundings for participants in early sessions. Most
of the ASD youngsters began to develop the strike movement when initial 2 intervention sessions, some 
even will master the motor ability throughout the very firt warm-up event. Though the robot might
provide verbal directions and demonstrations by voice command input from participants whenever they need it. However, 
majority of the participants didn't request such service whereas playing with NAO. This finding
suggests that fine motor control ability will be learned from specific well-designed activities for
young ASD population. \\

The purpose of using music teaching scenario as the main activity in the current research is to 
create a fine and natural turn-taking behavior chance during social interaction. By observing all 
experimental sessions, 6 out of 9 subjects could dominate proper turn-taking after one or two
sessions. Note that subject 107 had significant improvement in last few sessions comparing to the
baseline session. Subject 109 had trouble with focus on listening to the robot for most of the time.
However, with researcher interfering, this kid can perform better back and forth music activity 
for a short time period. For practicing turn-taking skill, a fun motivated activity should be 
designed for children with autism. Music teaching could be a good example for accomplish this
task by taking the advantage of customized songs which selected by individuals.\\

Starting the later half of the sessions, participants can start to recognize their favorite songs,
over half of the participants were getting more into the activities, although the difficulty for
playing proper notes were much higher. It is easy to notice that older kids who spent more time engaging with the activities 
during the song practice session comparing to younger kids, especially in half/whole song play sessions.
Several reasons can explain this situation, one is because of the more complex the music,
the more challenge and more concentration participants will face. Thus, older individuals may willingly accept the 
challenge and enjoy the sense of accomplishment afterwards based on their verbal feedback to the research at the end of  
each session. The music knowledge base could also be one of the reason that conducts this result, 
since older participants may have more chance to learn music at school. Game section of each 
session provides the highest engagement level of all time, not only because of this is for 
relax and fun play, but also offers an opportunity to participants regarding challenge the 
robot to mirror the free play from them, this interesting phenomenon can be considered as 
"revenge". Especially for subject 106, who spent significant amount of time in free play 
game mode. According to the session executioner and video annotators, this particular subject shows 
high level of engagement for all activities, including free play. Based on the conversation 
and music performance with robot, subject showed strong interest in challenging the robot 
with a friendly way. \\

Emotion study for children with autism is difficult. Bio-signal provides a possible way of 
doing that. Event-based emotion classification method presented in current research suggests
that same activity with different intensities can cause emotion change in arousal dimension,
although it is difficult to label the emotions based on facial expression change in video
annotation phase for ASD group. Less emotion fluctuate in certain activity presented in Table \ref{tab2} suggests that a mild 
friendly game like teaching system may motivate better social content learning for children with autism, 
even with repetitive movements. These well designed activities could provide a relaxed learning 
environment which helps participants to focus on learning music content with proper communication behaviors. 
This may explains the improvement for music play performance in song practice (S2) through intervention 
sessions in Figure \ref{song}. Comparing emotion patterns from baseline and exit sessions between TD and ASD groups in Table \ref{tab3}, 
difference can be found. This may suggests a potential way of assist autism diagnose using bio-signal 
in early age. According to annotators and observers, TD kids showed strong passion
in this research. Excitement, stressful, disappointment were easy to be recognized and labeled from
the videos recorded. On the other hand, limited facial expression changes can be detected in
ASD group. That makes it difficult to learn whether they have different feelings or they have
same feelings but different bio-signal activities comparing to TD group. This could be a interesting research to
dig into in the future. Further more, due to the limitation of the sample size, future research can be 
continued with different classification methods with larger population.\\