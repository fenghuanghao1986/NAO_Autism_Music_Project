\newpage
\pagestyle{empty}
\renewcommand{\baselinestretch}{1}
\begin{flushleft}
\small {
FENG, HUANGHAO  \hfill (M.S., Electrical and Computer Engineering)\\
\underline{} \hfill (\mydate)\\
\underline{}}

\vspace{0.25in}
Abstract of a comprehensive exam at the University of Denver.\\
\vspace{0.25in}
Dissertation supervised by Dr. Mohammad Mahoor.\\
No. of pages in text \underline{\pageref{LastPage}}.
\end{flushleft}
\renewcommand{\baselinestretch}{2}
\normalsize{Children with Autism Spectrum Disorders (ASDs) experience deficits in verbal and
	nonverbal communication skills including motor control, emotional facial expressions, and
	eye gaze/joint attention. In this manuscript, we focus on studying the feasibility and 
	effectiveness ofusing a social robot, called NAO, and a toy instrument, xylophone at 
	modeling and improving the social responses and	behaviors of children with autism. 
	In our investigation, we designed a autonomous social interactive music teaching 
	system to fulfill this mission. \\
	A novel module-based robot-music teaching system will be presented.
	Module 1 provides an autonomous self awareness positioning system for the robot to localize
	the instrument and make micro adjustment for arm joints in order to play the note bar properly.
	Module 2 allows the robot to be able to play any customized song of the user's request. This
	means that any songs which can be translated to either C-Major or a-minor key can have a well-trained
	person type in the hex-decimal playable score and allow the robot to be able to play it in seconds. Module 3
	is designed for providing real life music teaching experience for system users. Two key features
	of this module are designed: music detection and smart scoring feedback. Short time Fourier transform
	and Levenshtein distance are adopted to fulfill the requirement which allows the robot to understand
	music and provide proper dosage of practice and oral feedback to users.\\
	A new instrument has designed in order to present better emotions from music due to the limitation
	of the original xylophone. This programmable new design of xylophone can provide a wider frequency
	range of play notes, easily switch between Major and minor keys, super easy to control and have
	fun with. 
	An automated method for emotion classification in children using electrodermal activity 
	(EDA) signals. The time-frequency analysis of the acquired raw EDAs provides a feature space based on which 
	different emotions can be recognized. To this end, the complex Morlet (C-Morlet) wavelet function is applied 
	on the recorded EDA signals. The dataset used in this manuscript includes a set of multimodal recordings of 
	social and communicative behavior as well as EDA recordings of 100 children younger than 30 months old. The 
	dataset is annotated by two experts to extract the time sequence corresponding to three main emotions 
	including “Joy”, “Boredom”, and “Acceptance”. Various experiments are 
	conducted on the annotated EDA signals to classify emotions using a support vector machine (SVM) classifier. 
	The quantitative results show that the emotion classification performance remarkably improves compared to 
	other methods when the proposed wavelet-based features are used.
	By using this emotion classification, emotion engagement during sessions and feelings between different
	music can be detected after data analysis.
}
