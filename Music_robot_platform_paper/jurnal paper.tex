\documentclass[conference]{IEEEtran}
\usepackage{epsfig}
\usepackage{graphicx}
\IEEEoverridecommandlockouts
% The preceding line is only needed to identify funding in the first footnote. If that is unneeded, please comment it out.
\usepackage{cite}
\usepackage{amsmath,amssymb,amsfonts}
\usepackage{algorithmic}
\usepackage{graphicx}
\usepackage{textcomp}
\usepackage{xcolor}
\usepackage{multirow}
\def\BibTeX{{\rm B\kern-.05em{\sc i\kern-.025em b}\kern-.08em
    T\kern-.1667em\lower.7ex\hbox{E}\kern-.125emX}}
\begin{document}

\title{Music Education Robot Platform for \\Children with Autism\\
{\footnotesize \textsuperscript{*}Note: Sub-titles are not captured in Xplore and
should not be used}
\thanks{Identify applicable funding agency here. If none, delete this.}
}

\author{\IEEEauthorblockN{1\textsuperscript{st} Huanghao Feng}
\IEEEauthorblockA{\textit{Ritchie School of Engineering \& Computer Science} \\
\textit{University of Denver}\\
Denver, USA \\
huanghao.feng@du.deu}
\and
\IEEEauthorblockN{2\textsuperscript{nd} Mohammad H.Mahoor}
\IEEEauthorblockA{\textit{Ritchie School of Engineering \& Computer Science} \\
\textit{University of Denver}\\
Denver, USA \\
m.mahoor@du.edu}

}

\maketitle

\begin{abstract}

\end{abstract}

\begin{IEEEkeywords}
Social Robot, Autism, Music Therapy, Turn-Taking, Motor Control, Emotion Classification
\end{IEEEkeywords}


\section{Introduction}
Recent study indicates that music has played a important role in children's daily life such as 
waking time, streaming from radios, televisions, cell-phones, computers and 
toys \cite{young2007toward}. Since children with autism spent most of the time with 
technology product nowadays, music could play an important role in their life as well.
The symptoms of autism spectrum disorders, a disorder of neural development, include but not 
limited impaired social interaction and communication \cite{rapin2008autism}. In order to help
this population, different therapy methods have been developed and some are widely use in autism
recovery, such as behavior therapy, game therapy, art therapy, music therapy and more \cite{bryson2003autism}. 
Most of the time, treatment for autistic children, mediators are required because majority of them
may not able to play with kids with autism directly, for example, drawing for art therapy, game 
for game therapy and instrument for music therapy. 

Many researches shows that children with autism
have less interest in communicating with human due to sensing overwhelming issue. A robot with still
face could be a good agent with less intimidating characteristics for helping children with autism.
There is also researches show that kids with autism are more attracted to interact with humanoid 
social robots in daily life \cite{wainer2010collaborating, robins2012embodiment,costa2013your, feng2013can}.
That makes social assistive robot a perfect media for delivering certain therapy method, such as 
music therapy. Significant amount of reports suggest that using music as a assistive method, also known as music therapy, for 
helping individuals with autism can be beneficial. Composed songs and improvisational
music therapy were used as a music techniques in such activities. However, there 
was limited evidence to support the use of music interventions under certain
conditions to conduct social, communicative and behavioral skills in early age children 
with autism. Patients can get a feeling for the music by listening, singing, playing instruments, and moving. Music
therapy for children is conducted either in a one-on-one session or in a group session, and
it can help children with problems in communication, attention, and motivation, as well as 
with behavioral problems \cite{gifford2011using}. Motivation and emotion are essential to music 
education, together they ensure that students acquire new knowledge and skills in a meaningful way.
Much has been reported that music has been viewed as a means of engaging the children and 
therapists as a non-verbal aspect in musical-emotional communication \cite{warwick1991music}.

The rest of this paper is organized as follows: Section II presents some related works concerning
human-robot interaction in multiple intervention methods. Section III elaborates the experiment 
design process including hardware and session details. An intelligent music teaching platform is
presented in Section IV and experiment results are given in Section V. Finally, Section VI concludes
the paper with some remarks for future research. \\

\section{Related Works}
Music is effective method to involve children with autism in rhythmic and non-verbal
communication. Besides, music has often been used in therapeutic sessions with children who have suffer from mental and behavioral
disabilities \cite{roper2003melodic, boso2007effect}. Nowadays, at least 12\% of all treatment of individuals with autism consist
of music-based therapies \cite{bhat2013review}. Specifically, teaching and playing music to children
with autism spectrum disorders (ASD) in therapy sessions have shown great impact for improving social communication
skills \cite{lim2011effects}. Recorded music or human played back music are used in single and multiple subjects'
intervention session from many studies \cite{bhat2013review, corbett2008brief}. Different social skills are targeted and reported
(i.e. eye-gaze attention, joint attention and turn-taking activities) in using music-based therapy sessions 
\cite{stephens2008spontaneous, kim2008effects}. Noted that improving gross and fine motor skills for ASD
through music interventions is a missing part in this field of studies \cite{bhat2013review}.

Socially assistive robots are widely used in young age of autism population interventions these years. Some studies are
focusing on eye contact and joint attention \cite{feng2013can, mihalache2020perceiving, mavadati2014comparing}, 
showing that at some point the pattern of ASD group in perceiving eye gaze are similar to typically 
developed (TD) kid, and eye contact skills can be significantly improved after intervention sessions. Plus,
these findings also provides a strong evidence of ASD kids are easy to attracted to humanoid robots in
various type of social activities. Some groups start to use such robots to conduct music-based therapy
sessions nowadays. Children with autism are asked to imitate play music based on Wizard of Oz style
and Applied Behavior Analysis (ABA) models from humanoid robots in intervention sessions for practicing
eye-gaze and joint attention skills \cite{peng2014using, taheri2015impact, taheri2016social}. However, 
some disadvantages of such research due to lack of sample size and no automated system in human-robot 
interaction. Music can be used as unique window into the world of autism, lots of evidence suggest that
many individuals with ASD are able to understand simple and complex emotions in childhood using music-based
therapy sessions \cite{molnar2012music}. Although limited research has found in such area especially using
bio-signals for emotion recognition for ASD and TD kids \cite{feng2018wavelet} in understanding the 
relationship between activities and emotion changes. 

To this end, in current research a automated music-based social robot platform with activity-based emotion 
recognition system is presented in the following sections. The purpose of this platform is to provide a 
possible ultimate solution for assisting children with autism to improve motor skills, turn-taking skills 
and activity engagement initiation. Further more, by using bio-signals with Complex-Morlet (C-Morlet) wavelet feature 
extraction \cite{feng2018wavelet}, emotion classification and emotion fluctuation are analyzed based on different 
activities. TD kids are participated as control group in order to see the difference from ASD group.\\

\section{Experiment Design}
Nine ASD kids (average age: 11.73, std: 3.11) and 7 TD kids (average age: 10.22, std: 2.06) were recruited in this
study. For each participant in ASD group, 6 sessions were be delivered including baseline session,
intervention sessions and exit session. As for TD control group, only baseline and exit sessions 
were required for each participant. Each session lasts for 30-60 min total depends on the difficult
of each session and performance of individuals. \\

\subsection{Experiment Room}
All the sessions were held in a 11ft x 9.5ft x 10ft room with six HD surveillance cameras installed
at corners, side wall and ceiling of the experimental room see Figure \ref{room}. One mini hidden microphone attached at
the ceiling camera for sending real time audio to the observation room in order to let the care
giver to listen to. An external hand-held audio recorder were set in front of the participant during
sessions to be able to collect high quality audio for future process.

\begin{figure*}[tbp]
	\begin{center}
		\begin{tabular}{c}
			\epsfig{figure=./fig/room_pana.eps, scale = .25}\label{room} \\
		\end{tabular}
		\caption{Experiment Room} \label{room}
	\end{center}
\end{figure*}

As shown in Figure \ref{room}, the observation room is located at the back of the one-way
mirror facing at the back of participants in order to avoid distraction while sessions on going.
Real-time video and audio were broadcasting to the observation room during each session, which
allowed researchers observe and record in the meantime. Parents behind the mirror may also 
call off the session in case of emergency.\\

\subsection{NAO: A Humanoid Robot}
All communication content were delivered by a humanoid robot agent called NAO developed 
by Aldebaran Robotics in France. NAO is 58 cm (23 inches) tall, with 25 degrees of freedom. 
This robot can conduct most human behaviors. It also features an onboard multimedia 
system including four microphones for voice recognition and sound localization, 
two speakers for text-to-speech synthesis, and two HD cameras with maximum image 
resolution 1280 x 960 for online observation. As shown in Figure \ref{nao_body}, these 
utilities are located in the middle of the forehead and the mouth area. NAO’s 
computer vision module includes facial and shape recognition units. By using the 
vision feature of the robot, the robot can see the instrument from its lower camera and be 
able to implement color detection module for self-calibration system which allows 
the robot to have real-time micro-adjustment for its arm-joints in case of off positioning 
during xylophone playing.\\

\begin{figure}[tbp]
	\begin{center}
		\begin{tabular}{c}
			\epsfig{figure=./fig/naobody.eps, scale = .4}\label{nao_body} \\
		\end{tabular}
		\caption{A Humanoid Robot NAO: 25 Degrees of Freedom, 2 HD Cameras and 4 Microphones} \label{nao_body}
	\end{center}
\end{figure}

The robot arms have a length of approximately 31 cm. Each arm has five degrees 
of freedom and is equipped with sensors to measure the position of each 
joint. To determine the pose of the instrument and the mallets' heads, the robot 
analyzes images from the lower monocular camera located in its head, which has a 
diagonal field of view of 73 degrees. These dimensions allow us to choose a 
proper instrument.

The four microphone locations embedded on the NAO's head can be seen in Figure \ref{nao_body}. 
According to the official Aldebaran documentation, these microphones have sensitivity 
of 20mV/Pa +/-3dB at 1kHz, and an input frequency range of 150Hz - 12kHz. Data 
will be recorded as a 16 bit, 48000Hz, 4 channels wav file which meets the 
requirements for designing the online feedback audio score system described below.\\

\subsection{Hardware Accessories}
Due to the purpose of this study, some necessary accessories 
needed to be purchased and build before the robot was able to play music. 
All accessories will be discussed in the following.\\

\subsubsection{Xylophone: A Toy for Music Beginner}
In this system, due to NAO's open arms' length, we choose a Sonor Toy Sound SM 
soprano-xylophone with 11 sound bars of 2 cm in width. The instrument has a size of 
31 cm x 9.5 cm x 4 cm, including the resonating body. The smallest sound bar is 
playable in an area of 2.8 cm x 2 cm, the largest in an area of 4.8 cm x 2 cm. The 
instrument is diatonically tuned in C-Major/a-minor. For the beaters/mallets, we used 
the pair that came with the xylophone with a modified 3D printed grip (details in next 
subsection) to allow the robot's hands to hold them properly. The mallets 
are approximately 21 cm in length and include a head of 0.8 cm radius.
The 11 bars of the xylophone represent 11 different notes (11 frequencies) which covers 
approximately a one and half octave scale starting from C6 to F7. \\

\subsubsection{Mallet Gripper Design}
According to the size of Nao's hands, we designed and 3D printed a pair of gripers to 
have the robot be able to hold the mallets properly. Shown in Figure \ref{griper}.\\

\begin{figure}[tbp]
	\begin{center}
		\begin{tabular}{c}
			\epsfig{figure=./fig/grip.eps, scale = 0.4}\label{griper} \\
		\end{tabular}
		\caption{Mallet Griper} \label{griper}
	\end{center}
\end{figure}

\subsubsection{Instrument Stand Design}\label{AA}
A wooden base was designed and laser cut to hold the instrument in the proper place 
for the robot to be able to play music. Shown in Figure \ref{stand}.\\

\begin{figure}[tbp]
	\begin{center}
		\begin{tabular}{c}
			\epsfig{figure=./fig/front_view.eps, scale = 0.4} \label{front}\\
		\end{tabular}
		\caption{Instrument Stand Front View.} \label{stand}
	\end{center}
\end{figure}


\subsection{Experiment Sessions}
Two parts were included in the baseline and exit session which were 1) music practice and 2) music 
game play. In the music practice part, participants were asked to complete a full set of music play
activities including listen to the music, single note play, multiple notes play, half song play and
the whole song play. Three entertaining game modes were designed in the music game play part, participants
were allowed to communicate with robot regarding which mode to play with. Mode 1: robot will randomly
play a song from its song bank for kids to listen to; Mode 2: robot randomly generates a sequence of 
notes with consonance or dissonance style, requests an oral emotion feeling from participants and physical
playback afterwards; Mode 3: allows participants to have a 5 seconds of free play and challenge the robot to
imitate from the participants what just played. There was no limit for how many times each individual who wants to play
each time, but at least play each mode once in single session. The only difference between baseline and
exit session was the song which used in them, in baseline session, "Twinkle Twinkle Little Star" was 
used as a standard entry level song for all participants, and a customized song were chosen by each 
individual for exit session in order to motivate participants for better learning music, which makes it 
more difficult from the baseline session. 

Each intervention session has divided into three parts: S1) warm up; S2) single activity practice (with color hint); 
and S3) music game play. Starting from intervention sessions, customized song were used in the following 
sessions in order to motivate participants and have them more engage to multiple repetitive activities. 
The purpose of having warm up section is to have the motor control skill been practiced 
and meanwhile to help participants implement the motor skills in next activities. Single activity was 
based on music practice from baseline/exit session, other then those sessions, single activity will
only have one type of music practice each individual session, for instance, single note play were 
delivered in the first intervention session, then the next time this practice will become multiple notes 
play and the level of difficulty for music play were gradually increased session after session. This was in
order to make a challenge based engagement activity for ASD group for better motivation and emotion stimuli. 
As for music game play were remain the same as baseline/exit session.\\

%add a table of all activities here somewhere

\section{Xylo-Bot: An Interactive Music Education System}

In this section, a novel module-based robot-music education system will be presented. 
Three modules have been built in this intelligent system including Module 1: color-based 
self-calibration micro-adjustment system; Module 2: joint trajectory generator; and 
Module 3: real time performance scoring feedback. See Figure \ref{module}.\\

\begin{figure}[tbp]
	\begin{center}
		\begin{tabular}{c}
			\epsfig{figure=./fig/module_blocks.eps, scale = .3}\label{module} \\
		\end{tabular}
		\caption{Block Diagram of Module-Based Acustic Music Interactive System} \label{module}
	\end{center}
\end{figure}

\subsection{Module 1: Eye-hand Self-Calibration Micro-Adjustment}
Knowledge about the parameters of the robot's kinematic model is essential for 
tasks requiring high precision, such as playing xylophone. While the kinematic 
structure is known from the construction plan, errors can occur, e.g., due to the 
imperfect manufacturing. After multiple rounds of testing, it was found the targeted angle chain 
of arms never actually equals the returned chain. We therefore used a 
calibration method to accurately eliminate these errors.\\


\subsubsection{Color-Based Object Tracking}
To play the xylophone, the robot has to be able to adjust its motions according to
the estimated relative position of the instrument and the heads of the beaters it is 
holding. To estimate these poses, we used a color-based technique.
The main idea is, based on the RGB color of the center blue bar, given a hypothesis 
about the instrument's pose, one can project the contour of the object's model into the 
camera image and compare them to actually observed contour. In this way, it is possible 
to estimate the likelihood of the pose hypothesis. By using this method, it allows
the robot to track the instrument with very low cost in real-time. See Figure \ref{color_detection}.\\

\begin{figure}[bp]
	\begin{center}
		\begin{tabular}{c}
			\epsfig{figure=./fig/color_detection.eps, scale = 0.35} \label{color_detection_c}\\

		\end{tabular}
		\caption{Color Detection From NAO's Bottom Camera Color Based Edge Detection.} \label{color_detection}
	\end{center}
\end{figure}


\subsection{Module 2: Joint Trajectory Generator}
Our system parses a list of hex-decimal numbers (from 1 to b) to obtain the sequence
of notes to play. It converts the notes into a joint trajectory using the beating
configurations obtained from inverse kinematics as control points. The timestamps
for the control points will be defined by the user in order to meet the experiment requirement.
The trajectory is then computed using Bezier interpolation in joint space by the
manufacturer-provided API and then sent to the robot controller for execution. In this
way, the robot plays in-time with the song.\\

\subsection{Module 3: Real-Time Performance Scoring Feedback}
The purpose of this system is to provide a real-life interaction experience using 
music therapy to teach kids social skills and music knowledge.  In this scoring 
system, two core features were designed to complete the task: 1) music detection;
2) intelligent scoring-feedback system. All result were be saved in CSV files
for future data analysis.\\


\subsubsection{A. Music Detection}
Music, in the understanding of science and technology, can be considered as a combination 
of time and frequency. In order to make the robot detect a sequence of frequencies, we adopted the 
short-time Fourier transform (STFT) to this audio feedback system. This allows the robot to 
be able to understand the music played by users and provide the proper feedback as
a music teaching instructor.

The short-time Fourier transform (STFT) , is a Fourier-related transform used to 
determine the sinusoidal frequency and phase content of local sections of a signal 
as it changes over time. In practice, the procedure for computing STFTs is to divide 
a longer time signal into shorter segments of equal length and then compute the 
Fourier transform separately on each shorter segment. This reveals the Fourier 
spectrum on each shorter segment. In the discrete time case, the data to be transformed could 
be broken up into chunks or frames (which usually overlap each other, to reduce 
artifacts at the boundary). This can be expressed as:
\\

${\displaystyle \mathbf\{x[n]\}(m,\omega )\equiv X(m,\omega )=\sum _{n=-\infty }^{\infty }x[n]w[n-m]e^{-j\omega n}}$
\\

likewise, with signal x[n] and window w[n]. In this case, m is discrete and $\omega$ 
is continuous, but in most typical applications, the STFT is performed on a computer 
using the Fast Fourier Transform, so both variables are discrete and quantized.\\
The magnitude squared of the STFT yields the spectrogram representation of the Power 
Spectral Density of the function:
\\

${\displaystyle \operatorname {spectrogram} \{x(t)\}(\tau ,\omega )\equiv |X(\tau ,\omega )|^{2}}$\\

After the robot detects the notes from user input, a list of hex-decimal number will be
returned. This list will be used in two purposes: 1) to compare with the target list
for scoring and sending feedback to user; 2) used as a new input to have
robot playback in the game session as discussed in the next chapter.\\

\begin{figure}[tbp]
	\begin{center}
		\begin{tabular}{c}
			\epsfig{figure=./fig/stft.eps, scale = 1}\label{stft} \\
		\end{tabular}
		\caption{Melody Detection with Short Time Fourier Transform} \label{stft}
	\end{center}
\end{figure}

\subsubsection{B. Intelligent Scoring-Feedback System}
In order to compare the detected notes and the target notes, an algorithm
which is normally used in information theory linguistics called Levenshtein Distance.
This algorithm is a string metric for measuring the difference between two sequences.\\

In current case, the Levenshtein distance between two string-like hex-decimal numbers 
${\displaystyle a,b}$ (of length ${\displaystyle |a|}$ and ${\displaystyle |b|}$ respectively) 
is given by ${\displaystyle \operatorname {lev} _{a,b}(|a|,|b|)}$ where
\\

${\displaystyle \operatorname {lev} _{a,b}(i,j)={\begin{cases}\max(i,j)\\\min {\begin{cases}\operatorname {lev} _{a,b}(i-1,j)+1\\\operatorname {lev} _{a,b}(i,j-1)+1\\\operatorname {lev} _{a,b}(i-1,j-1)+1_{(a_{i}\neq b_{j})}\end{cases}}\end{cases}}}$\\
\\

where ${\displaystyle 1_{(a_{i}\neq b_{j})}}$ is the indicator function equal to 0 when 
${\displaystyle a_{i}=b_{j}}$ and equal to 1 otherwise, and ${\displaystyle \operatorname {lev} _{a,b}(i,j)}$ 
is the distance between the first ${\displaystyle i}$ characters of ${\displaystyle a}$ and the
first ${\displaystyle j}$ characters of ${\displaystyle b}$.

Note that the first element in the minimum corresponds to deletion (from ${\displaystyle a}$ to 
${\displaystyle b}$), the second to insertion and the third to match or mismatch, depending on 
whether the respective symbols are the same. 

Based on the real life situation, we defined a likelihood margin for determining whether the result
is good or bad: \\

${likelihood = \dfrac{len(target) - lev_{target,source}}{len(target)}}$\\

where if the likelihood is greater than 66\% - 72\%, the system will consider it as a good result.
This result will be passed to the accuracy calculation system to have the robot decide whether it
needs to add more dosage to the practice. More details will be discussed in the next chapters
as it relates to the experiment design.\\

\section{Experimental Results}

9 ASD and 7 TD participants finished this study in 8 months, all ASD subjects completed 
6 sessions including intervention sessions and TDs for 2 sessions with only baseline/exit session. By using Wizard of Oz control
style, a well trained researcher were conducting the baseline and exit sessions for better observation
and evaluation quality of performance. With well designed fully automated intervention sessions, NAO were
able to initiate music teaching activities with participants. 

Since the music detection method was sensitive to the audio input, that requires clear and long lasting 
sound from xylophone. From Figure \ref{warmup}, it is obvious that majority of subjects were able 
to strike or play xylophone in proper way after one or two sessions. Notice that subject 101 and 
102 had significant improvement curve during intervention sessions. Some of the subjects started 
at a higher accuracy rate, and kept this rate above 80\%, which can be considered as consistent 
motor control performance even with up and downs. Two subjects (103 \& 107) were having difficult 
time with playing xylophone and following turn-taking cues with agent robot. This fact affected 
the performance in following activities for both subjects.\\

\begin{figure*}[tbp]
	\begin{center}
		\begin{tabular}{c}
			\epsfig{figure=./fig/warm.eps, scale = 1}\label{warmup} \\
		\end{tabular}
		\caption{Motor Control Accuracy Result} \label{warmup}
	\end{center}
\end{figure*}

Figure \ref{song} shows the accuracy result of main music teaching activity for intervention sessions
across all participants. Learning how to play one's favorite song can be considered as a motivation for ASD
kids understanding and learning turn-taking skill. As described in previous section, the difficulty 
level of this activity were designed uprising. By this fact, accuracy of the performance from participants 
were expected to decrease. This activity requires participants able to concentrate and using joint attention skills
in robot teaching stage and also respond properly afterwards. Enough waiting time were given after robot
says: 'Now, you shall play right after my eye flashes', participants were also received an eye color change
cue from the robot in order to complete a desired music-based social interaction. Different from warm
up section, notes played in correct sequence of order can be considered as a good-count strike.
From Figure \ref{song}, most of the participants were able to complete single/multiple notes practice with an
average 77.36\%/69.38\% accuracy rate, although even with color hints, notes' pitch difference still can be a core 
challenge for them. Due to the difficulty of session 4 and 5, worse performance comparing to previous 
two sessions were accepted. However, more than half of the participants showed a consistent high performance
accuracy or even better result than previous sessions. Combining the report from video annotators, 6 out 
of 9 subjects showed strong engaging behavior in playing music, especially after first few sessions. Better 
learn-play turn-taking rotation were performed over time, and significant increase of performance by 3 subjects, 
reveal turn-taking skills were picked up from this activity.

\begin{figure*}[tbp]
	\begin{center}
		\begin{tabular}{c}
			\epsfig{figure=./fig/song.eps, scale = 1.05}\label{song} \\
		\end{tabular}
		\caption{Main Music Teaching Performance Accuracy} \label{song}
	\end{center}
\end{figure*}

EDA signal was also collected in this study. By using the annotation and analysis method from 
previous work \cite{feng2018wavelet}, a music-event-based emotion classification result will 
be presented below. In order to find out the emotion secret of ASD group, multiple comparison
were made after annotate the videos. Different activities may cause emotion arousal change. 
As presented above, warm up section and single activity practice section have same activity in 
different level of intensities, and game play has the lowest difficulty and more relax. 

In the first part of analysis, EDA signals were segmented into small event-based pieces according to 
the number of "conversations" in each section. One "conversation" was defined with 3 movements:
a) robot/participant demonstrates the note(s) to play; b) participant/robot repeat the note(s); 
c) robot/participant presents the result, and each segmentation last about 45 seconds. The 
continuous wavelet transform (CWT) of the data assuming complex Morlet (C-Morlet) wavelet function
was used inside a frequency range of (0.5, 50)Hz, a SVM classifier was then employed to classify
"conversation" segmentation among 3 sections using the wavelet-based features. Table \ref{tab1}
shows the classification accuracy for the SVM classifier with different kernel functions. 
As can be seen, emotion arousal change between S1 and S2, S2 and S3 can be classified using wavelet-
based feature extraction SVM classifier with average accuracy of 76\% and 70\%. With highest 64\% of accuracy
for S1 and S3, that may indicates less emotion changes between warm up and game sections. 

\begin{table*}[tbp]
	\caption{Emotion Change in Different Sections}
	\begin{center}
	\begin{tabular}{llllll}
		& Kernels                     & Accuracy & AUC & Precision & Recall \\
		\hline
		S1 vs S2       & \multirow{4}{*}{\textbf{Linear}}     & 75       & 78  & 76        & 72     \\
		S1 vs S3       &                             & 57       & 59  & 56        & 69     \\
		S2 vs S3       &                             & 69       & 72  & 64        & 86     \\
		S1 vs S2 vs S3 &                             & \multicolumn{4}{l}{52}              \\
		\hline
		S1 vs S2       & \multirow{4}{*}{\textbf{Polynomial}} & 66       & 70  & 70        & 54     \\
		S1 vs S3       &                             & 64       & 66  & 62        & 68     \\
		S2 vs S3       &                             & 65       & 68  & 62        & 79     \\
		S1 vs S2 vs S3 &                             & \multicolumn{4}{l}{50}              \\
		\hline
		S1 vs S2       & \multirow{4}{*}{\textbf{RBF}}        & 76       & 81  & 76        & 75     \\
		S1 vs S3       &                             & 57       & 62  & 57        & 69     \\
		S2 vs S3       &                             & 70       & 76  & 66        & 83     \\
		S1 vs S2 vs S3 &                             & \multicolumn{4}{l}{53}              \\
		\hline
	\end{tabular}
	\label{tab1}
	\end{center}
\end{table*}

In order to discover the emotion fluctuation inside of one task, each "conversation" section
has been carefully divided into 3 segments as described above. Each segment last about 10 - 20 seconds.
Table \ref{tab2} shows the full result of
emotion fluctuation in warm up (S1) and music practice (S2) sections from intervention session. Notice that
all of the segments cannot be classified properly using existing method. Both SVM and KNN show
the stable results. This may suggests that ASD group may have less emotion fluctuation or arousal
change once task starts even with various activities in it. Stable emotion arousal in single task
could also benefit from the proper activity content, including robot agent play music and language
usage during conversation. Friendly voice feedback was based on the performance delivered by participants
were well written and stored in memory, both positive award while receive correct input and encouragement
while play incorrect. Since emotion fluctuation can affect learning progress, less arousal change 
indicates the design of intervention session were robust. 

\begin{table*}[tbp]
	\caption{Emotion Change in Different Sections}
	\begin{center}
	\begin{tabular}{ccccccccc}
		\multicolumn{1}{l}{\multirow{3}{*}{}} & \multicolumn{8}{c}{Segmentation Comparison in Single Task}                                                                                                       \\
		\hline
		\multicolumn{1}{l}{}                  & \multicolumn{4}{c}{Warm up Section}                                                   & \multicolumn{4}{c}{Song Practice Section}                                                  \\
		\hline
		\multicolumn{1}{l}{}                  & Kernels                     & Accuracy & K value                & Accuracy & Kernels                     & Accuracy & K value                & Accuracy \\
		learn vs play                                   & \multirow{4}{*}{Linar}      & 52.62    & \multirow{4}{*}{K = 1} & 54       & \multirow{4}{*}{Linar}      & 53.79    & \multirow{4}{*}{K = 1} & 52.41    \\
		learn vs feedback                                   &                             & 53.38    &                        & 50.13    &                             & 53.1     &                        & 51.72    \\
		play vs feedback                                   &                             & 47.5     &                        & 50.38    &                             & 54.31    &                        & 50.86    \\
		learn vs play vs feedback                                 &                             & 35.08    &                        & 36.25    &                             & 35.52    &                        & 36.55    \\
		\hline
		learn vs play                                   & \multirow{4}{*}{Polynomial} & 49       & \multirow{4}{*}{K = 3} & 50.25    & \multirow{4}{*}{Polynomial} & 53.79    & \multirow{4}{*}{K = 3} & 50.69    \\
		learn vs feedback                                 &                             & 50.75    &                        & 50.13    &                             & 50.86    &                        & 50.34    \\
		play vs feedback                                   &                             & 49.87    &                        & 49.5     &                             & 49.14    &                        & 52.07    \\
		learn vs play vs feedback                                 &                             & 33.92    &                        & 35.83    &                             & 34.71    &                        & 35.29    \\
		\hline
		learn vs play                                   & \multirow{4}{*}{RBF}        & 54.38    & \multirow{4}{*}{K = 5} & 48.37    & \multirow{4}{*}{RBF}        & 50.86    & \multirow{4}{*}{K = 5} & 50.17    \\
		learn vs feedback                                   &                             & 55.75    &                        & 52.75    &                             & 53.97    &                        & 50.17    \\
		play vs feedback                                   &                             & 51.12    &                        & 50       &                             & 53.79    &                        & 52.93    \\
		learn vs play vs feedback                                 &                             & 36.83    &                        & 34.17    &                             & 34.83    &                        & 33.1    \\
		\hline
	\end{tabular}
	\label{tab2}
\end{center}
\end{table*}

Cross sections comparison also presented blow. Since each "conversation" contains 3 segments, it is 
necessary to have specific segments from one task to compare with the other task corresponded to. 
Table \ref{tab3} shows the classification rate in robot demo, kids play and robot feedback across
warm up (S1) and music practice (S2) sections. By using RBF kernel, wavelet-based SVM classification rate has
~80\% of accuracy for all 3 comparisons. This result also matches the result from Table \ref{tab1}. 

\begin{table*}[tbp]
	\caption{Emotion Change in Different Tasks}
	\begin{center}
		\begin{tabular}{lcccccc}
			\multicolumn{1}{c}{\multirow{2}{*}{}} & \multicolumn{3}{c}{Accuracy of SVM} & \multicolumn{3}{c}{Accuracy of KNN} \\
			\hline
			\multicolumn{1}{c}{}                  & Linear   & Polynomial    & RBF    & K = 1   & K = 3     & K = 5     \\
			\hline
			learn 1 vs learn 2                                 & 73.45    & 69.31   & 80.86  & 73.28   & 71.03   & 65      \\
			\hline
			play 1 vs play 2                                 & 75.34    & 68.79   & 80     & 74.48   & 69.14   & 64.31   \\
			\hline
			feedback 1 vs feedback 2                                 & 76.38    & 69.48   & 80.34  & 74.14   & 69.14   & 66.9   
		\end{tabular}
		\label{tab3}
	\end{center}
\end{table*}


The types of activities and process of the session between baseline session for both group were 
exactly the same. By using the "conversation" concept above, each of them has been segmented.
Comparing with target and control groups using same classifier, 80\% of accuracy for detecting
different groups. See Table \ref{tab3}. Video annotators also reported "unclear" in reading 
facial expressions from ASD group. These combined messages suggests that, even with same activities
different bio-reaction were completely opposite between TD and ASD groups. It has 
also been reported that, significant improvement of music performance were shown in ASD group, 
although both groups have similar performance at their baseline sessions. Further more, TD group 
were shown more willing to try to make their performance as better as possible while they made
mistakes.\\

\begin{table*}[tbp]
	\caption{TD vs ASD Emotion Changes from Baseline and Exit Sessions}
	\begin{center}
	\begin{tabular}{llllll}
	                                           & \textbf{Linear} & \textbf{Polynomial} & \textbf{RBF} \\
	\hline

	\textbf{Accuracy}                          & 75              & 62.5                & 80           \\
	\hline
	\multirow{2}{*}{\textbf{Confusion Matrix}} & 63  37          & 50  50              & 81  19       \\
											   & 12  88          & 25  75              & 25  75       \\
	\hline
	\end{tabular}
	\label{tab4}
\end{center}
\end{table*}


\section{Discussion, Conclusion and Future Work}
The results indicates that the presented music education platform can be considered as a good tool for help 
improving fine motor control, turn-taking skills and social activities engagement. The automated music 
detection system created a self-adjusting environment for participants in early sessions. Most
of the ASD kids started to pick up the strike movement after first two intervention sessions, some 
even can master the motor skill during the very first warm up activity. Although the robot could
provide verbal instructions and demonstrations by voice command input from participants whenever they need it. However, 
majority of the participants did not request such service while playing with the robot. This finding
suggests that fine motor control skill can be learned from specific well-designed activities for
young ASD population. 

The purpose of using music teaching scenario as the main activity in the current research is to 
create a fine and natural turn-taking behavior chance during social interaction. By observing all 
experimental sessions, 6 out of 9 subjects could dominate proper turn-taking after one or two
sessions. Note that subject 107 had significant improvement in last few sessions comparing to the
baseline session. Subject 109 had trouble with focus on listening to the robot for most of the time.
However, with researcher interfering, this kid can perform better back and forth music activity 
for a short time period. For practicing turn-taking skill, a fun motivated activity should be 
designed for children with autism. Music teaching could be a good example for accomplish this
task by taking the advantage of customized songs which selected by individuals.

Starting the later half of the sessions, participants can start to recognize their favorite songs,
over half of the participants were getting more into the activities, although the difficulty for
playing proper notes were much higher. It is easy to notice that older kids who spent more time engaging with the activities 
during the song practice session comparing to younger kids, especially in half/whole song play sessions.
Several reasons can explain this situation, one is because of the more complex the music,
the more challenge and more concentration participants will face. Thus, older individuals may willingly accept the 
challenge and enjoy the sense of accomplishment afterwards based on their verbal feedback to the research at the end of  
each session. The music knowledge base could also be one of the reason that conducts this result, 
since older participants may have more chance to learn music at school. Game section of each 
session provides the highest engagement level of all time, not only because of this is for 
relax and fun play, but also offers an opportunity to participants regarding challenge the 
robot to mirror the free play from them, this interesting phenomenon can be considered as 
"revenge". Especially for subject 106, who spent significant amount of time in free play 
game mode. According to the session executioner and video annotators, this particular subject shows 
high level of engagement for all activities, including free play. Based on the conversation 
and music performance with robot, subject showed strong interest in challenging the robot 
with a friendly way. 

Emotion study for children with autism is difficult. Bio-signal provides a possible way of 
doing that. Event-based emotion classification method presented in current research suggests
that same activity with different intensities can cause emotion change in arousal dimension,
although it is difficult to label the emotions based on facial expression change in video
annotation phase for ASD group. Less emotion fluctuate in certain activity presented in Table \ref{tab2} suggests that a mild 
friendly game like teaching system may motivate better social content learning for children with autism, 
even with repetitive movements. These well designed activities could provide a relaxed learning 
environment which helps participants to focus on learning music content with proper communication behaviors. 
This may explains the improvement for music play performance in song practice (S2) through intervention 
sessions in Figure \ref{song}. Comparing emotion patterns from baseline and exit sessions between TD and ASD groups in Table \ref{tab3}, 
difference can be found. This may suggests a potential way of assist autism diagnose using bio-signal 
in early age. According to annotators and observers, TD kids showed strong passion
in this research. Excitement, stressful, disappointment were easy to be recognized and labeled from
the videos recorded. On the other hand, limited facial expression changes can be detected in
ASD group. That makes it difficult to learn whether they have different feelings or they have
same feelings but different bio-signal activities comparing to TD group. This could be a interesting research to
dig into in the future. Further more, due to the limitation of the sample size, future research can be 
continued with different classification methods with larger population.\\

\section*{Acknowledgment}


%\section*{References}

\baselineskip 0.21in
\bibliographystyle{plain}
\bibliography{ref}


\end{document}
