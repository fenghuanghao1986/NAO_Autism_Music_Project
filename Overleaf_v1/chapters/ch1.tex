\chapter{Introduction}

\section{Autism Spectrum Disorders (ASD)}
Autism, defined on abnormal development of behavioral criteria such as
social interaction, communication, and imagination, is considered as a 
neurodevelopmental disorder (Kanner syndrome) \cite{kanner1944early, wing1997autistic}. 
Usually, autism could start at an early age, like infancy, at the latest, in
the first three years of life. Not using words to communicate can be the first
clue for the parents to be noticed, even the kid be able to repeat messages
from videotapes or speaks the alphabet. Social deficits
may not be seen immediately at an early age of childhood. However, it will gradually
be noticed while other children become more socially sophisticated and more
active. Children with autism usually do not have meaningful communication
with others, even when they have to. As age increase, some of the repetitive
behaviors begin to develop, for example, specific hand and finger movements, using
peripheral vision to look at objects, or forward and backward body shaking \cite{lord2000autism}.
Children with ASD could also experience deficits in inappropriate verbal and 
nonverbal communication skills, including motor control, emotional
facial expressions, and eye gaze attention \cite{RobotPlaymate2002}. 
About 1 in 54 children have been identified with ASD
according to estimates from CDC’s Autism and Developmental Disabilities 
Monitoring (ADDM) Network and government statistics suggest the prevalence rate of
ASD is increasing 10-17 percent annually \cite{maenner2020prevalence}. \\

It is often difficult for parents and professionals to recognize and 
judge the scientific validity of an intervention or treatment designed
to be used with individuals with ASD. National Research Council includes 
a list of the features the committee believes to be successful educational 
intervention services for ASD children. The components include: early age 
entry into an intervention program; active engagement in intensive instructional 
programming for the equivalent of a full school day, including services 
that may be offered in different sites, for a minimum of five days a 
week with full-year programming; use of planned teaching opportunities, 
organized around relatively brief periods for the youngest children 
(e.g., 15-20 minute intervals); and sufficient amounts of adult attention 
in one-to-one or minimal group instruction to meet individualized goals. \cite{national2001educating}
Multiple treatments for ASD population can be categories as follows:
(1) interpersonal relationship, (2) skill-based, (3) cognitive,  (4) 
physiological/biological/neurological, and (5) other interventions and 
treatments. \cite{de2004autism} \\

Some of the treatments have been proven that has significant and convincing 
support for ASD children, such as Applied Behavior Analysis (ABA) \cite{cooper2007applied}, 
Discrete Trial Teaching (DTT) \cite{sarokoff2004effects}, and Pivotal 
Response Training (PRT) \cite{pierce1995increasing}. Currently, 
ABA \cite{RollingRobot2002, MobileRobotic2002} has focused on teaching individuals with ASD
appropriate social skills in an effort to make them more successful in social situations \cite{wolff1964behavioural}.
With the concern of the growing number of children diagnosed with ASD, there is a high
demand for finding alternative solutions such as innovative computer technologies and/or
robotics to facilitate autism therapy. Therefore, research on how to design and use modern
technology that would result in clinically robust methodologies for autism intervention is
vital. Assistive Technology \cite{mirenda2001autism}, Joint Action Routines (JARS) \cite{drew2002pilot}, 
Cognitive Behavioral Modification \cite{epstein1989cognitive}, Structured Teaching \cite{schopler1995structured}, 
and Social Stories \cite{gray1993social,feng2013can, mavadati2014comparing}, 
such intervention and treatments also provide promising results for most of the cases, even 
though these methods still requires additional scientific support in the future. \cite{de2004autism}\\

In human social interaction, non-verbal facial behaviors (e.g., facial expressions,
gaze direction, and head pose orientation, etc.) convey important information between
individuals. For instance, during an interactive conversation, the peer may regulate their
facial activities and gaze directions actively to indicate their interests or boredom. However,
the majority of individuals with ASD show the lack of exploiting and understanding these
cues to communicate with others. These limiting factors have made crucial difficulties for
individuals with ASD to illustrate their emotions, feelings, and also interact with other
human beings. Studies have shown that individuals with autism are much interested to
interact with machines (e.g., computers, iPad, robots, etc.) than humans \cite{fong2003survey}. In this regard,
in the last decade, several studies have been conducted to employ machines in therapy
sessions and examine the behavioral responses of people with autism. These studies have
assisted researchers in understanding better, model and improve the social skills of individuals
on the autism spectrum.\\

With the rise in the prevalence of autism, the number of therapies for this condition 
has correspondingly increased. In general, practitioners accept the need for appropriate 
treatments. Effectiveness is usually thought to mean the use of reliable research with precise 
control over internal and external challenges to validity. Therefore, only therapies with 
constant clinical support that show effectiveness in alleviating negative autism symptomology 
should be widely disseminated for use. There are, however, many fad therapies that have no 
such evidence of efficacy. Use these therapies is wasting time and resources and preying on 
parents' and caregivers' emotional weakness. \cite{zane2008cost} Computer technology is expected 
to be increasingly used by a new generation of children in a variety of contexts (professional, 
educational and recreational), including interactive robotic toys, digitally enhanced objects 
, and tangible interfaces \cite{laurel2013computers, tapscott1998growing, cassell2000barbie, druin2000robots}. 
Modern digital technologies and modern implementations are also vulnerable to affect therapy 
and recovery methods. The physical structure and behavior of socially intelligent agents, 
demonstrating facets of social intelligence in the human form \cite{dautenhahn1998art}, are 
likely to alter how we can teach social intelligence to people who have trouble recognizing 
and expressing social behaviour.\\

A robotic platform is hoped to provide the necessary stimulation to reinforce the child's responses
according to Treatment in Education of Autistic and Related Communication Handicapped Children 
(TEACCH) treatment method. This should promote interaction by providing a pleasant stimulus, 
strengthening it by reacting in specific, non-threatening ways. A robot is expected to allow the child 
to relax and view the activity as play, reducing the amount of fear presented. It should, therefore, 
appear to be a new and interesting toy, while at the same time extending the interactive and communicative 
limits of the individual child through a playable medium. Bridging the gap between the inner 
world of autism and the unpredictable yet appropriate teacher, thereby offering a stable method of 
educating the child about the fundamentals of interaction in a gradual manner and adapting to the 
child's development should also be done by the robot platform \cite{werry1999applying}.\\

This dissertation presents the methodology and results of a study that aimed to design a 
autonomous human-robot interaction education platform for capturing, modeling, and enhancing
the social skills of children with autism. Such a platform should complete the following requirements:
(1) fully autonomous to conduct an intervention session, (2) provide a life-like teaching-learning
environment scenario, (3) in particular aiming motor control and turn-taking skills improvement,
(4) stimulate emotional change in different social activities, and (5) be able to investigate
how ASD and Typically Developing (TD) children react to such an education platform with a humanoid
robot. In the following section, a brief introduction of existing assistive robots that have been
used in autism applications will be introduced. \\

\section{Socially Assistive Robotics (SAR)}
SAR can be considered as the intersection of Assistive
Robotics (AR) and Socially Interactive Robotics (SIR), which has referred to robots that
assist human with physical deficits and also can provide certain terms of social interaction
abilities \cite{feil2005defining}. SAR includes all the characteristics of the SIR mentioned in it \cite{fong2003survey}, 
as well as a few additional attributes such as 1) user populations (e.g., elders; individuals
with physical impairments; kids diagnosed with ASD; students); 2) social skills (e.g., speech
ability; gestures movement); 3) objective tasks (e.g., tutoring; physical therapy; daily life
assistance); 4) robot function (depends on the task the robot has been assigned for) \cite{feil2005defining}.
Companion robots \cite{AnalysisFactor2002} is one type of SAR that is widely used for older adults
for health care supports. Research shows that this type of social robot can reduce the stress
and depression of individuals in the elderly stage \cite{AnimalAssist2002}. Service social robots are able to
accomplish a variety of tasks for individuals with physical impairments \cite{Fetch2002}. Studies have
shown that SAR can be used in therapy sessions for those individuals who suffer from
cognitive and behavioral disorders (e.g., autism). SAR provides an efficient, helpful
medium to teach certain types of skills to these groups of individuals \cite{RobotPlaymate2002, RollingRobot2002, MobileRobotic2002}.\\

Nowadays, there are very few companies that have designed and developed socially beneficial 
robots. The majority of existing SARs are not yet commercialized, and because they are expensive 
and not well-designed user interfaces, they are mostly used for research purposes. Honda, SoftBank 
Robotics and Hanson Robokind are the leading companies that are currently developing humanoid robots. 
Ideally, socially helpful robots can have fully automated systems for detecting and expressing 
social behavior while interacting with humans. Some of the existing robot-human interfaces 
are semi-autonomous and can recognize some basic biometrics (e.g., user visual and audio 
commands) and behavioral responses. In addition, most of the existing robots are very complicated 
to work with. As a result, in the last few years, several companies have begun to make these robots 
more user-friendly and responsive to both user needs and potential caregivers' commands. \cite{feil2005defining}.
In all, service social robots are able to do a variety of tasks for individuals with physical impairments. 
SAR can be used in therapy sessions for those individuals who have autism. SAR provides an 
efficient, helpful medium to teach certain types of skills to these groups of individuals.\\

Intelligent SARs strive at being able to understand visual or auditory instructions, objects, 
and basic human movements. Any of these robots have the power to identify human faces or simple 
facial expressions. For example, ASIMO, a robot created by Honda, the company, has a sensor for detecting 
movements of multiple objects using visual information obtained from two cameras on its head.
Besides, its "eyes" will determine the distance between objects and robots. \cite{ASIMO2011}
Another example is Softbank Robotics, which builds small-scale humanoid robots called the NAO. 
The NAO robot has two cameras mounted to it that are used to take single photographs and video 
sequences. This capture module enables NAO to see and recognize the different sides of an object 
for future use. Besides, NAO has a remarkable ability to recognize faces and to detect moving 
objects. More details will be discussed in the following chapters. The speech recognition system
has been embedded in both of the aforementioned robots, which provide a strong voice communication
ability to accomplish more natural social interaction with human beings. NAO is able to understand
words and sentences which have been pre-programmed in the memory for running specific commands. 
However, ASIMO is able to distinguish between voices and other sounds. This feature empowers
ASIMO to perceive the direction of a human’s speaker or recognize other companion robots
by tracking their voice \cite{DSMIV2000}. Several language packages can be installed into
NAO, which feature gives the robot a strong social communication functionality to interact
world widely. \\

\subsection{Socially Assistive Robots for Autism Therapy}
Socially assistive robots are emerging technologies in the field of robotics that aim
to utilize social robots to increase the engagement of users as communicating with robots, and
elicit novel social behaviors through their interaction. One of the goals in SAR is to use
social robots either individually or in conjunction with caregivers to improve the social skills
of individuals who have social, behavioral deficits. One of the early applications of SAR is
autism rehabilitation. As mentioned before, autism is a spectrum of complex
developmental brain disorders, causing qualitative impairments in social interaction.
Children with ASD experience deficits in inappropriate verbal and nonverbal communication
skills, including motor control, emotional facial expressions, and gaze regulation. These
skill deficits often pose problems in the individual’s ability to establish and maintain social
relationships and may lead to anxiety surrounding social contexts and behaviors \cite{wolff1964behavioural}.
Unfortunately, there is no single accepted intervention, treatment, or known cure for
individuals with ASD.\\

Recent research suggests that children with autism exhibit certain positive social
behaviors when interacting with robots compared to their peers that do not interact with
robots \cite{pierno2008robotic, villano2011domer, feil2005defining, fong2003survey}. 
These positive behaviors include showing emotional facial
expressions (e.g., smiling), gesture imitation and eye gaze attention. Studies show that
these behaviors are rare in children with autism, but evidence suggests that robots trigger
children to demonstrate such practices. These investigations propose that interaction with
robots may be a promising approach for rehabilitation of children with ASD.\\

Several research groups investigated the response of children with
autism to both humanoid robots and non-humanoid toy-like robots in the hope that these
systems will be useful for understanding affective, communicative, and social differences
seen in individuals with ASD (see Diehl et al., \cite{fong2003survey}), and to utilize robotic systems to develop
novel interventions and enhance existing treatments for children with ASD \cite{ASIMO2011, DSMIV2000, DoesMatter2006}.
Mazzei et al. \cite{dautenhahn2004towards}, for example, designed the robot “FACE” to show the details realistically
of human facial expressions. A combination of hardware, wearable devices, and software
algorithms measured the subject’s affective states (e.g., eye gaze attention, facial expressions, vital signals, 
skin temperature and EDA signals), were used for controlling the robot
reactions and responses.\\

Reviewing the literature in SAR \cite{feil2005defining, fong2003survey} shows that there are surprisingly very few
studies that used an autonomous robot to model, teach, or practice the social skills of
individuals with autism. Amongst, explaining how to regulate eye-gaze attention, perceiving
, and expressing emotional facial expressions are the most important ones. Designing robust
interactive games and employing a reliable social robot that can sense users’ socioemotional
behaviors and can respond to emotions through facial expressions or speech is
an exciting area of research. In addition, the therapeutic applications of social robots
impose conditions on the robot’s requirements, feedback model, and user interface. In other
words, the robot that aims for autism therapy may not be directly used for depression
treatment and hence every SAR application requires its attention, research, and
development.\\
 
Only a few adaptive robot-based interaction settings have been designed and
employed for communication with children with ASD. Proximity-based closed-loop
robotic interaction \cite{LookApproach1972}, haptic interaction \cite{DiffEffect1966}, and adaptive game interactions based on
affective cues inferred from physiological signals \cite{SysObserv1968} are some of these studies. Although
all of these studies were conducted to analyze the functionality of robots for socially interacting with individuals 
with ASD, these paradigms were limited explored and
focused on their core deficits (i.e., Facial expression, eye gaze, and joint attention skills).
Bekele and colleagues \cite{AutisticDist1943} studied the development and application of a humanoid
robotic system capable of intelligently administering joint attention prompts and adaptively
responding based on within system measurements of gaze and attention. They found out
that preschool children with ASD have more frequent eye contact toward the humanoid
robot agent, and also more accurate response in joint attention stimulation. This suggests
that robotic systems have the enhancements for successfully improve the coordinated
attention in kids with ASD.\\

Considering the existing SAR system and the significant social deficits that individuals
with autism may have, we have designed and conducted robot-based therapeutic sessions
that are focused on different aspects of the social skills of children with autism. In this thesis,
we employed NAO, which can autonomously communicate with the children.
We conducted two different designs to examine the music social skills of children with autism
and provide feedback to improve their behavioral responses. \\

\section{Music Therapy in ASD Treatment}
Early pioneers in the 1940s, music therapy were used in psychiatric hospitals, institutions, 
and schools for children with autism. Back in that time, since both autism diagnosis and 
the music therapy profession were emerging simultaneously, there was no official documentation
in such a field can be found. In the 1950s, the apparent unusual musical abilities of children with 
autism intrigued many music therapists. By the end of the 1960s, music therapists started delineating 
goals and objectives. The beginning of the 1970s encountered the emergence of theoretically 
grounded music therapists working toward a more clearly defined approach to improving the lives 
of children with autism. "A great deal of research needs to be done in many directions. For the 
present, we have to use whatever approach has some value, and from our experience, there is 
no doubt, music therapy has value" \cite{reschke2011history}. However, for decades, music therapists 
are not using a consistent assessment method with autism spectrum disorder clients. The lack of a 
quality, universal assessment tool has caused difficulty for music therapists. Music therapists are 
in danger of activity-based, non-goal driven treatment. Without a common language, it is difficult 
for music therapy to be recognized as a valid, evidence-based approach \cite{thaut2000scientific}. 
Music therapists have continued to implement many of the techniques of the preceding few decades in 
recent years, such as music games and singing music as a reinforcement \cite{starr1998understanding, dellatan2003use}. 
The spectrum of therapeutic strategies has since been expanded to involve family-based music 
therapy prescriptive songs and to include clients and parents with music therapy services 
for use beyond music therapy \cite{brownell2002musically, kern2006using, katagiri2009effect}.\\

In order to deliver a solid music therapy intervention solution with a consistent assessment
method with ASD children, a humanoid social assistive robot could be a perfect choice. Many researches 
show that children with autism have less interest in communicating with humans due to sensing 
overwhelming issues. A robot with a still face could be a good agent with less intimidating 
characteristics for helping children with autism. There are also researches show that 
kids with autism are more attracted to interact with humanoid social robots in daily life 
\cite{wainer2010collaborating, robins2012embodiment,costa2013your, feng2013can}.
That makes the socially assistive robot a perfect medium for delivering certain therapy methods, such as 
music therapy. A significant amount of reports suggest that using music as an assistive method, also 
known as music therapy, for helping individuals with autism can be beneficial. Composed songs and 
improvisational music therapy have been used as performance strategies in these practices. However, 
there was limited evidence to support the use of music interventions to conduct social, communicative, 
and behavioral skills in children with autism at an early age under certain conditions. By listening, 
singing, playing instruments, and moving, patients can get a feeling for the music. Children's music 
therapy is performed either in a one-on-one session or a group session. It can help children with 
communication, attention, and motivation problems as well as behavioral issues \cite{gifford2011using}. 
Motivation and emotion are essential to music education, and together they ensure that students 
acquire new knowledge and skills in a meaningful way. Much has been reported that music has been 
viewed as a means of engaging the children and therapists as a non-verbal aspect in musical-emotional 
communication \cite{warwick1991music}.\\

\section{Contributions}
The major contributions of this dissertation are as follows:
\bi
\item 
Developing a wavelet-based approach to event-based emotion classification
using Electrodermal activity signal from early age children. In our work, 
the dataset is first annotated to label perceived emotions (e.g., Acceptance, 
Joy, Boredom) expressed by each subject. Afterward, we utilize the continuous 
wavelet transform to develop a new feature space for classification purposes. 
Using the complex Morlet function, the wavelet coefficients of the EDA signal 
at different scales are calculated, providing a more detailed representation of 
the input signal. The performance of the proposed feature space on emotion 
classification task is evaluated using the canonical support vector machine 
(SVM) classifier with different types of kernel functions as well as the 
K-nearest neighborhood (KNN) classifier. And this method is applied
to music teaching/playing therapy intervention for a better
understanding of emotional engagement.\\

\item 
Developing an autonomous social interactive robot music teaching system
for children with autism. A novel module-based robot-music teaching system will be presented. 
Three modules have been built in this intelligent system including module 1: eye-hand 
self-calibration micro-adjustment to prevent a minor change of relative position
between a musical instrument and robot; module 2: joint trajectory generator to 
play any meaningful customized melody; and module 3: real-time performance scoring 
feedback using short-time Fourier transform and Levenshtein distance to provide
an autonomous real-time music learning experience.\\

\item 
Designing a new instrument call X-Elophone, which allows users to create
more types of melody. This unique design brings more possibilities for young
children who are willing to learn music and music emotion understanding.\\


\item 
Proposing a set of music teaching session using a humanoid social robot
NAO to deliver a unique music teaching experience to kids with autism.
After intervention sessions, participants will be able to have better
eye-gaze/joint attention performance, better motor control skills and 
better music understanding ability. By using newly designed X-Elophone, 
participants would learn music
emotions.\\

\ei


\section{Organization}
This dissertation is organized as follows: Chapter 2 presents related work related to autism spectrum disorders, emotions classification, music
therapy in autism treatment, and social robots in autism therapy. Chapter 3 introduces a wavelet-based feature extraction approach for emotion 
classification as a pre-study for music interaction emotion recognition. Chapter 4 explains a novel approach in designing the autonomous social interactive robot
music teaching system with experimental session design. Chapter 5 illustrates all the experimental results. Finally, Chapter 6 presents X-Elophone, a new instrument for music playing and Chapter 7 concludes the dissertation with some discussions, remarks, and proposed future work.

7