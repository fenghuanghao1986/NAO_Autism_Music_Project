\chapter{Related Works in Autism and Robots}

\section{Autism}
Verbal and non-verbal communication impairments have often been associated with individuals 
autism spectrum disorder, who has experience specific deficits including language delay, 
social communication issues, emotion recognition, and eye gaze attention, etc. Autism
is a disorder that appears in infancy \cite{lotter1966epidemiology}. Some of the kids
were diagnosed as high-functioning autism, even though, some of the social areas sill 
can be difficult to them such as (1) fine motor control (e.g., unable to perform precisely 
handy work), (2) having difficulty in understanding emotions from others (e.g., no empathy
skills or not be able to read/perform proper facial expressions) and more remarkably, (3)
joint attention (e.g., less eye contact and eye gaze attention)\cite{lotter1966epidemiology}.
It is known that no single accepted intervention, treatment, or cure for ASDs; however, a 
successful treatment and better recovery would be performed if intervention been delivered
in early diagnosis stages. Almost no clue can be found at a very early stage of individuals with 
autism; however, signs may emerge after trying to interact with them for a certain period of time.
The first thing that may be noticed is not responding by calling their names, while communicating
eye contact may not present as well. Repetitive abnormal body movement may also appear, for
example, body rocking overtimes or head banging against the wall, which some of the gestures
may hurt them permanently. In the early 1990s, researchers in the University of California at 
San Diego aimed to find out the connections between autism and the nervous system (i.e., mirror 
neurons). Mirror neuron \cite{ramachandran2006broken} is a neuron that is activated 
either when a human acts an action or observes the same action performed by others. 
As these neurons are involved with the abilities such as empathy and perception of 
other individual’s intentions or emotions, they came up with malfunctioning of a mirror 
neuron in individuals with ASD \cite{ramachandran2006broken}. There are several studies
that focus on the neurological deficits of individuals with autism and studying their
brain activities. Figure \ref{brains} demonstrates brain activity difference between groups
in forward speech \cite{redcay2008deviant}. \\

\begin{figure}[tbp]
	\begin{center}
		\begin{tabular}{c}
			\epsfig{figure=./chapters/fig/brains.eps, scale = .25}\label{brains} \\
		\end{tabular}
		\caption{Chronological age-matched, ASD and Mental age-matched brain activities in forward speech \cite{redcay2008deviant}.} 
	\end{center}\label{brains}
\end{figure}

Individuals with autism might also have several other unusual social developmental
behaviors that may appear in infancy or childhood. For instance, children with autism show
less attention to social stimuli (e.g., facial expressions, joint attention), and respond less
when calling their names. Compared with typically developing children, older children or
adults with autism can read facial expressions less effectively and recognize emotions
behind specific facial expressions or the tone of voice with difficulties \cite{popper2005logic}. 
In contrast to TD individuals, children with autism (e.g., high-functioning, Asperger syndrome) may be
overwhelmed with social signals such as facial behaviors and expression and complexity
of them, and they suffer from interacting with other individuals. Therefore they would prefer
to be alone. That is why it would be difficult for individuals with autism to maintain social
interaction with other \cite{bartak1975comparative}.\\

\subsection{Motor Control}
According to previous researches, some of the impairments seem not defined as core features in 
ASD, such as motor control or turn-taking skills. However, it has been widely accepted that 
these skills are nevertheless high prevalent and can have a significant impact on improving 
social life for individuals with autism \cite{gowen2013motor}. Recent studies show that individuals
with autism can be observed with abnormal motor skills in the early age stage \cite{teitelbaum1998movement, provost2007comparison, brian2008clinical}.
This deficit sticks with them throughout childhood and even in their adulthood as 
well as \cite{fournier2010motor, ming2007prevalence, van2010stability}. It has been reported that
the prevalence of motor skill deficits is between 21 and 100 \% \cite{ghaziuddin1994clumsiness, manjiviona1995comparison, miyahara1997brief},
which highlighted that motor control problem is a significant but potentially variable aspect of ASD.
Research showed that motor ability is correlated with daily living skills in children with autism \cite{jasmin2009sensori}, 
and in order to decrease the severity of ASDs in their future life which requires better motor control 
skills for more practice in early age \cite{sutera2007predictors}. To this end, increasing the 
understanding of the etiology of motor deficits in ASD is, therefore, a crucial step towards treating 
this potential developmental cascade and preventing that \cite{gowen2013motor}.\\

Motor control is systematic movement regulation in organisms that have a nervous system. 
The motor regulation involves aspects of movement that can be related to reflex \cite{wolff1964behavioural}. 
Motor control as a field of study is essentially a psychology or neurology sub-discipline.
Recent motor control psychological theories present it as a process through which humans 
and animals use their brain/cognition to activate and coordinate the muscles and limbs 
involved in performing a motor skill. Through this mixed psychological viewpoint, motor 
control is simply the integration of sensory input, both about the environment and the 
actual state of the body, to decide the correct collection of muscle forces and joint 
activation to produce any desired movement or motion. This process involves mutual 
cooperation between the central nervous system and the musculoskeletal system and is 
, therefore, a question of information processing, communication, dynamics, physics, and 
cognition \cite{pierno2008robotic, tang2011enhancing}. Effective motor control is essential 
for communicating with the environment, not only deciding capabilities for intervention 
but also controlling equilibrium and stability. Although the modern motor control analysis 
is an increasingly interdisciplinary field, research issues have been traditionally 
described as either physiological or psychological, depending on whether the emphasis 
is on physical and biological properties, or organizational and systemic rules \cite{villano2011domer}.
Research areas related to motor control include motor synchronization, motor learning, 
signal processing and the theory of perceptual function.\\

Hippotherapy (HPOT) \cite{ajzenman2013effect} as a treatment strategy that uses the 
horse's movement as a tool to affect functional outcomes in autism therapy. While
offering necessary support in challenging the cognitive-sensorimotor system, HPOT 
also considers the context of the therapy sessions, which makes it a unique treatment
strategy for children with ASD \cite{engel2007enhancing}. Also, active engagement 
helps to improve in adaptation and increased willingness to participate in daily
activities after the therapy sessions \cite{brown2010relationship}, and similar 
effectiveness in using the horse's movements in the HPOT treatment tool \cite{ajzenman2013effect}. 
In \cite{ajzenman2013effect}, children with ASD showed improved postural stability and 
improvements in receptive communication, coping, and daily activity participation after 
12 weekly HPOT sessions. Researchers also claim that ASD kids may possibly pick up 
automatic postural mechanisms to better adapt to the therapeutic activities due to 
randomly changed stability from the horse \cite{ajzenman2013effect}. Gross motor (GM) 
and fine motor (FM) development in children with ASD has also been studied in \cite{provost2007levels}. 
By comparing with children with developmental delay (DD) without ASD, useful results
were found. In a total of 38 children, half ASD half DD was assessed using the Peabody
Developmental Motor Scales, Second Edition (PDMS-2). Each participant was requested to 
complete one motor control activity based on his/her levels of skills. Research showed
that most of the ASD children had similar levels of GM and FM development, the analogous
result also reflects the two groups DD and ASD, they both show identical motor skills
development \cite{provost2007levels}.\\

\subsection{Turn-Taking Skills}
Social communication can be initiated by typically developing kids in their infant 
stage \cite{neel1990social}. Eye contact, initiate turn-taking communicative exchanges 
, and play tricks with someone who familiar with, such social skills are frequently used
by kids as well \cite{brazelton1974origins, trevarthen1978secondary, reddy1991playing}.
However, these skills can be impossible for children with autism, and can be noticed at
7 to 9 months old, according to \cite{lord1984development, mundy1986defining}. Research
also found that it is impaired to utilizing appropriate turn-taking interaction skills
or fluent verbal interchanges and play turns between partners while communicating for 
children with ASD \cite{kaczmarek2002assessment}. Moreover, these communicative behaviors 
have been linked to important developmental outcomes in children with ASD \cite{mcduffie2004relationship, sigman1999continuity, stone2001predicting}. Some researchers have argued that improving turning and initiating 
joint attention can reduce ASD's severity since social reciprocity is one of the core 
deficits of autism \cite{aldred2004new, mundy1997joint}. It is also found that in difficulty 
utilizing proper turn-taking behaviors for preschoolers with ASD due to the lack of core 
social communication skills. This can cause interactive issues with their communicative 
partners in fluency interchanges or verbal and play turns in daily life \cite{kaczmarek2002assessment, edition2013diagnostic}.\\

Turn-taking is a form of conversational and dialogue organization, where members talk 
in alternating turns one at a time. In practice, it includes processes to create inputs, 
respond to previous comments, and move to another speaker using a variety of linguistic 
and non-linguistic indications \cite{wolff1964behavioural}. While the arrangement is 
substantially uniform, that is, simultaneous talk is usually avoided, and silence is 
reduced between turns, rules for turn-taking differ by culture and society. In specific 
ways, norms vary, such as how rolls are handled, how shifts are indicated, or how long the 
typical distance between turns is \cite{pierno2008robotic, tang2011enhancing}. Conversation 
turns are a desirable way of engaging in social life in many ways, and thus subject to 
rivalry \cite{villano2011domer}. Turn-taking approaches are also thought to vary by class; 
thus, turn-taking has become a topic of intensive analysis in gender studies. Although early 
work backed gendered assumptions, such as men interrupting more than women and women chatting 
more than men, recent research has found inconsistent evidence of gender-specific conversational 
approaches, and few consistent trends have emerged \cite{feil2005defining, fong2003survey}. \\

A core component for targeting turn-taking behaviors of children with autism is early 
intervention treatment. Few reasons can explain this: (1) the back-and-forth shared structure
is considered as a critical framework for early studying, (2) social acceptance in preschoolers
are highly connected with turn-taking behaviors \cite{diamond2008promoting, guralnick1997designing, harrist2002dyadic, rieth2014identifying}.
Nonetheless, behavioral approaches to enhance turn-taking habits have scarcely been tested 
empirically and quantitatively through experiment monitoring as opposed to measures to develop 
abilities in communicative, emotional, and behavior \cite{brok2010engaging, diehl2012clinical, rieth2014identifying, scassellati2012robots}.\\

One of the therapeutic treatment is called Responsive Education and Prelinguistic Milieu
Teaching (RPMT). Unfortunately, it is hard to find publications related to turn-taking
on the efficacy of RPMT. However, the RPMT directly teaches object exchange as a means
of turn-taking. Past work suggests that the RPMT is successful in promoting interventions 
of non-autistic children with mixed etiology developmental delays and in encouraging the 
introduction of joint treatment in children with developmental delays with originally unfortunate 
introduction of joint attention \cite{yoder2002effects}. While this would appear to bode 
well for children with ASD, it should be remembered that the children used, even more, 
facilitating shared focus in this previous study initially than most young children 
with ASD do. It is not clear that the RPMT facilitates the initiation of joint attention 
in deficient motivated children to communicate for care or social connection \cite{yoder2006randomized}.
The effectiveness of LEGO© therapy has been examined by a group of researchers recently \cite{legoff2004use, legoff2006long, owens2008lego}.
Participants are encouraged to use both verbal and visual input to learn social skills 
by constructing LEGO © constructs within a community or adult environment. One of the 
drawbacks of the current literature is that studies have focused on developing cognitive 
competence in high functioning autism (HFA) or Asperger's syndrome (AS) in school-age 
children and teenagers. No scientific data confirm the efficacy of therapy for young 
children with autism spectrum disorders on turn-taking behaviors \cite{kim2015case}.
Another research has found in examined the turn-taking behaviors in young children with 
autism is in \cite{rieth2014identifying}, which tested the effect of different types of 
turn-taking on language and play skills. Based on the Pivotal Response Training (PRT), 
Four types of turn-taking skills have been examined. This was found that the turn-taking 
actions of the educator favorably impacted children with autism 's sensitivity. Specifically, 
the guiding tools of the teacher and having a predetermined reaction from the subject child 
were the two main factors that dictated play and language skills development.\\

\subsection{Music Therapy}
Music is an effective method to involve children with autism in rhythmic and non-verbal
communication. Besides, music has often been used in therapeutic sessions with children who 
have suffered from mental and behavioral disabilities \cite{roper2003melodic, boso2007effect}. 
Nowadays, at least 12\% of all treatment of individuals with autism consists of music-based 
therapies \cite{bhat2013review}. Specifically, teaching and playing music to children
with autism spectrum disorders (ASD) in therapy sessions have shown a great impact on improving 
social communication skills \cite{lim2011effects}. Recorded music or human played back music 
are used in single and multiple subjects' intervention sessions from many studies \cite{bhat2013review, corbett2008brief}. 
Different social skills are targeted and reported (i.e., eye-gaze attention, joint attention 
and turn-taking activities) in using music-based therapy sessions \cite{stephens2008spontaneous, kim2008effects}. 
Noted that improving gross and fine motor skills for ASD through music interventions is a 
missing part of this field of studies \cite{bhat2013review}. \\

Early affective activity evolves into relationships in the regular boy, where games dominate \cite{reddy1997communication}. 
An adult who is familiar with the child attempts to engage with him or her through play during 
the musical interaction therapy. The purpose of musical engagement therapy is to create and 
improve any sociability the child may have by making music that offers fun opportunities for 
the child and familiar adults to come together and experience a mutual interest by developing 
a musical conversation. The accompanying live music strengthens both the actions of the carer 
and the understanding of that by the infant. Jordan and Libby comment that ‘Music is 
usually helpful to children with autism in that it seems to add both interest and meaning to 
social situations where they would otherwise be lacking’ \cite{jordan2011developing}. The musician 
is prepared to fill in, support, or enhance the role of either partner in what begins as a preverbal 
discourse. Within this case, the use of 'service' suggests that the music is part of the connection 
of both making it more explicit and keeping the series together \cite{wimpory1999musical}. 
The caregiver and musician aim to build a gift-and-take communication experience between the caregiver 
and the child. Such knowledge may allow the child to communicate with willfulness. The caregiver 
tries to adapt the volume and pacing of the feedback to the degree of responsiveness of the infant 
by being attentive to non-verbal signals and facial gestures of the infant \cite{burford1988action}.
Wimpory and Nash (in press) identified three themes at every stage of their active process that runs 
through musical interaction therapy. These topics include the scaffolding \cite{bruner1985child} of 
caregiver interaction that affords communicative control to the child. The contributions and child's 
efforts (whether intended or not) provide artistic encouragement from the artist who deals with them 
on a clinical basis. The musician also provides scaffolding, but in time does so fewer \cite{wimpory1999musical}.\\

\section{Human Robot Interaction in Autism}
Children with ASD experience deficits in inappropriate verbal and non-verbal communication skills 
including motor control, emotional facial expressions, eye-gaze attention, and joint attention.
Many studies have been conducted to identify therapeutic methods that can benefit children with 
ASD \cite{ricks2010trends}. However, only a few groups used humanoid robots for teaching or practicing social 
communication skills \cite{feil2008robot, dautenhahn2004towards, robins2005robotic, robins2006does, kozima2005interactive, pioggia2005android}.\\

For some of the social behaviors, such as eye contact, joint attention, facial expressions 
recognition, that are rarely seen in interactions of children ASD, several pieces of evidence suggest that 
robots can trigger them more effectively than human \cite{dautenhahn2000issues}. Researchers observed that individuals 
with ASD have more interest in a robot therapeutic partner than a human. In most cases, participants 
showed better speech and movement imitation compared with the response to a human partner \cite{werry2001investigating}. Although a 
recent case study \cite{ricks2010trends}, which was done by Ricks (2010) suggests that this approach might have clinical 
utility, still this area is obviously in its infancy. Studies have shown that positive feedback from 
the robot on the participants’ performance is an effective way to encourage children with ASD to 
communicate more \cite{ricks2010trends}. Other studies have also examined the use of affect recognition (e.g., emotional 
state, arousal level) based on psychophysiological responses to modify the behaviors during a robotic 
game. However, there is limited information on the utility of humanoid robots’ positive feedback in 
interventions for individuals with ASD.

\subsection{Interactive and Therapeutic Robots Designs for Autism}
Different types of robots have been used in autism research for various purposes. Some researchers 
have been attempting to utilize a realistic human appearance \cite{robins2005robotic}, while others have 
created robots with very mechanical forms \cite{dautenhahn2004towards}, and others have developed robots 
with a cartoonish or animal form \cite{kozima2005interactive}. Generally speaking different categories of 
the robot that has been used for autism research can be placed either into Non-Humanoid and Humanoid robots 
group \cite{ricks2010trends}, which will be explained in the following sections.

\subsubsection{Non-Humanoid Robots}
Non-humanoid robots are those robots that do not have the same body joint and facial appearance as 
a human does. It contains those animals like cartoonish, or non-human like appearances. These robots 
have been used by several researchers in the last two decades. This category of robots is generally 
easier to design and develop and less expensive; therefore, several initial robot-human interaction 
for individuals with ASD was conducted by non-humanoid robots. The bubble-blowing robot at USC (while 
children approached it, the robot will node head make voice or blow a bubble from the lower part of robot 
body), for instance, was not a human form robot and can be built simply \cite{feil2008robot}. Another non-humanoid 
robot used by researchers from the University of Hertfordshire called Labo-1 \cite{dautenhahn2004towards}, which can play tag 
games (tip you’re it or tig), with children. (In the game, several children play with the robot 
together, the robot uses its heat sensor to approach kids as a type of interaction.) At Yale University, 
researchers were using a mobile robotic dinosaur named Pleo, who can show emotions and desires by 
using its sounds and body movements. Children in the clinic have been helped by Pleo’s pet-like 
appearance, expressiveness, and versatility. The reason why researchers using non-humanoid robots 
is that they found out that when children with ASD see humans, they usually will choose to avoid 
and not to interact with them. On the contrary, an animal shape or toy shape robot would be more accessible 
for kids to engage with and have a better interaction.\\

\subsubsection{Humanoid Robots}
Humanoid robots generally provide the human-like appearance and consist of body parts such as 
humanoid head, body, and arms. The advanced humanoid robot would be able to move different parts of 
its body to walk or dance (NAO). Some of the humanoid robots also have the capability to show facial 
expressions (e.g., ZENO). This type of robot, unlike non-humanoid robot, they have the ability to 
accomplish more complicated social communication tasks than non-humanoid robot, but those tasks 
will be less complicated than human-human interaction. This capability can help us to design 
interaction sessions and therapeutic sessions for children with autism and assist them in improving 
their social behaviors.\\

Robins from the University of Hertfordshire, who is one of the pioneers which employed a study to 
evaluate the importance of the robot’s appearance for autism research. A doll-like robot called Robota 
was asked to interact with children with autism \cite{robins2005robotic}. This example shows that children appeared to 
be more interested in interaction with less-human like robots. Researchers conclude that children 
with ASD would prefer a simple non-complexity and fewer details of humans but still hold the humanoid 
form. So, a robot called KASPAR has been developed by Robins to fit this design criteria \cite{robins2006does}.
Similar conclusions have been made by researchers at the National Institute of Information and 
Communications Technology (NIICT) in Japan. They found out that when kids with ASD have interaction 
with their designed robot called Infanoid, the children tend to pay more attention to the mechanical 
parts of the robot’s body than communicating with the robot itself \cite{ricks2010trends}. A small soft snowman-shaped 
robot, called Keepon, was designed to represent as a simple, repeatable, mechanical robot regarding 
the reason mentioned above \cite{kozima2005interactive}. Keepon can express its emotions conveyed by shaking, rocking, and 
bobbing up and down, which can be used as a super fun toy companion for kids with ASD. Another humanoid 
robot that has been designed by researchers at the University of Pisa is known as FACE. The purpose 
of their project is to create a robot as realistic as possible to a human face for evaluating how humans 
react as the FACE displays different expressions \cite{pioggia2005android}. (During the sessions, the child (IQ around 85) with 
autism did not show any interest in FACE at the beginning. However, with the verbal suggestion, the kid replied 
to the expression by using the word “damsel” which is from a fairy tale, though the FACE showing a sad 
expression on it.) This study suggested that by using FACE, it is possible to extend emotional recognition 
skills to children with autism. In the last few years, several different types of non-humanoid and 
humanoid robots have been used for autism therapeutics sessions that we will discuss them in 
the next session.\\

\subsection{Different Therapeutic approaches for Individuals with ASD}
Different individuals with autism might suffer from various types of social 
or developmental behavior. Therefore to have an effective therapeutic intervention setting, we 
need to focus on multiple tasks and treatments. Bellow we will provide different intervention aspects 
that the majority of children with ASD may suffer from.\\

\subsubsection{Self-Initiated Interactions}
The difficulty of initiating a social conversation or interaction is one of the impaired social skills 
of children with ASD. This problem may represent a difficulty in conveying what they want and why 
they want it. For example, when a child at an early age intends to urinate, he might have to ask for 
parent’s help rather than hold it there or let it be. Clinicians try to encourage those kids to ask 
to play certain toys, and a reward will be given after they did it. Instead of human therapists, 
the researcher extended this idea using robots to encourage the children to engage the robot proactively. 
The robot has built at USC, which has a large button on its back, and it was programmed to 
encourage social interaction with children. For example, the robot nods its head and makes a 
sound to encourage the kid to approach it; when the kid walks away, it moves its head down and 
make a sad kind of sound to imply the child and ask him/her to come closer to the robot. If the child 
presses that button on the robot, it blows bubbles and turn. In this study, one hundred minutes 
experiments have been recorded; three different conditions have been considered, which are the time 
kids spent near 1) the wall, 2) the parent, and 3) behind the robot. Kids have been separated into two 
groups: ‘Group A’ (children like the robot) and ‘Group B’ (children do not like the robot), a total 
number of eight children with ASD. The result shows that the Group A spent more than 60\% of the 
time playing with the robot, and Group B spent more than 50\% of the time showing the negative 
reaction (i.e., go away from the robot, play with himself) from avoiding the robot. This study might not 
be compelling because it is free to play with the robot; the experimental settings haven’t 
kept the same, and the limited numbers of participants. Also, without a control group like typically 
developing, they could not compare the differences between ASD and TD children, within the robot games. 
However, it shows the capability of encouraging children to communicate with a robot and lead the 
conversation \cite{feil2008robot}.\\

\subsubsection{Turn-Taking Activities}
At the University of Hertfordshire and the University of California, researchers have built small 
mobile robots that focused on helping children with ASD in turn-taking behaviors \cite{feil2008robot, dautenhahn2004towards}. It 
is easy to found out that children with ASD have a hard time allowing their conversation partner 
to participate. The researchers try to use these robots to help them become accustomed to waiting 
for responses after they say or do something. Labo-1 built by the University of Hertfordshire, 
which can play a game called tag with children. This game forces them to alternate between 
engaging and avoiding the robot \cite{dautenhahn2004towards}.\\

Labo-1 is a mobile platform that has an AI system resembled in a sturdy flat-topped buggy. 
Children have been allowed to freely play with Labo-1 as a teacher was deciding about how 
to switch between different games/sessions considering children appear (i.e., different reactions 
of children like tired or less interested in robot). From their initial trials, children were 
overall happy to play with robots. At the beginning of the game, the robot showed several simple 
behavior patterns, such as going forward and backward. Kids showed a positive response to these 
behaviors and enjoyed to keep playing with Labo-1. Children were also enjoyed interacting with the 
robot while it used a feature called ‘heat following behavior’; they moved away from the robot and 
saw if the robot can follow or not. There were five trials in total, three of them lasted around four 
minutes, and the remaining two had a duration of approximately fourteen minutes. Researchers realized 
that the issues that may cause this difference might be related to the levels of the children’s functioning. 
Since children are not in complete control the robot’s actions, and children’s response was totally 
different, some of them either walked or crawled around the room, some of them just simply lay on 
the floor to interact with robot only use arm movement \cite{dautenhahn2004towards}. During the interactions, it is obvious 
to notice that robots need more advanced behaviors to be developed, and the scenario should have more 
control for data analysis and get more convincing results. Also, the functioning level become another 
important element that needs to be considered.\\

\subsubsection{Expression/Emotion Recognition and Imitation}
Another critical difficulty of individuals with ASD is to recognize the expressions and emotions, 
besides appropriately imitating them. Studies show that kids with ASD have a hard time recognizing 
emotions and facial expressions. It would be difficult for them to deliver their emotions through 
their faces' actions. Researchers pointed out that to kids with ASD, such emotion type information 
that contained faces or eye contact can result in overwhelming or sensory overload. For example, a 
person could smile twice, and the child with ASD might pick two entirely different expressions from 
those two smiles. The robot can provide more constancy repeatable, consistent behaviors than a human does, 
and it would be a better way to teach children expressions and emotions.\\

KASPAR, a child-sized doll-like robot which has a silicon-rubber face on it, developed by the 
The University of Hertfordshire has been used to show bodily expressions by move head and arms. KASPAR 
was operated via wireless remote. Sessions are designed to allow the children to have free play 
interaction with the robot. Some behaviors had been pre-programmed in the robot, those behaviors allows 
KASPAR show several facial expressions, hand waving, and drumming on the tambourine on its legs 
to express different emotions. During the interaction, three types of touch using the hands had been 
identified: grasping (different tension levels), stroking, and poking. The forces of touching can be 
detected by the tactile sensors equipped with various places of KASPAR’s arms, hands, face, and shoulders. 
By identifying different levels of touching, KASPAR would provide different movements or expressions 
to tell the children the emotions or feelings of it. Emotion and facial expressions recognition could 
be taught via these outputs KASPAR given. The limitation of this study is very few numbers of children 
(five children in total) had participated in this study. Besides, limited facial expressions (happiness, 
displeasure, surprise, etc.) have employed in the robot system, and those expressions are hard to 
distinguish by the images they provided. There is no verbal communication between kids and robots, 
which is another weakness of this study \cite{robins2006does}. FACE is a robot designed at the University of Pisa 
point to closely approximate a real human look and show detailed facial expressions. Children would 
be asked to imitate those expressions to practice their ability in facial expression recognition and 
imitation. Specific scenarios (i.e., 1) facial expression association: a) facial matching, b) emotion 
labeling; 2) emotion contextualization) would be given to kids and ask them to pick up an appropriate 
emotional expression for FACE to make. Several experiments have been implemented to help the children 
to generalize the information they learn from the therapy sessions. After practicing with FACE, the 
children were tested using the Childhood Autism Rating Scale, and the results showed that while working 
with FACE, the ability of categories emotions and expressions for all kids (total number of 4 kids) have 
been improved. Also, researchers found out that those children can imitate facial expressions from FACE 
better than from humans, and it easier for a therapist because of the automate repeatable of 
the robots process. However, still, a minimal number of kids participated in the study that made the 
results somehow not wholly untenable \cite{pioggia2005android}.\\

\subsection{Using NAO in Autism}
NAO is a multifunctional humanoid robot that was developed by Aldebaran Robotics and as it has 
capabilities such as making the different gesture, moving separate arm and leg movement and hear 
orientations, It has been used for different human-robot interaction sessions. In this section 
we will talk about the existing interactions sessions that were conducted by NAO and later in the 
next chapter we will explain about our therapy sessions and designed game based on NAO for children 
with ASD.\\

In the University of Teknologi MARA, NAO was used to conduct seven interactions modules for interacting 
kids with autism. Each module lasts four minutes, and one minute break was provided between two sessions. 
Different interaction tasks have been contained in those modules (i.e., static interaction, joint 
attention, necessary language skills). The frequency of child looking at the robot and the duration of each occurrence 
of communication has been reported. After all, they concluded that those seven modules could be applied to 
develop human-robot integration therapy sessions for children with autism \cite{shamsuddin2012initial1}. The same year, these 
researchers use 5 of those seven modules did a case study, with the same setting, they recruited one 
high-functioning (with IQ 107) to complete those five tasks. They aimed to discover whether that child 
can provide a better exposure behavior with a robot compared with the activity in the class. After 
running the five tasks for only one instance, they concluded that the child behavior had been 
improved significantly with the robot than in the class, they also suggested that humanoid robot NAO 
can be used as a significant platform to support and initiate interaction with children with ASD \cite{shamsuddin2012initial2}. 
After this case study, they recruited five other children with ASD (low IQ, average around 50) and did 
the same experimental interaction sessions with them. Out of five children showed better performance 
during robot interaction compared with daily in-class performance \cite{shamsuddin2012humanoid}. Further research has been done 
by this group, and they added the emotion recognition module into the interaction sessions. Five body gesture 
emotions (hungry, happy, mad, scared, and hug/love) have been implemented in the program. Two boys 
have been enrolled in this study. After finished the session, researchers pointed out that NAO 
has the inherent capability to teach head and bod posture related to social emotions for children 
with autism without provided any statistical analysis only based on observations \cite{shamsuddin2013humanoid}. This group 
has been initiated working with NAO for autism therapeutic sessions and implementing and compared 
different scenarios based on NAO. Reviewing the existing papers demonstrate that the number of 
participants and interaction sessions for these studies is very limited. They have used only one 
session for each subject. Therefore they could not analyze the social responses of individuals with 
ASD statistically.\\

In our study we employ NAO since it has several functionalities that are embedded in it (e.g. 
text-to-speech, tactical sensor, face recognition, voice recognition, etc.). This would help us to 
build a social-communicative task for human-robot interaction. Based on the size of the robot and 
the friendly appearance of the robot we design, conduct and analyze the gaze related responses of 
ASD individuals and compare it with the TD control group. The details of our experiment and the results 
will be discussed in Chapter 4.\\

\section{Music Therapy in robot}
Socially assistive robots are widely used in the young age of autism population interventions these years. Some studies are
focusing on eye contact and joint attention \cite{feng2013can, mihalache2020perceiving, mavadati2014comparing}, 
showing that at some point, the pattern of ASD group in perceiving eye gaze is similar to typically 
developed (TD) kid, and eye contact skills can be significantly improved after intervention sessions. Plus,
these findings also provide strong evidence of ASD kids are easy to attract to humanoid robots in
various types of social activities. Some groups start to use such robots to conduct music-based therapy
sessions nowadays. Children with autism are asked to imitate play music based on Wizard of Oz style
and Applied Behavior Analysis (ABA) models from humanoid robots in intervention sessions for practicing
eye-gaze and joint attention skills \cite{peng2014using, taheri2015impact, taheri2016social}. However, 
some disadvantages of such research due to lack of sample size and no automated system in human-robot 
interaction. Music can be used as a unique window into the world of autism, lots of evidence suggest that
many individuals with ASD are able to understand simple and complex emotions in childhood using music-based
therapy sessions \cite{molnar2012music}. Although limited research has found in such areas, especially using
bio-signals for emotion recognition for ASD and TD kids \cite{feng2018wavelet} in understanding the 
relationship between activities and emotion changes. \\

To this end, in current research, an automated music-based social robot platform with an activity-based emotion 
recognition system is presented in the following sections. The purpose of this platform is to provide a 
possible ultimate solution for assisting children with autism to improve motor skills, turn-taking skills 
, and activity engagement initiation. Furthermore, by using bio-signals with Complex-Morlet (C-Morlet) wavelet feature 
extraction \cite{feng2018wavelet}, emotion classification, and emotion fluctuation are analyzed based on different 
activities. TD kids have participated as a control group to see the difference from ASD group.\\

\section{Summary}
The current chapter reviewed the research-related work in Autism Spectrum Disorder, Socially Assistive Robotics,
and the interdisciplinary for both topics. Went through the history of researches, the author described details in
the problems of autism, including motor control, turn-taking behavior, eye-gaze, and joint attention. Some of
the treatments have also been discussed in this chapter, such as music therapy. As time moving forward, robotic 
solutions starting to become popular nowadays. Several types of research have been listed in this chapter, and all these studies
focused on different aspects of autism spectrum disorder. In the end, NAO has been briefly introduced and will 
have more details in further chapters.
