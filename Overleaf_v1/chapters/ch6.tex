\chapter{X-Elophone, A New Instrument}
After the first phase of the experiments done, some limitations for the current design were noticed.
A limitation of the music representation can be found out of the acoustic instrument, especially
when certain songs with rich music contents are played. Some of the participants requested rock or electric 
songs for practices, thus an acoustic xylophone barley meet the requirements due to the 
sound quality. In this chapter, the design and prototyping of a novel instrument based on Xylophone are described.
The purpose of this design is to add more possibilities for playing different timber and major/minor keys. 
The increased amounts of possible notes would allow the system to play more customized songs for kids.\\

\section{Xylophone Modification}

\subsection{Components Selection}

\subparagraph{A. Piezo Vibration Sensor: }
The LDT0-028K is a flexible component comprising a 28 $\mu$m thick piezoelectric PVDF
polymer film with screen-printed Ag-ink electrodes laminated to a 0.125 mm polyester 
substrate and fitted with two crimped contacts. As the Piezo film is uprooted from 
the mechanical unbiased pivot, bowing makes high strain inside the Piezo polymer, 
in this manner high voltages are created. At the point when the get together is avoided by direct 
contact, the gadget goes about as an adaptable "switch", and the created yield is adequate 
to trigger the MOSFET or CMOS arranges legitimately. If the assembly is supported by its contacts 
and left to vibrate "in a free space" (with the inertia of the clamped/free beam creating 
bending stress), the device will behave as an accelerometer or vibration sensor. Increasing 
the mass or adjusting the free length of the component by clipping can change the thunderous 
recurrence and affectability of the sensor to suit explicit applications. Multi-axis reaction 
can be accomplished by situating the mass askew. The LDTM-028K is a vibration sensor, 
where the detecting component contains a cantilever bar stacked by an extra mass to 
offer high affectability at low frequencies. Figure \ref{sensor_s} shows the schematic of 
a Piezo vibration sensor and Figure \ref{sensor} shows how it looks like.\\

\begin{figure}[tbp]
	\begin{center}
		\begin{tabular}{c}
			\epsfig{figure=./chapters/fig/peizoSensorPic.eps, scale = 0.3}\label{sensor_s} \\
		\end{tabular}
		\caption{Piezo Sensor Schematic} \label{sensor_s}
	\end{center}
\end{figure}

\begin{figure}[tbp]
	\begin{center}
		\begin{tabular}{c}
			\epsfig{figure=./chapters/fig/sensor.eps, scale = 0.8}\label{sensor} \\
		\end{tabular}
		\caption{Piezo Sensor VS A Quarter} \label{sensor}
	\end{center}
\end{figure}

\subparagraph{B. Op-Amp: }
An operational amplifier (often called op-amp or opamp) is a DC-coupled high-gain electronic 
voltage amplifier with a differential input and, usually, a single-ended output. In 
this configuration, an op-amp produces an output potential (relative to circuit ground) 
that is typically hundreds or thousands of times larger than the potential difference 
between its input terminals. Operational amplifiers had their origins in analog computers, 
where they were used to perform mathematical operations in many linear, non-linear, and 
frequency-dependent circuits.\\

The popularity of the op-amp as a building block in analog circuits is due to its 
versatility. By using negative feedback, the characteristics of an op-amp circuit, its 
gain, input and output impedance, bandwidth, etc, are determined by external components 
and have little dependency on temperature coefficients or engineering tolerance in the 
op-amp itself.\\

Op-amps are among the most widely used electronic devices today, being used in a vast 
array of consumer, industrial, and scientific devices. Many standard IC op-amps cost 
only a few cents in moderate production volume; however, some integrated or hybrid 
operational amplifiers with special performance specifications may cost over US 100 in 
small quantities. Op-amps may be packaged as components or used as elements of more 
complex integrated circuits. Figure \ref{opAmp} shows the schematic of MCP6002 IC.\\

\begin{figure}[tbp]
	\begin{center}
		\begin{tabular}{c}
			\epsfig{figure=./chapters/fig/opAmpPic.eps, scale = 0.4}\label{opAmp} \\
		\end{tabular}
		\caption{Schematic of Op-Amp MCP 6002} \label{opAmp}
	\end{center}
\end{figure}

The op-amp is one type of differential amplifier. Other types of differential amplifiers 
include the fully differential amplifier (similar to the op-amp, but with two outputs), 
the instrumentation amplifier (usually built from three op-amps), the isolation amplifier 
(similar to the instrumentation amplifier, but with tolerance to common-mode voltages 
that would destroy an ordinary op-amp), and negative-feedback amplifier (usually built 
from one or more op-amps and a resistive feedback network). Figure \ref{sensor_op}\\

\begin{figure}[tbp]
	\begin{center}
		\begin{tabular}{c}
			\epsfig{figure=./chapters/fig/sensorUseagePic.eps, scale = 0.6}\label{sensor_op} \\
		\end{tabular}
		\caption{Schematic of Piezo Sensor Application as A Switch} \label{sensor_op}
	\end{center}
\end{figure}


\subparagraph{C. Multiplexer: }
In electronics, a multiplexer (or mux) is a device that selects between several analog 
or digital input signals and forwards it to a single output line. A multiplexer of 
${\displaystyle 2^{n}} 2^{n}$ inputs has ${\displaystyle n}$ n select lines, which are 
used to select which input line to send to the output. Multiplexers are mainly used 
to increase the amount of data that can be sent over the network within a certain amount 
of time and bandwidth. A multiplexer is also called a data selector. Multiplexers can 
also be used to implement Boolean functions of multiple variables.\\

An electronic multiplexer makes it possible for several signals to share one device or 
resource, for example, one A/D converter or one communication line, instead of having one 
device per input signal.\\

Conversely, a demultiplexer (or demux) is a device taking a single input and selecting 
signals of the output of the compatible mux, which is connected to the single input, and 
a shared selection line. A multiplexer is often used with a complementary demultiplexer 
on the receiving end.\\

An electronic multiplexer can be considered as a multiple-input, single-output switch, 
and a demultiplexer as a single-input, multiple-output switch. The schematic symbol 
for a multiplexer is an isosceles trapezoid with the longer parallel side containing the 
input pins and the short parallel side containing the output pin. 
The ${\displaystyle sel}$ sel wire connects the desired input to the output.
The 74HC4051; 74HCT4051 is a single-pole octal-throw analog switch (SP8T) suitable for 
use in analog or digital 8:1 multiplexer/demultiplexer applications. The switch features 
three digital select inputs (S0, S1 and S2), eight independent inputs/outputs (Yn), a 
common input/output (Z) and a digital enable input (E). When E is HIGH, the switches are 
turned off. Inputs include clamp diodes. This enables the use of current limiting resistors 
to interface inputs to voltages in excess of VCC. Figure \ref{mux} shows the multiplexer
used in this study.\\

\begin{figure}[tbp]
	\begin{center}
		\begin{tabular}{c}
			\epsfig{figure=./chapters/fig/mux.eps, scale = 0.4}\label{mux} \\
		\end{tabular}
		\caption{SparkFun Multiplexer Breakout 8 Channel (74HC4051)
		} \label{mux}
	\end{center}
\end{figure}

\subparagraph{D. Arduino UNO:}
The Arduino Uno see Figure \ref{arduino} is an open-source micro-controller board based on the Microchip ATmega328P 
micro-controller and developed by Arduino.cc. The board is equipped with sets of digital 
and analog input/output (I/O) pins that may be interfaced to various expansion boards (shields) 
and other circuits. The board has 14 Digital pins, 6 Analog pins, and is programmable with 
the Arduino IDE (Integrated Development Environment) via a type B USB cable. It can be 
powered by the USB cable or by an external 9-volt battery, though it accepts voltages between 
7 and 20 volts. It is also similar to the Arduino Nano and Leonardo. The hardware 
reference design is distributed under a Creative Commons Attribution Share-Alike 2.5 license 
and is available on the Arduino website. Layout and production files for some versions of 
the hardware are also available.\\

\begin{figure}[tbp]
	\begin{center}
		\begin{tabular}{c}
			\epsfig{figure=./chapters/fig/arduino.eps, scale = 0.4}\label{arduino} \\
		\end{tabular}
		\caption{Arduino Uno Microprocessor
		} \label{arduino}
	\end{center}
\end{figure}

The word "uno" means "one" in Italian and was chosen to mark the initial release of the 
Arduino Software. The Uno board is the first in a series of USB-based Arduino boards, 
and it and version 1.0 of the Arduino IDE were the reference versions of Arduino, now evolved 
to newer releases. The ATmega328 on the board comes pre-programmed with a bootloader 
that allows uploading new code to it without the use of an external hardware programmer.\\

While the Uno communicates using the original STK500 protocol, it differs from all 
preceding boards in that it does not use the FTDI USB-to-serial driver chip. Instead, 
it uses the Atmega16U2 (Atmega8U2 up to version R2) programmed as a USB-to-serial converter.\\

\subsection{ChucK: An On-the-fly Audio Programming Language Based on C++}
The computer has long been considered an extremely attractive tool for creating,
manipulating, and analyzing sound. Its precision, possibilities for new timbres, and
potential for fantastical automation make it a compelling platform for expression
and experimentation, but only to the extent that we are able to express to the
computer what to do, and how to do it. To this end, the programming language
has perhaps served as the most general, and yet most precise and intimate interface
between humans and computers. Furthermore, “domain-specific” languages can
bring additional expressiveness, conciseness, and perhaps even different ways of
thinking to their users \cite{wang2003chuck}.\\

\section{Hardware and Software Design}
In order make X-Elophone produce different sound than normal xylophone. Two major problem need
to be done: 1) circuit design, in order to collect vibration from the xylophone and convert
analog signal to digital signal, and 2) software control, in order to transfer digital signals
into melody and playback. These design will be discussed in following sections. \\

\subsection{Circuit Design}
Piezo sensors are attached at the back of each metal bar of the xylophone in order to pick up
the vibration. Voltage generated from the Piezo sensor will be compared with the voltage across 
a potentiometer using the MCP6002 Op-Amp in order to filter out the noises from the signal such
as slight move of the instrument. All potentiometers were carefully adjusted individually using
oscilloscope to make sure strikes or touches of the note bars could create clear and perfect
peaks. One Piezo sensor and one potentiometer are connect in parallel, then series with one Op-Amp.
This setup is considered as one line for single note bar which contains 11 lines for the whole 
system. Output wire from each Op-Amps are connect to multiple input channels (labeled Y0 to Y7) of the multiplexers. 
11 lines are connected with two multiplexes in parallel. C6 to B6 are connected in mux-1 correspond 
to channel Y0 to Y6,and the rest notes are in mux-2 from channel Y0 to Y3. Common output Z1 and Z2 
are connect to the analog inputs A0 and A1 of UNO board. Six digital select inputs are connect with
the digital ports from UNO for analyzing the analog inputs from the instrument. Figure \ref{schematic} shows 
the schematic of circuit, Figure \ref{real} shows the actuarial circuit on bread board, and Figure \ref{oscil}
shows a signal representation from the oscilloscope of multiple hits through the newly designed xylophone.
Yellow signal represents the filtered voltage change of seven strikes from the xylophone, and the green
pulse signal represents the output digital signals converted by the mux. Note that the input voltage level were
around 2.5v and the output signals are amplified to 5.0v.\\

\begin{figure}[tbp]
	\begin{center}
		\begin{tabular}{c}
			\epsfig{figure=./chapters/fig/schematic.eps, scale = 0.4}\label{schematic}\\
		\end{tabular}
		\caption{The schematic design for the circuit.} \label{schematic}
	\end{center}
\end{figure}

\begin{figure}[tbp]
	\begin{center}
		\begin{tabular}{c}
			\epsfig{figure=./chapters/fig/real.eps, scale = 0.4}\label{real}\\
		\end{tabular}
		\caption{Circuit board in real size.} \label{real}
	\end{center}
\end{figure}

\begin{figure}[tbp]
	\begin{center}
		\begin{tabular}{c}
			\epsfig{figure=./chapters/fig/oscil.eps, scale = 0.28}\label{oscil}\\
		\end{tabular}
		\caption{A sample input and output: green channel comes from the output of the mux, yellow channel comes from the output of the Op-Amp.} \label{oscil}
	\end{center}
\end{figure}

\subsection{Software Design}
As mentioned above, well designed circuit has been tested and implemented. However, to make xylophone
sound different, software control and design also plays important role. Figure \ref{pin} to Figure \ref{input}
shows the code detail from Ardurino in collecting sensor signal and filter out small vibration noise from 
accident gestures. 

\begin{figure}[tbp]
	\begin{center}
		\begin{tabular}{c}
			\epsfig{figure=./chapters/fig/pin.eps, scale = 2}\label{pin} \\
		\end{tabular}
		\caption{Arduino code for pin assignment of mutiplexers.
		} \label{pin}
	\end{center}
\end{figure}

\begin{figure}[tbp]
	\begin{center}
		\begin{tabular}{c}
			\epsfig{figure=./chapters/fig/11pin.eps, scale = 2}\label{11pin} \\
		\end{tabular}
		\caption{Loop in checking all pins for getting signals from the instrument.
		} \label{11pin}
	\end{center}
\end{figure}

\begin{figure}[tbp]
	\begin{center}
		\begin{tabular}{c}
			\epsfig{figure=./chapters/fig/muxpinselect.eps, scale = 2}\label{pinselect} \\
		\end{tabular}
		\caption{Selecting proper mux by decoding the signals.
		} \label{pinselect}
	\end{center}
\end{figure}

\begin{figure}[tbp]
	\begin{center}
		\begin{tabular}{c}
			\epsfig{figure=./chapters/fig/inputcontrol.eps, scale = 2}\label{input} \\
		\end{tabular}
		\caption{Control the input level in order to filter out small viberation noise to make sure getting meaningful input.
		} \label{input}
	\end{center}
\end{figure}

One of the most important reason select using ChucK as the music design tool is 
due to its real-time sound synthesis and music creation. 
Figure \ref{chuckflow} shows the flow chart for music software design. This design
allows user to switch keys between different scales and Major/minor in order to
create emotional music. From Figure \ref{serial} to Figure \ref{in1} shows the Chuck 
code in detail.\\

\begin{figure}[tbp]
	\begin{center}
		\begin{tabular}{c}
			\epsfig{figure=./chapters/fig/chuckflow.eps, scale = 0.5}\label{chuckflow} \\
		\end{tabular}
		\caption{A Flow Chart of Using ChucK in Designing Sound Control System to X-Elophone 
		} \label{chuckflow}
	\end{center}
\end{figure}

\begin{figure}[tbp]
	\begin{center}
		\begin{tabular}{c}
			\epsfig{figure=./chapters/fig/serial.eps, scale = 2}\label{serial} \\
		\end{tabular}
		\caption{ChucK Code Part I: Allows ChucK program to accept text information as strings via a serial interface specifically for Arduino
			board at port 9600. Each time Arduino sends a new string, events will be triggered, allowing following code get the string and process.
		} \label{serial}
	\end{center}
\end{figure}

\begin{figure}[tbp]
	\begin{center}
		\begin{tabular}{c}
			\epsfig{figure=./chapters/fig/sound.eps, scale = 2}\label{sound} \\
		\end{tabular}
		\caption{ChucK Code Part II: Create 2 STK instrument Mandolin and BeeThree with 2 sets of scales. Different sound effect also been 
			assigned to instrument.
		} \label{sound}
	\end{center}
\end{figure}

\begin{figure}[tbp]
	\begin{center}
		\begin{tabular}{c}
			\epsfig{figure=./chapters/fig/instrument.eps, scale = 2}\label{insel} \\
		\end{tabular}
		\caption{ChucK Code Part III: Using 1 and 0 to change instrument sound for broadcasting.
		} \label{insel}
	\end{center}
\end{figure}

\begin{figure}[tbp]
	\begin{center}
		\begin{tabular}{c}
			\epsfig{figure=./chapters/fig/in2.eps, scale = 2}\label{in1} \\
		\end{tabular}
		\caption{ChucK Code Part IV: Once in instrument 1, when the 11th note gets hit, 'change' value starts to switch the sound
			to the other instrument. Each note will be played between .25 to .5 second. All play information will be displayed on
			computer.
		} \label{in1}
	\end{center}
\end{figure}