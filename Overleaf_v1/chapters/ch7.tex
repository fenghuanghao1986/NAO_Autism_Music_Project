\chapter{Discussion, Conclusion and Future Work}
\section{Each Participant Has A Story}
In this section, a brief general report from each participant will be given. This would the readers better understand how each child behaved in our study sessions and how different one child is from another when it comes to social behaviors and interaction with our robot and music therapy method.\\

\subsection{Subject 101}
As the first ASD participant who joined our study, subject 101 provided excellent support for the future 
intervention sessions for him and other participants. There is one kid song fascinated him greatly, which is called "Baby Shark" taken off since 2016 from Pinkfong, a South Korean education company. 
Due to the popularity of this music all over the world, it made it a perfect song for children in 
such music intervention sessions. During the session, subject 101 showed special love in "Baby Shark".
Every time the robot asks what he wanted to play, that song
was chosen by him. In the game session, when the free play part is one, a "Baby Shark"
song was played all the time. To make this more challenging for the subject, three different versions of
"Baby Shark" were pre-programmed in the system. Thanks to the simplicity of this song, which allows 
three different keys to be rearranged to it with the availability of the bars on the current xylophone. 
With the familiarity of this song, subject 101 provided a stable constant level of performance even
with the three keys of that song. \\

Subject 101 is a music lover and showed a strong passion for playing the instrument. Based on the recorded 
videos, motor control has been appropriately taught during the first few sessions and nice and clean notes were 
played by him. However, sometimes subject 101 may hit the bar a bit too hard, which 
potentially caused the damage to the instrument or the base stand. And there was one time that one of 
the base handles broke. Fortunately, this issue did not affect the rest of the sessions. 
Such a high level of engagement supports the confidant of this platform in use of the practice in the future.\\

\begin{figure}[tbp]
	\begin{center}
		\begin{tabular}{c}
			\epsfig{figure=./chapters/fig/101.eps, scale = 1.5}\label{101} \\
		\end{tabular}
		\caption{Subject 101} \label{101}
	\end{center}
\end{figure}

\subsection{Subject 102}
The most impression from subject 102 is the significant improvement we observed in all the sessions, including all 
aspects of this study. Playing instruments seemed difficult for him at the beginning. At first this participant had a hard time striking the xylophone accurately. Hitting gesture was somehow challenging to him, and the muffled sound was continuously played even with
the warm-up activity. Breaks between activities were often requested across all the sessions, especially 
during the intervention session with repetitive works. Most of the time he was counting the 
trials during each activity to leave the experiment room and take a break. However, he never
quit any of the sessions or activities, and always completed all the activities as needed. \\

This participant was willing to learn and improve his skills gradually based on the attitude during
the intervention sessions. He started showing no care to the performance at the beginning sessions; to
the end, he wanted to play better and appreciated the encouragement from the robot and the researcher.
However, it still did not change the fact that he enjoyed taking breaks after each activity and counting
the numbers of each trial. The song he used was "Twinkle Twinkle Little Star" due to the fact that there
was no favorite song from this subject. \\

\begin{figure}[tbp]
	\begin{center}
		\begin{tabular}{c}
			\epsfig{figure=./chapters/fig/102.eps, scale = 1.5}\label{102} \\
		\end{tabular}
		\caption{Subject 102} \label{102}
	\end{center}
\end{figure}

\subsection{Subject 103}
Subject 103 is one of the participants who did not learn much out of this study, which is unfortunate. 
After all the sessions, he could not learn the hitting technique properly, which means his motor skills
were not successfully improved. According to the videos, subject 103 showed acceptable turn-taking
behavior across all the sessions. However, he needed some help in concentrating on the tasks most of the 
time. There was one time that his father had to jump in and interrupt his behaviors while he was not willing
to play xylophone but walking and talking of other stuff. \\

This kid also had some speech issue, which makes the robot had significant trouble with some of the
responses from him. For example, the robot could not recognize him saying "yes" and always provided a default
behavior. Usually, a "yes" response was requested when participants needed help robot's help. Because 
the robot could not understand the child's response, it did not deliver the proper content.
Hence, whenever the robot asks him whether help is needed, he always responds to "I don't know."
To this end, the robot did not have any chance to teach the motor control skills to subject 103. \\

\begin{figure}[tbp]
	\begin{center}
		\begin{tabular}{c}
			\epsfig{figure=./chapters/fig/103.eps, scale = 1.5}\label{103} \\
		\end{tabular}
		\caption{Subject 103} \label{103}
	\end{center}
\end{figure}

\subsection{Subject 104}
70\% of accuracy for playing a rock song called "I Feel Fantastic" makes his entire session perfect.\\

\begin{figure}[tbp]
	\begin{center}
		\begin{tabular}{c}
			\epsfig{figure=./chapters/fig/104.eps, scale = 1.5}\label{104} \\
		\end{tabular}
		\caption{Subject 104} \label{104}
	\end{center}
\end{figure}

\subsection{Subject 105}
As mentioned before, there was one activity in this system to ask the feelings of participants regarding
randomly generated music by the robot: "How do you feel about what you just heard?". Subject 105 was the only one who shared/expressed his feelings with the robot. 
This exciting finding showed the potential for children in understanding music emotions.
According to the responses from the participants in this study, most of them were thinking about the difficulty in how
to playback regarding that random melody, but not how they feel about this piece of music. It is hard to conclude the reason for
this result. One possible reason could be the age difference. Older participants may have a better understanding
of the questions from the robot as well as the music emotion. \\

Subject 105 adapted to this music teaching system and performed with a very high accuracy in all the activities. It is also worth mentioning that at the exit session, there was a hidden challenge
for all the participants, which added the harmonics to the song they practiced for the past sessions. Subjects
would be asked to try this challenge if they wanted. To this end, subject 105 did the best on this task.
He almost aced this challenge in playing his requested song, which no other participants could complete this.\\

\begin{figure}[tbp]
	\begin{center}
		\begin{tabular}{c}
			\epsfig{figure=./chapters/fig/105.eps, scale = 1.5}\label{105} \\
		\end{tabular}
		\caption{Subject 105} \label{105}
	\end{center}
\end{figure}

\subsection{Subject 106}
As the only ASD girl and the only basketball player in this study, one word can be describer her: 
COMPETITIVE. She was also claimed that she has some experiences in playing the saxophone. Although the instrument 
was different from a xylophone, the musical experience could have provided some useful knowledge in 
understanding music concepts such as keys, scales, and melody constructions. Surprisingly, this 
girl had a difficult time in striking xylophone correctly for the first baseline session, and gradually
picked up the technique from the warm-up activity at her first intervention session. Once she got approved
by the robot, she began to connect with this research. According to the annotator, subject 106 showed
strong engagement in the sessions, focusing on decoding the message one after another. This could also 
be reflected from the statics result: on average of 80\% of the accuracy in the main song practice activity.
In this system, a color hint was always be given for all trials; however, the same color sometimes means different
pitch on the xylophone. This girl has her way of conquering this challenge. She will play both notes and
compare it with the note played by the robot speakers, which makes a perfect play by her.
By choosing the song "Three Little Birds" composed by Bob Marley, this performance is impressive. 
Because she wanted to be better. This could suggest that sports may affect human's mindset and change their
behaviors in other activities in daily life.\\

The most joyful moment for most of the participants would be the free playtime, in this activity it allows
kids to challenge NAO to play whatever they just played for 5 seconds each time and with no limitation of
melody structure. Particularly with subject 106, she spent most of the time in this section every single session.
Also worth to be mentioned is, after each time the robot playback to the participants, NAO will ask for a 
grade from them. Most of the participants did not take this seriously and sometimes will provide a ridiculously
high or low score regardless of the actual performance of the robot. Other than these subjects, participant
106 would carefully rate the robot's performance base on what she felt like, and most of the time is pretty
reasonable, according to the researcher in the room. This activity usually took over 15 minutes every time she 
visits NAO and, most of the time, have to end the session manually from the computer side. \\

\begin{figure}[tbp]
	\begin{center}
		\begin{tabular}{c}
			\epsfig{figure=./chapters/fig/106.eps, scale = 1.5}\label{106} \\
		\end{tabular}
		\caption{Subject 106} \label{106}
	\end{center}
\end{figure}

\subsection{Subject 107}
At the beginning of the session, this subject needed significant help from his caregiver for the entire
time. It is also apparent to notice that he may request extra help in future visits. According to his mom, 
music therapy treatment had been given to subject 107 before, and it is also clear that he did enjoy
this method of treatment. The baseline session was not good; this boy did not listen to the robot and 
could not provide a meaningful turn-taking behavior then. Since he had music therapy before, it was unclear
whether the therapist used xylophone before or how frequently it has been used. Regardless of this uncleared
question, subject 107 did a good job of playing the xylophone, and a nice and clear note could be played by him
for most of the time. \\

Subject 107 showed signs in constant repetitive hand gestures during sessions, which somehow made it 
difficult to hold the mallet properly in the first few sessions. This fact created some delay in responding
to the robot during the intervention session. From Figure \ref{song}, it is easy to see that this subject
could not follow the turn-taking rule properly; most of the results were not recorded in the right way.
However, according to the report of the annotators, "...this participant was able to understand the color
hint from the robot and may provide correct input to the robot but off the time limit, which will not 
be recorded in the system and could be determined as incorrect answers from the computer side...".\\

As session ongoing, this particular participant started to show more engagement behavior for the last 
intervention session. The first verbal response to the question from the robot happened in session 5.
After all the intervention, finally, subject 107 started to respond to the music agent for the very
first time, and multiple times in this session as well positively. One significant difference reflected
in the music playing activity. The participant started to repeat the color names while trying to strike the 
bars accordingly. This phenomenon may suggest that for some particular users, they may need more time 
to adjust themselves in getting used to the new system or platform. Besides, proper turn-taking 
behavior can be improved by using this platform after intervention sessions.\\

\begin{figure}[tbp]
	\begin{center}
		\begin{tabular}{c}
			\epsfig{figure=./chapters/fig/107.eps, scale = 1.5}\label{107} \\
		\end{tabular}
		\caption{Subject 107} \label{107}
	\end{center}
\end{figure}

\subsection{Subject 108}
One talented musician among all subjects. Subject 108, a violin player, who suppose to have the most 
precise ear playing ability, surprisingly did not perform very well from the static result. 68.75\%
is the accuracy for the customize song play across all intervention sessions. After dig into his 
files, the mystery starts to appear. The biggest challenge is the song he chooses: "Can Can" by Offenbach.
Such a famous and difficult classical music in string instrument. And he accepted this challenge willingly.
From the performance result, subject 108 shows decreasing in the play accuracy rate after sessions, which
makes perfect sense in such a high-level challenge. Based on the setting of this practice, the notes will
significantly increase at the last two intervention sessions in order to have the whole song or main
melody covered to be taught to the kids. "Can Can" which has already included a massive amount of notes in
it, and this could suggest the lousy performance at the last intervention session of subject 108. However,
almost 70\% of accuracy in such a problematic song; he did an ace job. \\

Quiet, focus, pinpoint, can be the best definition of subject 108. This participant was able to complete
the tasks smoothly with a high level of engagement. At the very beginning of the session, he also confused
about the technique of how to play percussion properly since it is different from strings. 
However, it did not take long for him to figure out. Starting from the first intervention session, 100\%
of the accuracy from the warm-up task explains everything. \\

\begin{figure}[tbp]
	\begin{center}
		\begin{tabular}{c}
			\epsfig{figure=./chapters/fig/108.eps, scale = 1.5}\label{108} \\
		\end{tabular}
		\caption{Subject 108} \label{108}
	\end{center}
\end{figure}

\subsection{Subject 109}
Unstoppable can be suitable to describe this participant. According to his caregiver, subject 109 
has a hearing disability from one side of his ear. However, after all sessions, this fact did not seem
to be an influence on his performance in playing music. Ever since he started this study, he showed 
hyperactive in the experiment room with the robot. Among all kids, he was the one who touched the robot
most, and there was one time almost accidentally pushed the robot to lean backward. This makes the researcher
had to pay extra attention to protecting both participants and robots not to get injured. Impatience was
also an issue for subject 109; the researcher had to spend time to help this participant focusing on the
tasks most of the time. \\

Although the session did not go smoothly as usual, the music performance from subject 109 was acceptable.
Without much help, he quickly picked up the striking technique from the robot and be able to hit the notes
accurately. One thing that needs to be mentioned is his hearing ability, he is one of the few participants who
have the sensitivity in distinguishing different pitches with the same color bars by only listen to the sound.
Somehow even better then the ones who claimed has the musical experience, for example, subject 108. Subject 108
is gifted. And after few sessions, he began to accept this platform, and showed more extended concentration period
in the last few sessions, which provided the potential uses for this assistive music teaching platform
a chance in daily life.\\

\begin{figure}[tbp]
	\begin{center}
		\begin{tabular}{c}
			\epsfig{figure=./chapters/fig/109.eps, scale = 1.5}\label{109} \\
		\end{tabular}
		\caption{Subject 109} \label{109}
	\end{center}
\end{figure}

\section{Discussion and Conclusion}
The results of our study indicate that the conferred music education platform will be thought-about as a decent tool to facilitate 
improving fine motor control, turn-taking skills, and social activities engagement in children with autism. The automated music 
detection system created a self-adjusted surrounding for participants in early sessions. Most
of the ASD youngsters began to develop the strike movement whiting the initial 2 intervention sessions; some 
even mastered the motor ability throughout the very first warm-up event. Though the robot might
provide verbal directions and demonstrations by voice command input from participants whenever they need it. However, 
the majority of the participants did not request such a service while playing with NAO. This finding
suggests that fine motor control ability will be learned from specific well-designed activities by
the young ASD population. \\

The purpose of using a music teaching scenario as the main activity in the current research is to 
give the kids a chance for practicing fine and natural turn-taking behaviors during social interactions. By observing all 
experimental sessions, 6 out of 9 subjects could dominate proper turn-taking skill after one or two
sessions. Note that subject 107 had significant improvement in the last few sessions compared to the
baseline session. Subject 109 had trouble listening to the robot for most of the time.
However, with the researcher's help, this kid could perform better music activities with social interaction skills
for a short time period. For practicing turn-taking skills, a fun, motivating activity should be 
designed for children with autism. Music teaching could be a good example for accomplishing this
task by taking the advantage of customized songs selected by each individual.\\

Starting the latter half of the sessions, participants could start to recognize their favorite songs,
where over half of the participants were getting more into the activities, although the difficulty level for
playing proper notes was much higher. It was easy to notice that older kids who spent more time engaging with the activities 
during the song practice session comparing to younger kids, especially in half/whole song play sessions.
Several reasons can explain this situation: one is because the more complex the music was,
the more challenging it was and more concentration was needed by the participants. Thus, older individuals might be willingly to accept the challenges and enjoy the sense of accomplishment afterward based on their verbal feedback to the research at the end of  
each session. Having some knowledge of music could also be one of the reasons that caused this result, 
since older participants might have had more chance to learn music at school. High engagement level can be 
found in the music game play section among all sessions. The reason behind it is not only because of the 
music game was fun and more relax, but also provided an opportunity in switching the role from student to instructor
for participants. Some of the participants enjoyed this role play activity 
%The games played in each session provided the highest engagement level of all time, not only because of it was relaxing and fun play, but also it offered an opportunity to challenge the robot to mirror the free play from them, this exciting phenomenon can be considered as "revenge." 
especially for subject 106, who spent a significant amount of time in free play 
game mode, this was more notable. According to the session executioner and video annotators, this particular subject (106) showed 
a high level of engagement in all the activities including the free play. Based on the conversation 
and music performance with the robot, the subject showed a strong interest in challenging the robot 
in a friendly way. \\

Measuring emotion in children with autism is often difficult and challenging. Bio-signals provide a possible way of 
doing that. The vvent-based emotion classification method presented in this research suggests
that the same activity with different intensities can cause emotion changes in the arousal dimension,
although it is difficult to label the emotions based on facial expression changes in the video
annotation phase for the ASD group. Less emotion expressed in particular activities presented in Table \ref{tab2} suggests that a mild, 
friendly game-like teaching system may motivate better social content learning for children with autism, 
even with repetitive movements. These well-designed activities could provide a relaxed learning 
environment that helped participants to focus on learning music with proper communication behaviors. 
This may explain the improvement of music play performance in the song practice (S2) through the intervention 
sessions illustrated in Figure \ref{song}. Comparing emotion patterns from baseline and exit sessions between TD and ASD groups in Table \ref{tab3}, Some differences could be found. This may suggest a potential way to assist autism diagnoses using bio-signals 
an early age. According to annotators and observers, TD kids showed a strong passion. Excitement, stressful, and disappointment were easy to be recognized and labeled from the recorded videos. On the other hand, limited facial expression changes could be detected in
the ASD group. That makes it challenging to learn whether they have different feelings or have
the same feelings but different bio-signal activities compared to the TD group. Further research need to be done to better understand these issues. Furthermore, due to the limitation of the sample size, future research can focus on this problem with different classification methods and with a larger sample size.\\

\section{Future Work: New Style Session Proposal}
Our newly designed electric xylophone has two significant differences compared to the 
original acoustic one. The most obvious improvement is the sound. Various timbers, keys, and 
scales can be programmed on the board and switched in real-time, which provides the 
infinite possibilities with limited note bars. Gentle touch has also been embedded in the play
style. Previously, one with only proper motor control could strike a sweet melody. However, by using this 
new design, one soft touch of the fine-tuned bar will also provide a friendly and clean note out
of the speaker. These two designs provide unlimited possibilities of music play based on the proposed music teaching platform.\\
