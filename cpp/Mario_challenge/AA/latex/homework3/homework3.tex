\documentclass{article}       % use "amsart" instead of "article" for AMSLaTeX format
\usepackage{geometry}                              % See geometry.pdf to learn the layout options. There are lots.
%\geo4metry{landscape}                            % Activate for for rotated page geometry
%\usepackage[parfill]{parskip}                  % Activate to begin paragraphs with an empty line rather than an indent
\usepackage{graphicx}                         % Use pdf, png, jpg, or eps§ with pdflatex; use eps in DVI mode
% TeX will automatically convert eps --> pdf in pdflatex         
\usepackage{amssymb}
\usepackage{amsmath}
\usepackage{mathtools}
\usepackage{booktabs}
\usepackage{tabto}
\usepackage{algorithm}
\usepackage{algpseudocode}
\usepackage{tikz}

\DeclarePairedDelimiter\ceil{\lceil}{\rceil}
\DeclarePairedDelimiter\floor{\lfloor}{\rfloor}


\usetikzlibrary{shapes.geometric}
\tikzset{
        buffer/.style={
                draw,
                shape border rotate=0,
                regular polygon,
                regular polygon sides=3,
                fill=white,
                node distance=2cm,
                minimum height=4em
        }
}
\tikzset{
        buffer2/.style={
                draw,
                shape border rotate=180,
                regular polygon,
                regular polygon sides=3,
                fill=white,
                node distance=2cm,
                minimum height=4em
        }
}



\title{Adv Data Struct $\&$ Algorithm: Homework 3}
\author{Zimian Li}

\begin{document}
        
\maketitle

\begin{enumerate}
        \item[1.] Suppose you have a biased source of random bits that in $O(1)$ time returns 1 with probability $p$, and 0 with probability $1-p$. However, you do not know the value of $p$.
        \begin{enumerate}
        	\item[(a)] Design an efficient algorithm that uses the defective source to return 1 with probability 1/2 and 0 with probability 1/2.
        	\begin{algorithm}[H]
        		\caption{Unbiased Random}
        		\small
        		\begin{algorithmic}[1]
        			\Procedure {UnbiasedRandom}{}
        			\While {true}
        			\State a = BiasedRandom()
        			\State b = BiasedRandom()
        			\If {a $\neq$ b}
        			\State \Return a
        			\EndIf
        			\EndWhile
        			\EndProcedure
        		\end{algorithmic}\label{p1}
        	\end{algorithm}	
        	I called BiasedRandom() twice, there are 4 kinds of result:\newline
        	1. $\{a=0, b=0\}$, with probability $(1-p)^2$;\newline
        	2. $\{a=0, b=1\}$, with probability $p(1-p)$;\newline
        	3. $\{a=1, b=0\}$, with probability $p(1-p)$;\newline
        	4. $\{a=1, b=1\}$, with probability $p^2$;\newline
        	And events $\{a=0, b=1\}$ and $\{a=1, b=0\}$ have the same probability, thus I decide to return $a$, it has $1/2$ probability to return 0, and $1/2$ probability to return 1.\newline
        	\item[(b)] As a function of $p$ what is the expected running time of your algorithm.\newline\newline
        	The probability of event $\{(a=0, b=1), (a=1, b=0)\}$ is $2p(1-p)$, so the expected running times of the while-loop is $1/(2p(1-p))$. And all other operations including getting a biased random bit is in $O(1)$ time. So the expected running time is $O(1/(2p(1-p)))$.\newline
        	\item[(c)] Discuss how to modify your algorithm to return a random value from an arbitrary bernoulli trial, i.e., a random variable that returns 1 with probability $q$ and 0 with probability $1-q$. (You may assume that $q$ is a rational number with a finite binary expansion.)\newline\newline
        	$q$ is a rational number, so it can be written as a fraction. Suppose $q = \frac{x}{y}, 0 \leq x \leq y, y \neq 0$.\newline
        	Let's get back to the original algorithm of (a) above. I ran BiasedRandom() twice there, but this time let's suppose I modified it to ran BiasedRandom() $n$ times, then I can get a string of 0s and 1s with length $n$. Think about all possible outputs of this string, it's obvious that there are $n \choose i$ of them with $i \cdot 1$ and $(n-i) \cdot 0$, $0 \leq i \leq n$. And all of them have the same probability $p^i(1-p)^{n-i}$.\newline
        	Ok, now I can construct an event with probability $q = \frac{x}{y}$:\newline
        	Step 1: Choose a right pair of $n$ and $i$ that can meet ${n \choose i} = y$. The easist way is let $n = y, i = 1$.\newline
        	Step 2: Construct the sample space, all 01 strings with length $y$ and only one "1".\newline
        	Step 3: Construct the event, select $x$  arbitrary strings from the sample space.\newline
          \begin{algorithm}[H]
        		\caption{Unbiased Random with probability $q = \frac{x}{y}$}
        		\small
        		\begin{algorithmic}[1]
        			\Procedure {UnbiasedRandomQ}{x, y}
        			\State Sample $\gets$ all 01 strings with length $y$ and only one "1"
        			\State Event $\gets$ Sample[0:x]
        			\While {true}
        			\State str = [], count = 0
        			\For {i $\gets$ 0 to y-1}
        			\State a = BiasedRandom()
        			\State str.append(a)
        			\If {a == 1}
        			\State count += 1
        			\EndIf
        			\If {count $>$ 1}
        			\State break
        			\EndIf 
        			\EndFor
        			\If {count == 1 and str is in Event}
        			\State \Return 1
        			\Else \If {count == 1 and str is not in Event}
        			\State \Return 0
        			\EndIf
        			\EndIf
        			\EndWhile
        			\EndProcedure
        		\end{algorithmic}\label{p1}
        	\end{algorithm}	
        \end{enumerate}
     \item[2.] Consider a regular binary search tree $T$ build on the elements of a set $S$. Without loss of generality, we may assume that $S = \{1,2, ..., n\}$. In general, there are many different binary search trees for S. The exact shape of one of these trees depends on the order in which the elements of S were inserted into T. Assume that the elements are inserted in random order. For $x \in S$ and fixed $T$, let $A_x$ denote the set of ancestors of $x$ in $T$. Clearly, the worst case of $d(x)$ is $n-1$. However, we are interested in $E[d(x)]$. Let $x _{\leq} = \{1,2, ..., x\}$ and $x_{\geq} = \{x,x+1, ..., n\}$. Let $Y$ and $Z$ be random variables defined as follows:\newline
     $Y = |x_{\leq} \cap A_x|, Z = |x_{\geq} \cap A_x|$\newline
 	You are asked to analyze the expected values of $Y$ and $Z$.
	\begin{enumerate}
		\item Express $d(x)$ in terms of random variables $Y$ and $Z$.\newline\newline
		I'm not sure if the ancestors of a node should contain itself or not. Let's just suppose $x \notin A_x$ here.\newline 
		$\because x_{\leq} \cup x_{\geq}  = S, x_{\leq} \cap x_{\geq}  = \{x\}, x \notin A_x$\newline
		$\therefore (x_{\leq} \cap A_x) \cup (x_{\geq} \cap A_x) = A_x, (x_{\leq} \cap A_x) \cap (x_{\geq} \cap A_x) = \emptyset$\newline
		$\therefore d(x) = |A_x| = |(x_{\leq} \cap A_x) \cup (x_{\geq} \cap A_x)| = |x_{\leq} \cap A_x| + |x_{\geq} \cap A_x| = Y + Z$\newline
		\item Let $y \in x_{\leq}$. Provide necessary and sufficient conditions for $y \in A_x$.\newline\newline
		$y$ was inserted before $x$.\newline
		\item For $y \in x_{\leq}$, let $I_y = I\{y\in A_x\}$. What is $E[I_y]$?\newline\newline
		$E[I_y] = Pr[I_y] = Pr[y\in A_x] = Pr[y$ was inserted before $x] = 0 \cdot \frac{1}{n} + \frac{1}{n-1} \cdot \frac{1}{n} + \frac{2}{n-1} \cdot \frac{1}{n} + ... +\frac{n-1}{n-1} \cdot \frac{1}{n}  = \frac{1+2+...+n-1}{n-1}\cdot \frac{1}{n} = \frac{n}{2} \cdot \frac{1}{n} = \frac{1}{2}$\newline
		\item Express $Y$ as a sum of indicator variables and compute $E[Y]$.\newline\newline
		$E[Y] = \sum^{y \in x_{\leq}} E[I_y] = x \cdot \frac{1}{2} = \frac{x}{2}$\newline
		\item Proceed similarly to compute $E[Z]$.\newline\newline
		Similarly, $E[I_z] = \frac{1}{2}$, and $E[Z] = \sum^{z \in x_{\geq}} E[I_z] = (n-x+1) \cdot \frac{1}{2} = \frac{n-x+1}{2}$\newline
		\item Finally, use the results above to estimate $E[d(x)]$.\newline\newline
		$E[d(x)] = E[Y] + E[Z] = \frac{x}{2} + \frac{n-x+1}{2} = \frac{n+1}{2}$
	\end{enumerate}
	\item[3.] Suppose that two remote computers $M_1$ and $M_2$ store databases $DB_1$ and $DB_2$, respectively. The two databases need to be synchronized and need to store exactly the same contents. Let $x = x_1x_2 ... x_n$ denote the binary contents of $DB_1$ and $y = y_1y_2 ... y_n$ the contents of $DB_2$. You want to determine if $x = y$. Since a deterministic algorithm would have to transfer $n$ bits to accomplish your goals, you decide, instead, to use the following randomized algorithm:
	\begin{enumerate}
		\item[\textbf{Step 1}] Machine $M_1$ chooses a prime $p \leq n^2$ at random with uniform probability and computes s = x mod p.
		\item[\textbf{Step 2}] Machine $M_1$ sends $s$ and $p$ to $M_2$.
		\item[\textbf{Step 3}] Machine $M_2$ receives $s$ and $p$ and computes t = y mod p.
		\item[\textbf{Step 4}] Machine $M_2$ compares $s$ with $t$ and concludes that $x = y$ if $s = t$; otherwise, it concludes that $x \neq y$.
	\end{enumerate}
	Suppose. for example, that $x = 01111$, and $y = 10110$. $M_1$ chooses $p = 5 \leq 5^2$ from $\{2,3,5,7,11,13,17,19,23\}$. Since x mod 5 = 15 mod 5 = 0, $M_1$ sends 0 and 5 to $M_2$. Upon computing y mod 5 = 22 mod 5 = 2, $M_2$ concludes that $x \neq y$, so the databases are not synchronized.
	\begin{enumerate}
		\item[(a)] Clearly, this randomized algorithm is of the Monte Carlo variety. Describe the circumstances under which it makes a wrong decision.\newline\newline
		Sometimes $s=t$ which means by mod $p$, $x$ and $y$ get the same remainder, but $x \neq y$.\newline
		For example,  $x = 01111$, and $y = 01010$. $M_1$ chooses $p = 5 \leq 5^2$ from $\{2,3,5,7,11,13,17,19,23\}$.  Since x mod 5 = 15 mod 5 = 0, $M_1$ sends 0 and 5 to $M_2$. Then $M_2$ computes y mod 5 = 10 mod 5 = 0, $M_2$ concludes that $x = y$, but actually $x\neq y$.\newline
		\item[(b)] What is the probability that the algorithm makes a mistake for each of the following pairs of "databases"?\newline
		\begin{tabular}{ | r | c | c | }
			\hline
			Set & x & y \\ \hline
			1 & 01111 & 10110 \\ \hline
			2 & 1000011110100001 & 0010011001011000 \\ \hline
			3 & 010011011100100110110100001101 & 100100011100110011001011001110 \\ \hline
		\end{tabular}\newline\newline
		If $x$ mod $p = y$ mod $p$, then $x = i \cdot p + r, y = j \cdot p + r$, assume $x >= y$, thus $x-y = (i-j)p$. So if $|x-y|$ is some times of $p$, it'll make mistakes.\newline
		For set 1, the prime number set is $\{2,3,5,7,11,13,17,19,23\}$, $|15-22| = 7$,
		hence, the probability is $\frac{1}{9}$.\newline
		For set 2, $|34721 - 9816|=24905=5\times 17 \times 293$, and I need to choose $p$ from $\leq 16^2=256$, there are 54 prime numbers. Hence, the probability is $\frac{2}{54} = \frac{1}{27}$.\newline
		For set 3, $|326266125-611529422|=285263297=11\times 1999\times 12973 \times 11 \times 1999 \times 12973$, and need to choose $p$ from $\leq 30^2 = 900$, and there are 154 prime numbers. The probability is $\frac{1}{154}$.\newline
		\item[(c)] How many bits does the randomized algorithm need to transmit in general? How much faster than the deterministic algorithm is the randomized algorithm when comparing databases of 100Mb.\newline\newline
		Suppose $x = x_1x_2 ... x_n$. Therefore $p \leq n^2$, $s = x$ mod $p < p \leq n^2$. In this randomized algorithm, I need to transmit $s$ and $p$. The size should be $O(2 \cdot (\floor{log(n^2)} + 1))$.\newline
		If the size of database is 100Mb, the deterministic algorithm should transmit 100Mb data, however the randomized algorithm should transmit no more than $2 \cdot (\floor{log((100Mb)^2)} + 1) \approx 26.6$Mb data.\newline
		Hence, the randomized algorithm is more than $3.8$ times faster than the deterministic algorithm.\newline
		\item[(d)] Show that the number of distinct primes that divide an n-bit integer is less than n.\newline\newline
		\textbf{Proof:} Obviously, the largest n-bit integer is $2^n - 1$. Suppose a positive integer $x$ and $x <= 2^n -1$, assume $x$ can be can divided by $m (m \geq n)$ distinct prime numbers.\newline
		Thus, $x = c \cdot p_1 \cdot p_2 \cdot ... \cdot p_m$, $p_1, p_2, ..., p_m$ are distinct prime numbers, $c$ is a positive integer. As we all know, 2 is the smallest prime number, so $x = c \cdot p_1 \cdot p_2 \cdot ... \cdot p_m > c \cdot 2^m$. Because $c \geq 1$ and $m \geq n$, therefore $x > c \cdot 2^m \geq 2^n$. But $x <= 2^n-1$, it's a contradiction!\newline
		Hence, the number of distinct primes that divide an n-bit integer is less than n.\newline
		\item[(e)] Compute the probability that the randomized algorithm makes a mistake for a database of n bits.\newline\newline
		The subtraction of two n-bit databases is a n-bit integer, and per the conclusion of (d), the number of distinct prime numbers that divide this integer is less than n.\newline
		According to the prime counting function $\pi (x) \sim \frac{x}{ln(x)}$. The number of prime numbers that at most $n^2$ is around $\frac{n^2}{2ln(n)}$.\newline
		Considering the prime number that can divide the n-bit integer could be more than $n^2$, the probability that the randomized algorithm makes a mistake for a database of n bits is $< n / (\frac{n^2}{2ln(n)}) = \frac{2ln(n)}{n}$.\newline
		
		\item[(f)] Given a maximum tolerance for error $0 < \varepsilon \leq 1$, adapt the proposed algorithm so that the probability of reaching a wrong conclusion is no more than $\varepsilon$.\newline\newline
		We need to repeat this algorithm more times, if all rounds of result return $x =y$, then return $x=y$, else if one of them return $x \neq y$, return $x \neq y$.\newline
		And the repeating number of times $i$ should meet $(\frac{2ln(n)}{n})^i \leq \varepsilon$, so $i \geq \log_{\frac{2ln(n)}{n}} \varepsilon$.\newline
    \end{enumerate}
	\item[4.] You are given a set of n items $a_1, ..., a_n$ with $a_i$ stored in location ${\ell}_i$ of a linked list A. To search for $a_i$ you scan the list from left to right. Thus, the time to find $a_i$ is proportional to ${\ell}_i$. Let $p_i$ be the probability that the item searched for is $a_i$. An arrangement, i.e., a choice of ${\ell}_i$'s, is optimal if it minimizes the average search cost.
	\begin{enumerate}
		\item[(a)] What is the average search cost?\newline\newline
		Average search cost = ${\ell}_1 \cdot p_1+{\ell}_2 \cdot p_2+...+{\ell}_n \cdot p_n = \sum_{1}^n ({\ell}_i \cdot p_i)$.\newline
		\item[(b)] Suppose that you are allowed to rearrange the items as you please and that you know the access probabilities $p_1, ..., p_n$. What list arrangement minimizes the average search cost? Prove your claim.\newline\newline
		Put the item with the highest access probability to the first location, then the second highest to the second location, and so forth, until put the item with smallest access probability to the last location.\newline
		Suppose $p_1 \geq p_2 \geq ... \geq p_n$, $p_1 + p_2 + ... + p_n = 1$, the minimal average search cost should be $p_1+2p_2+ ... + np_n$.\newline
		\textbf{Proof:} Choose two arbitrary items of $p_1+2p_2+ ... + np_n$, suppose them be $ip_i$ and $jp_j$, and $i < j$, $p_i \geq p_j$.\newline
		Swap $i$ and $j$, then get $jp_i$ and $ip_j$. Compare them with the original ones. $(jp_i+ip_j)-(ip_i+jp_j) = (j-i)(p_i-p_j) \geq 0$.\newline
		Therefore, every swapping of factors of $p_i$ and $p_j$ will increase the whole value. So the current value is the minimal.\newline
		\item[(c)] We assume now that there is an underlying access probability distribution that this is not known to you. To obtain a reasonable search time you start with a randomly arranged list and decide that every time an element is accessed you will move it to the head of the list, leaving the rest if the items in the same relative ordering.
		\begin{enumerate}
			\item[i.] Suppose that $a_i$ and $a_j$ were accessed in the past. What is the probability that ${\ell}_i < {\ell}_j$?\newline\newline
			Suppose $c_i$ is the number of $a_i$ were accessed, $c_j$ is the $a_j$ were accessed.\newline
			Pr(${\ell}_i < {\ell}_j$) = $c_i/(c_i+c_j)$\newline
			\item[ii.] Define decision variables $X_{ij} = I(a_i$ appears before $a_j)$. Express ${\ell}_i$ in terms of these decision variables and find $E({\ell}_i)$.\newline\newline
			${\ell}_i = E(X_{1i}) + E(X_{2i}) + ... + E(X_{ni}) = \sum_{j=1,j\neq i}^{n} E(X_{ji})$\newline
			Suppose $c$ be the total access number.\newline
			$E({\ell}_i) = \sum_{i=1}^{n} c_i{\ell}_i/c$\newline
			\item[iii.] Assume that the algorithm has been running for a while. What is the expected search cost? (It is ok to analyze the performance in terms of a probability distribution that is not known to the algorithm).\newline\newline
			$E({\ell}_i) = \sum_{i=1}^{n} c_i{\ell}_i/c$\newline
			
			\item[iv.] Compare your answer to that of part (b) and explain why you have, in fact, designed a good approximation algorithm.\newline\newline
			${\ell}_i = E(X_{1i}) + E(X_{2i}) + ... + E(X_{ni}) = \frac{c_1}{c_i+c_1} + \frac{c_2}{c_i+c_2} + ... + \frac{c_n}{c_i+c_n}$.\newline
			It's obvious that ${\ell}_i$ is inversely proportional to $c_i$.\newline
			$E({\ell}_i) = \sum_{i=1}^{n} (c_i/c)\cdot {\ell}_i$\newline
			When $(c_i/c)$ gets larger, ${\ell}_i$ gets smaller. Actually, it's similar with (b) part. That means the expected search cost of this algorithm is minimal.
		\end{enumerate}	
	\end{enumerate}
                
\end{enumerate}
\end{document}
